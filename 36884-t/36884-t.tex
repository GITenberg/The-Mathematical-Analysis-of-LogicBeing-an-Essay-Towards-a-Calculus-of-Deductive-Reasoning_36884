% %%%%%%%%%%%%%%%%%%%%%%%%%%%%%%%%%%%%%%%%%%%%%%%%%%%%%%%%%%%%%%%%%%%%%%% %
%                                                                         %
% Project Gutenberg's The Mathematical Analysis of Logic, by George Boole %
%                                                                         %
% This eBook is for the use of anyone anywhere at no cost and with        %
% almost no restrictions whatsoever.  You may copy it, give it away or    %
% re-use it under the terms of the Project Gutenberg License included     %
% with this eBook or online at www.gutenberg.org                          %
%                                                                         %
%                                                                         %
% Title: The Mathematical Analysis of Logic                               %
%        Being an Essay Towards a Calculus of Deductive Reasoning         %
%                                                                         %
% Author: George Boole                                                    %
%                                                                         %
% Release Date: July 28, 2011 [EBook #36884]                              %
%                                                                         %
% Language: English                                                       %
%                                                                         %
% Character set encoding: ISO-8859-1                                      %
%                                                                         %
% *** START OF THIS PROJECT GUTENBERG EBOOK THE MATHEMATICAL ANALYSIS OF LOGIC ***
%                                                                         %
% %%%%%%%%%%%%%%%%%%%%%%%%%%%%%%%%%%%%%%%%%%%%%%%%%%%%%%%%%%%%%%%%%%%%%%% %

\def\ebook{36884}
%%%%%%%%%%%%%%%%%%%%%%%%%%%%%%%%%%%%%%%%%%%%%%%%%%%%%%%%%%%%%%%%%%%%%%
%%                                                                  %%
%% Packages and substitutions:                                      %%
%%                                                                  %%
%% book:     Required.                                              %%
%% inputenc: Latin-1 text encoding. Required.                       %%
%% babel:    Greek language capabilities. Required.                 %%
%%                                                                  %%
%% ifthen:   Logical conditionals. Required.                        %%
%%                                                                  %%
%% amsmath:  AMS mathematics enhancements. Required.                %%
%% amssymb:  Additional mathematical symbols. Required.             %%
%%                                                                  %%
%% alltt:    Fixed-width font environment. Required.                %%
%% array:    Enhanced tabular features. Required.                   %%
%%                                                                  %%
%% indentfirst: Indent first line of section. Required.             %%
%% footmisc: Start footnote numbering on each page. Required.       %%
%%                                                                  %%
%% caption:  Caption customization for table. Required.             %%
%%                                                                  %%
%% calc:     Length calculations. Required.                         %%
%%                                                                  %%
%% fancyhdr: Enhanced running headers and footers. Required.        %%
%%                                                                  %%
%% geometry: Enhanced page layout package. Required.                %%
%% hyperref: Hypertext embellishments for pdf output. Required.     %%
%%                                                                  %%
%%                                                                  %%
%% Producer's Comments:                                             %%
%%                                                                  %%
%%   OCR text for this ebook was obtained on July 25, 2011, from    %%
%%   http://www.archive.org/details/mathematicalanal00booluoft.     %%
%%                                                                  %%
%%   Minor changes to the original are noted in this file in three  %%
%%   ways:                                                          %%
%%     1. \Typo{}{} for typographical corrections, showing original %%
%%        and replacement text side-by-side.                        %%
%%     2. \Chg{}{} and \Add{}, for inconsistent/missing punctuation %%
%%        and capitalization.                                       %%
%%     3. [** TN: Note]s for lengthier or stylistic comments.       %%
%%                                                                  %%
%%                                                                  %%
%% Compilation Flags:                                               %%
%%                                                                  %%
%%   The following behavior may be controlled by boolean flags.     %%
%%                                                                  %%
%%   ForPrinting (false by default):                                %%
%%   If false, compile a screen optimized file (one-sided layout,   %%
%%   blue hyperlinks). If true, print-optimized PDF file: Larger    %%
%%   text block, two-sided layout, black hyperlinks.                %%
%%                                                                  %%
%%                                                                  %%
%% PDF pages: 101(if ForPrinting set to false)                      %%
%% PDF page size: 5.5 x 7.5" (non-standard)                         %%
%%                                                                  %%
%% Summary of log file:                                             %%
%% * No warnings                                                    %%
%%                                                                  %%
%% Compile History:                                                 %%
%%                                                                  %%
%% July, 2011: (Andrew D. Hwang)                                    %%
%%             texlive2007, GNU/Linux                               %%
%%                                                                  %%
%% Command block:                                                   %%
%%                                                                  %%
%%     pdflatex x2                                                  %%
%%                                                                  %%
%%                                                                  %%
%% July 2011: pglatex.                                              %%
%%   Compile this project with:                                     %%
%%   pdflatex 36884-t.tex ..... TWO times                           %%
%%                                                                  %%
%%   pdfTeXk, Version 3.141592-1.40.3 (Web2C 7.5.6)                 %%
%%                                                                  %%
%%%%%%%%%%%%%%%%%%%%%%%%%%%%%%%%%%%%%%%%%%%%%%%%%%%%%%%%%%%%%%%%%%%%%%
\listfiles
\documentclass[12pt]{book}[2005/09/16]

%%%%%%%%%%%%%%%%%%%%%%%%%%%%% PACKAGES %%%%%%%%%%%%%%%%%%%%%%%%%%%%%%%
\usepackage[latin1]{inputenc}[2006/05/05]

\usepackage[greek,english]{babel}[2005/11/23]
\languageattribute{greek}{polutoniko}

\usepackage{ifthen}[2001/05/26]  %% Logical conditionals

\usepackage{amsmath}[2000/07/18] %% Displayed equations
\usepackage{amssymb}[2002/01/22] %% and additional symbols

\usepackage{alltt}[1997/06/16]   %% boilerplate, credits, license
\usepackage{array}[2005/08/23]   %% extended array/tabular features

\usepackage{indentfirst}[1995/11/23]
\usepackage[perpage,symbol]{footmisc}[2005/03/17]

\usepackage[labelformat=empty,labelfont=small]{caption}[2007/01/07]

\usepackage{calc}[2005/08/06]

\usepackage{fancyhdr} %% For running heads

%%%%%%%%%%%%%%%%%%%%%%%%%%%%%%%%%%%%%%%%%%%%%%%%%%%%%%%%%%%%%%%%%
%%%% Interlude:  Set up PRINTING (default) or SCREEN VIEWING %%%%
%%%%%%%%%%%%%%%%%%%%%%%%%%%%%%%%%%%%%%%%%%%%%%%%%%%%%%%%%%%%%%%%%

% ForPrinting=true                     false (default)
% Asymmetric margins                   Symmetric margins
% 1 : 1.62 text block aspect ratio     3 : 4 text block aspect ratio
% Black hyperlinks                     Blue hyperlinks
% Start major marker pages recto       No blank verso pages
%
\newboolean{ForPrinting}

%% UNCOMMENT the next line for a PRINT-OPTIMIZED VERSION of the text %%
%\setboolean{ForPrinting}{true}

%% Initialize values to ForPrinting=false
\newcommand{\Margins}{hmarginratio=1:1}     % Symmetric margins
\newcommand{\HLinkColor}{blue}              % Hyperlink color
\newcommand{\PDFPageLayout}{SinglePage}
\newcommand{\TransNote}{Transcriber's Note}
\newcommand{\TransNoteCommon}{%
  The camera-quality files for this public-domain ebook may be
  downloaded \textit{gratis} at
  \begin{center}
    \texttt{www.gutenberg.org/ebooks/\ebook}.
  \end{center}

  This ebook was produced using scanned images and OCR text generously
  provided by the University of Toronto McLennan Library through the
  Internet Archive.
  \bigskip

  Minor typographical corrections and presentational changes have been
  made without comment. Punctuation has been regularized, but may be
  easily reverted to match the original; changes are documented in the
  \LaTeX\ source file.
  \bigskip
}

\newcommand{\TransNoteText}{%
  \TransNoteCommon

  This PDF file is optimized for screen viewing, but may be recompiled
  for printing. Please consult the preamble of the \LaTeX\ source file
  for instructions and other particulars.
}
%% Re-set if ForPrinting=true
\ifthenelse{\boolean{ForPrinting}}{%
  \renewcommand{\Margins}{hmarginratio=2:3} % Asymmetric margins
  \renewcommand{\HLinkColor}{black}         % Hyperlink color
  \renewcommand{\PDFPageLayout}{TwoPageRight}
  \renewcommand{\TransNote}{Transcriber's Note}
  \renewcommand{\TransNoteText}{%
    \TransNoteCommon

    This PDF file is optimized for printing, but may be recompiled for
    screen viewing. Please consult the preamble of the \LaTeX\ source
    file for instructions and other particulars.
  }
}{% If ForPrinting=false, don't skip to recto
  \renewcommand{\cleardoublepage}{\clearpage}
}
%%%%%%%%%%%%%%%%%%%%%%%%%%%%%%%%%%%%%%%%%%%%%%%%%%%%%%%%%%%%%%%%%
%%%%  End of PRINTING/SCREEN VIEWING code; back to packages  %%%%
%%%%%%%%%%%%%%%%%%%%%%%%%%%%%%%%%%%%%%%%%%%%%%%%%%%%%%%%%%%%%%%%%

\ifthenelse{\boolean{ForPrinting}}{%
  \setlength{\paperwidth}{8.5in}%
  \setlength{\paperheight}{11in}%
% ~1:1.67
  \usepackage[body={5.25in,8.75in},\Margins]{geometry}[2002/07/08]
}{%
  \setlength{\paperwidth}{5.5in}%
  \setlength{\paperheight}{7.5in}%
  \raggedbottom
% ~3:4
  \usepackage[body={5.25in,6.6in},\Margins,includeheadfoot]{geometry}[2002/07/08]
}

\providecommand{\ebook}{00000}    % Overridden during white-washing
\usepackage[pdftex,
  hyperref,
  hyperfootnotes=false,
  pdftitle={The Project Gutenberg eBook \#\ebook: The Mathematical Analysis of Logic},
  pdfauthor={George Boole},
  pdfkeywords={University of Toronto, The Internet Archive, Andrew D. Hwang},
  pdfstartview=Fit,    % default value
  pdfstartpage=1,      % default value
  pdfpagemode=UseNone, % default value
  bookmarks=true,      % default value
  linktocpage=false,   % default value
  pdfpagelayout=\PDFPageLayout,
  pdfdisplaydoctitle,
  pdfpagelabels=true,
  bookmarksopen=true,
  bookmarksopenlevel=0,
  colorlinks=true,
  linkcolor=\HLinkColor]{hyperref}[2007/02/07]


%% Fixed-width environment to format PG boilerplate %%
\newenvironment{PGtext}{%
\begin{alltt}
%\fontsize{9.2}{11}\ttfamily\selectfont}%
\fontsize{10}{12}\ttfamily\selectfont}%
{\end{alltt}}

% Errors found during digitization
\newcommand{\Typo}[2]{#2}

% Stylistic changes made for consistency
\newcommand{\Chg}[2]{#2}
%\newcommand{\Chg}[2]{#1} % Use this to revert inconsistencies in the original
\newcommand{\Add}[1]{\Chg{}{#1}}

%% Miscellaneous global parameters %%
% No hrule in page header
\renewcommand{\headrulewidth}{0pt}

% Match array row separation to AMS environments
\setlength{\extrarowheight}{3pt}

% Scratch pad for length calculations
\newlength{\TmpLen}

%% Running heads %%
\newcommand{\FlushRunningHeads}{\clearpage\fancyhf{}}
\newcommand{\InitRunningHeads}{%
  \setlength{\headheight}{15pt}
  \pagestyle{fancy}
  \thispagestyle{empty}
  \ifthenelse{\boolean{ForPrinting}}
             {\fancyhead[RO,LE]{\thepage}}
             {\fancyhead[R]{\thepage}}
}

\newcommand{\SetRunningHeads}[1]{%
  \fancyhead[C]{\textsc{\MakeLowercase{#1}}}
}

\newcommand{\BookMark}[2]{\phantomsection\pdfbookmark[#1]{#2}{#2}}

%% Major document divisions %%
\newcommand{\PGBoilerPlate}{%
  \pagenumbering{Alph}
  \pagestyle{empty}
  \BookMark{0}{PG Boilerplate.}
}
\newcommand{\FrontMatter}{%
  \cleardoublepage
  \frontmatter
}
\newcommand{\MainMatter}{%
  \FlushRunningHeads
  \InitRunningHeads
  \mainmatter
}
\newcommand{\PGLicense}{%
  \FlushRunningHeads
  \pagenumbering{Roman}
  \InitRunningHeads
  \BookMark{0}{PG License.}
  \SetRunningHeads{License.}
}

%% Sectional units %%
% Typographical abstraction
\newcommand{\ChapHead}[1]{%
  \section*{\centering\normalfont\normalsize\MakeUppercase{#1}}
}

% To refer internally to chapters by number
\newcounter{ChapNo}

% Cross-ref: \ChapRef{number}{Title}
\newcommand{\ChapRef}[2]{\hyperref[chap:#1]{\textit{#2}}}

% \Chapter{Title}
\newcommand{\Chapter}[1]{%
  \FlushRunningHeads
  \InitRunningHeads
  \BookMark{0}{#1}
  \refstepcounter{ChapNo}\label{chap:\theChapNo}
  \SetRunningHeads{#1}
%[** TN: Project-dependent behavior]
  \ifthenelse{\equal{#1}{Introduction.}}{%
    \begin{center}
      \textbf{MATHEMATICAL ANALYSIS OF LOGIC.} \\
      \bigskip
      \tb
    \end{center}
    \ChapHead{\MakeUppercase{#1}}
  }{%
    \ChapHead{\MakeUppercase{#1}}
    \begin{center}
      \tb
    \end{center}
  }
}

\newcommand{\Section}[1]{
  \subsection*{\centering\normalsize\normalfont\textit{#1}}
}

\newcommand{\Signature}[1]{\nopagebreak[4]\bigskip

  {\small #1}
}


% Smaller text at the start of four chapters
\newenvironment{Abstract}{\small}{\normalsize\medskip}

% One-off environment for title page Greek quote
\newenvironment{Quote}{\begin{minipage}{\textwidth}
\normalfont\hspace*{1.5em}
\selectlanguage{greek}}{\end{minipage}}

% Italicized theorem-like structure; may start in-line or have a run-in heading
\newenvironment{Rule}[1][Rule. ]{%
  \textsc{#1}\itshape\ignorespaces}{\upshape\par}

% Cross-ref-able proposition
\newcommand{\Prop}[1]{%
  \textsc{Prop.~#1}\phantomsection\label{prop:#1}%
}

\newcommand{\PropRef}[1]{\hyperref[prop:#1.]{Prop.~#1}}

\newcommand{\Pagelabel}[1]{\phantomsection\label{page:#1}}
\newcommand{\Pageref}[1]{\hyperref[page:#1]{p.~\pageref*{page:#1}}}
\newcommand{\Pagerefs}[2]{%
  \ifthenelse{\equal{\pageref*{page:#1}}{\pageref*{page:#2}}}{%
    \hyperref[page:#1]{p.~\pageref*{page:#1}}%
  }{% Else
    pp.~\hyperref[page:#1]{\pageref*{page:#1}},~\hyperref[page:#2]{\pageref*{page:#2}}%
  }%
}

% Page separators
\newcommand{\PageSep}[1]{\ignorespaces}

% Miscellaneous textual conveniences (N.B. \emph, not \textit)
\newcommand{\eg}{\emph{e.\,g.}}
\newcommand{\ie}{\emph{i.\,e.}}
\newcommand{\etc}{\text{\&c}}

\renewcommand{\(}{{\upshape(}}
\renewcommand{\)}{{\upshape)}}

\newcommand{\First}[1]{\textsc{#1}}

% Decorative rule
\newcommand{\tb}[1][0.75in]{\rule{#1}{0.5pt}}

%% Braces for alignments; smaller than AMS defaults
% \Rbrace{2} spans two lines
\newcommand{\Rbrace}[1]{%
  \makebox[8pt][l]{%
    $\left.\rule[4pt*#1]{0pt}{4pt*#1}\right\}$%
  }\
}
\newcommand{\Lbrace}[1]{%
  \makebox[4pt][r]{%
    $\left\{\rule[4pt*#1]{0pt}{4pt*#1}\right.$%
  }\!\!%
}

% Small-type column headings for alignments
\newcommand{\ColHead}[1]{\text{\footnotesize#1}}

%% Miscellaneous mathematical formatting %%
\DeclareInputMath{183}{\cdot}

%% Selected upright capital letters in math mode
\DeclareMathSymbol{A}{\mathalpha}{operators}{`A}
\DeclareMathSymbol{B}{\mathalpha}{operators}{`B}
\DeclareMathSymbol{C}{\mathalpha}{operators}{`C}
\DeclareMathSymbol{D}{\mathalpha}{operators}{`D}
\DeclareMathSymbol{E}{\mathalpha}{operators}{`E}
\DeclareMathSymbol{I}{\mathalpha}{operators}{`I}
\DeclareMathSymbol{O}{\mathalpha}{operators}{`O}

\DeclareMathSymbol{V}{\mathalpha}{operators}{`V}
\DeclareMathSymbol{W}{\mathalpha}{operators}{`W}
\DeclareMathSymbol{X}{\mathalpha}{operators}{`X}
\DeclareMathSymbol{Y}{\mathalpha}{operators}{`Y}
\DeclareMathSymbol{Z}{\mathalpha}{operators}{`Z}

\renewcommand{\epsilon}{\varepsilon}

% \PadTo[alignment]{width text}{visible text}
\newcommand{\PadTo}[3][c]{%
  \settowidth{\TmpLen}{$#2$}%
  \makebox[\TmpLen][#1]{$#3$}%
}
\newcommand{\PadTxt}[3][c]{%
  \settowidth{\TmpLen}{#2}%
  \makebox[\TmpLen][#1]{#3}%
}

% Cross-ref-able Arabic equation tags...
\newcommand{\Tag}[2][eqn]{\phantomsection\label{#1:#2}\tag*{\ensuremath{#2}}}
\newcommand{\Eqref}[2][eqn]{\hyperref[#1:#2]{\ensuremath{#2}}}

% ...and Greek equation tags
\newcommand{\GrTag}[2][]{%
  \phantomsection\label{eqn:#1}
  \tag*{\ensuremath{#2}}
}
\newcommand{\GrEq}[2][]{\hyperref[eqn:#1]{\ensuremath{#2}}}

% Boole uses (a) and (b) as "local" tags; no need to cross-ref
\newcommand{\atag}{\rlap{\quad$(a)$}}
\newcommand{\aref}{$(a)$}

\newcommand{\btag}{\rlap{\quad$(b)$}}
\newcommand{\bref}{$(b)$}

% "Label" tag: Other tag-like labels on displayed equations; no cross-refs
\newcommand{\Ltag}[1]{%
  \ifthenelse{\equal{#1}{I}}{%
    \tag*{#1\,\qquad} % Pad "I" on the right
  }{
    \tag*{#1\qquad}
  }
}
%%%%%%%%%%%%%%%%%%%%%%%% START OF DOCUMENT %%%%%%%%%%%%%%%%%%%%%%%%%%
\begin{document}
%% PG BOILERPLATE %%
\PGBoilerPlate
\begin{center}
\begin{minipage}{\textwidth}
\small
\begin{PGtext}
Project Gutenberg's The Mathematical Analysis of Logic, by George Boole

This eBook is for the use of anyone anywhere at no cost and with
almost no restrictions whatsoever.  You may copy it, give it away or
re-use it under the terms of the Project Gutenberg License included
with this eBook or online at www.gutenberg.org


Title: The Mathematical Analysis of Logic
       Being an Essay Towards a Calculus of Deductive Reasoning

Author: George Boole

Release Date: July 28, 2011 [EBook #36884]

Language: English

Character set encoding: ISO-8859-1

*** START OF THIS PROJECT GUTENBERG EBOOK THE MATHEMATICAL ANALYSIS OF LOGIC ***
\end{PGtext}
\end{minipage}
\end{center}
\newpage
%% Credits and transcriber's note %%
\begin{center}
\begin{minipage}{\textwidth}
\begin{PGtext}
Produced by Andrew D. Hwang
\end{PGtext}
\end{minipage}
\vfill
\end{center}

\begin{minipage}{0.85\textwidth}
\small
\BookMark{0}{Transcriber's Note.}
\subsection*{\centering\normalfont\scshape%
\normalsize\MakeLowercase{\TransNote}}%

\raggedright
\TransNoteText
\end{minipage}
%%%%%%%%%%%%%%%%%%%%%%%%%%% FRONT MATTER %%%%%%%%%%%%%%%%%%%%%%%%%%
\PageSep{i}
\FrontMatter
\begin{center}
\bfseries\large THE MATHEMATICAL ANALYSIS
\vfil

\Large OF LOGIC,
\vfil

\normalsize
BEING AN ESSAY TOWARDS A CALCULUS \\
OF DEDUCTIVE REASONING.
\vfil

BY GEORGE BOOLE.
\vfil

\begin{Quote}
>Epikoinwno~usi d`e p~asai a<i >epist~hmai >all'hlais kat`a t`a koin'a. \Typo{Koin'a}{Koin`a} d`e
l'egw, o>~is qr~wntai <ws >ek to'utwn >apodeikn'untes; >all'' o>u per`i <~wn deikn'uousin,
\Typo{o>ude}{o>ud`e} <`o deikn'uousi. \\
\selectlanguage{english}
\null\hfill\textsc{Aristotle}, \textit{Anal.\ Post.}, lib.~\textsc{i}. cap.~\textsc{xi}.
\end{Quote}
\vfil\vfil

CAMBRIDGE: \\
MACMILLAN, BARCLAY, \& MACMILLAN; \\
LONDON: GEORGE BELL. \\
\tb[0.25in] \\
1847
\normalfont
\end{center}
\PageSep{ii}
\newpage
\normalfont
\null
\vfill
\begin{center}
\scriptsize
PRINTED IN ENGLAND BY \\
HENDERSON \& SPALDING \\
LONDON. W.I
\end{center}
\PageSep{1}
\MainMatter


\Chapter{Preface.}

\First{In} presenting this Work to public notice, I deem it not
irrelevant to observe, that speculations similar to those which
it records have, at different periods, occupied my thoughts.
In the spring of the present year my attention was directed
to the question then moved between Sir W.~Hamilton and
Professor De~Morgan; and I was induced by the interest
which it inspired, to resume the almost-forgotten thread of
former inquiries. It appeared to me that, although Logic
might be viewed with reference to the idea of quantity,\footnote
  {See \Pageref{42}.}
it
had also another and a deeper system of relations. If it was
lawful to regard it from \emph{without}, as connecting itself through
the medium of Number with the intuitions of Space and Time,
it was lawful also to regard it from \emph{within}, as based upon
facts of another order which have their abode in the constitution
of the Mind. The results of this view, and of the
inquiries which it suggested, are embodied in the following
Treatise.

It is not generally permitted to an Author to prescribe
the mode in which his production shall be judged; but there
are two conditions which I may venture to require of those
who shall undertake to estimate the merits of this performance.
The first is, that no preconceived notion of the impossibility
of its objects shall be permitted to interfere with that candour
and impartiality which the investigation of Truth demands;
the second is, that their judgment of the system as a whole
shall not be founded either upon the examination of only
\PageSep{2}
a part of it, or upon the measure of its conformity with any
received system, considered as a standard of reference from
which appeal is denied. It is in the general theorems which
occupy the latter chapters of this work,---results to which there
is no existing counterpart,---that the claims of the method, as
a Calculus of Deductive Reasoning, are most fully set forth.

What may be the final estimate of the value of the system,
I have neither the wish nor the right to anticipate. The
estimation of a theory is not simply determined by its truth\Add{.}
It also depends upon the importance of its subject, and the
extent of its applications; beyond which something must still
be left to the arbitrariness of human Opinion. If the utility
of the application of Mathematical forms to the science of
Logic were solely a question of Notation, I should be content
to rest the defence of this attempt upon a principle which has
been stated by an able living writer: ``Whenever the nature
of the subject permits the reasoning process to be without
danger carried on mechanically, the language should be constructed
on as mechanical principles as possible; while in the
contrary case it should be so constructed, that there shall be
the greatest possible obstacle to a mere mechanical use of it.''\footnote
  {Mill's \textit{System of Logic, Ratiocinative and Inductive}, Vol.~\textsc{ii}. p.~292.}
In one respect, the science of Logic differs from all others;
the perfection of its method is chiefly valuable as an evidence
of the speculative truth of its principles. To supersede the
employment of common reason, or to subject it to the rigour
of technical forms, would be the last desire of one who knows
the value of that intellectual toil and warfare which imparts
to the mind an athletic vigour, and teaches it to contend
with difficulties and to rely upon itself in emergencies.
\Signature{\textsc{Lincoln}, \textit{Oct.}~29, 1847.}
\PageSep{3}


%[**TN: Macro prints heading "MATHEMATICAL ANALYSIS OF LOGIC."]
\Chapter{Introduction.}

\First{They} who are acquainted with the present state of the theory
of Symbolical Algebra, are aware, that the validity of the
processes of analysis does not depend upon the interpretation
of the symbols which are employed, but solely upon the laws
of their combination. Every system of interpretation which
does not affect the truth of the relations supposed, is equally
admissible, and it is thus that the same process may, under
one scheme of interpretation, represent the solution of a question
on the properties of numbers, under another, that of
a geometrical problem, and under a third, that of a problem
of dynamics or optics. This principle is indeed of fundamental
importance; and it may with safety be affirmed, that the recent
advances of pure analysis have been much assisted by the
influence which it has exerted in directing the current of
investigation.

But the full recognition of the consequences of this important
doctrine has been, in some measure, retarded by accidental
circumstances. It has happened in every known form of
analysis, that the elements to be determined have been conceived
as measurable by comparison with some fixed standard.
The predominant idea has been that of magnitude, or more
strictly, of numerical ratio. The expression of magnitude, or
\PageSep{4}
of operations upon magnitude, has been the express object
for which the symbols of Analysis have been invented, and
for which their laws have been investigated. Thus the abstractions
of the modern Analysis, not less than the ostensive
diagrams of the ancient Geometry, have encouraged the notion,
that Mathematics are essentially, as well as actually, the Science
of Magnitude.

The consideration of that view which has already been stated,
as embodying the true principle of the Algebra of Symbols,
would, however, lead us to infer that this conclusion is by no
means necessary. If every existing interpretation is shewn to
involve the idea of magnitude, it is only by induction that we
can assert that no other interpretation is possible. And it may
be doubted whether our experience is sufficient to render such
an induction legitimate. The history of pure Analysis is, it may
be said, too recent to permit us to set limits to the extent of its
applications. Should we grant to the inference a high degree
of probability, we might still, and with reason, maintain the
sufficiency of the definition to which the principle already stated
would lead us. We might justly assign it as the definitive
character of a true Calculus, that it is a method resting upon
the employment of Symbols, whose laws of combination are
known and general, and whose results admit of a consistent
interpretation. That to the existing forms of Analysis a quantitative
interpretation is assigned, is the result of the circumstances
by which those forms were determined, and is not to
be construed into a universal condition of Analysis. It is upon
the foundation of this general principle, that I purpose to
establish the Calculus of Logic, and that I claim for it a place
among the acknowledged forms of Mathematical Analysis, regardless
that in its object and in its instruments it must at
present stand alone.

That which renders Logic possible, is the existence in our
minds of general notions,---our ability to conceive of a class,
and to designate its individual members by a common name.
\PageSep{5}
\Pagelabel{5}%
The theory of Logic is thus intimately connected with that of
Language. A successful attempt to express logical propositions
by symbols, the laws of whose combinations should be founded
upon the laws of the mental processes which they represent,
would, so far, be a step toward a philosophical language. But
this is a view which we need not here follow into detail.\footnote
  {This view is well expressed in one of Blanco White's Letters:---``Logic is
  for the most part a collection of technical rules founded on classification. The
  Syllogism is nothing but a result of the classification of things, which the mind
  naturally and necessarily forms, in forming a language. All abstract terms are
  classifications; or rather the labels of the classes which the mind has settled.''---\textit{Memoirs
  of the Rev.\ Joseph Blanco White}, vol.~\textsc{ii}. p.~163. See also, for a very
  lucid introduction, Dr.~Latham's \textit{First Outlines of Logic applied to Language},
  Becker's \textit{German Grammar,~\etc.} Extreme Nominalists make Logic entirely
  dependent upon language. For the opposite view, see Cudworth's \textit{Eternal
  and Immutable Morality}, Book~\textsc{iv}. Chap.~\textsc{iii}.}
Assuming the notion of a class, we are able, from any conceivable
collection of objects, to separate by a mental act, those
which belong to the given class, and to contemplate them apart
from the rest. Such, or a similar act of election, we may conceive
to be repeated. The group of individuals left under consideration
may be still further limited, by mentally selecting
those among them which belong to some other recognised class,
as well as to the one before contemplated. And this process
may be repeated with other elements of distinction, until we
arrive at an individual possessing all the distinctive characters
which we have taken into account, and a member, at the same
time, of every class which we have enumerated. It is in fact
a method similar to this which we employ whenever, in common
language, we accumulate descriptive epithets for the sake of
more precise definition.

Now the several mental operations which in the above case
we have supposed to be performed, are subject to peculiar laws.
It is possible to assign relations among them, whether as respects
the repetition of a given operation or the succession of
different ones, or some other particular, which are never violated.
It is, for example, true that the result of two successive acts is
\PageSep{6}
unaffected by the order in which they are performed; and there
are at least two other laws which will be pointed out in the
proper place. These will perhaps to some appear so obvious as
to be ranked among necessary truths, and so little important
as to be undeserving of special notice. And probably they are
noticed for the first time in this Essay. Yet it may with confidence
be asserted, that if they were other than they are, the
entire mechanism of reasoning, nay the very laws and constitution
of the human intellect, would be vitally changed. A Logic
might indeed exist, but it would no longer be the Logic we
possess.

Such are the elementary laws upon the existence of which,
and upon their capability of exact symbolical expression, the
method of the following Essay is founded; and it is presumed
that the object which it seeks to attain will be thought to
have been very fully accomplished. Every logical proposition,
whether categorical or hypothetical, will be found to be capable
of exact and rigorous expression, and not only will the laws of
conversion and of syllogism be thence deducible, but the resolution
of the most complex systems of propositions, the separation
of any proposed element, and the expression of its value in
terms of the remaining elements, with every subsidiary relation
involved. Every process will represent deduction, every
mathematical consequence will express a logical inference. The
generality of the method will even permit us to express arbitrary
operations of the intellect, and thus lead to the demonstration
of general theorems in logic analogous, in no slight
degree, to the general theorems of ordinary mathematics. No
inconsiderable part of the pleasure which we derive from the
application of analysis to the interpretation of external nature,
arises from the conceptions which it enables us to form of the
universality of the dominion of law. The general formul� to
which we are conducted seem to give to that element a visible
presence, and the multitude of particular cases to which they
apply, demonstrate the extent of its sway. Even the symmetry
\PageSep{7}
of their analytical expression may in no fanciful sense be
deemed indicative of its harmony and its consistency. Now I
do not presume to say to what extent the same sources of
pleasure are opened in the following Essay. The measure of
that extent may be left to the estimate of those who shall think
the subject worthy of their study. But I may venture to
assert that such occasions of intellectual gratification are not
here wanting. The laws we have to examine are the laws of
one of the most important of our mental faculties. The mathematics
we have to construct are the mathematics of the human
intellect. Nor are the form and character of the method, apart
from all regard to its interpretation, undeserving of notice.
There is even a remarkable exemplification, in its general
theorems, of that species of excellence which consists in freedom
from exception. And this is observed where, in the corresponding
cases of the received mathematics, such a character
is by no means apparent. The few who think that there is that
in analysis which renders it deserving of attention for its own
sake, may find it worth while to study it under a form in which
every equation can be solved and every solution interpreted.
Nor will it lessen the interest of this study to reflect that every
peculiarity which they will notice in the form of the Calculus
represents a corresponding feature in the constitution of their
own minds.

It would be premature to speak of the value which this
method may possess as an instrument of scientific investigation.
I speak here with reference to the theory of reasoning, and to
the principle of a true classification of the forms and cases of
Logic considered as a Science.\footnote
  {``Strictly a Science''; also ``an Art.''---\textit{Whately's Elements of Logic.} Indeed
  ought we not to regard all Art as applied Science; unless we are willing, with
  ``the multitude,'' to consider Art as ``guessing and aiming well''?---\textit{Plato,
  Philebus.}}
The aim of these investigations
was in the first instance confined to the expression of the
received logic, and to the forms of the Aristotelian arrangement,
\PageSep{8}
but it soon became apparent that restrictions were thus introduced,
which were purely arbitrary and had no foundation in
the nature of things. These were noted as they occurred, and
will be discussed in the proper place. When it became necessary
to consider the subject of hypothetical propositions (in which
comparatively less has been done), and still more, when an
interpretation was demanded for the general theorems of the
Calculus, it was found to be imperative to dismiss all regard for
precedent and authority, and to interrogate the method itself for
an expression of the just limits of its application. Still, however,
there was no special effort to arrive at novel results. But
among those which at the time of their discovery appeared to be
such, it may be proper to notice the following.

A logical proposition is, according to the method of this Essay,
expressible by an equation the form of which determines the
rules of conversion and of transformation, to which the given
proposition is subject. Thus the law of what logicians term
simple conversion, is determined by the fact, that the corresponding
equations are symmetrical, that they are unaffected by
a mutual change of place, in those symbols which correspond
to the convertible classes. The received laws of conversion
were thus determined, and afterwards another system, which is
thought to be more elementary, and more general. See Chapter,
\ChapRef{5}{On the Conversion of Propositions}.

The premises of a syllogism being expressed by equations, the
elimination of a common symbol between them leads to a third
equation which expresses the conclusion, this conclusion being
always the most general possible, whether Aristotelian or not.
Among the cases in which no inference was possible, it was
found, that there were two distinct forms of the final equation.
It was a considerable time before the explanation of this fact
was discovered, but it was at length seen to depend upon the
presence or absence of a true medium of comparison between
the premises. The distinction which is thought to be new
is illustrated in the Chapter, \ChapRef{6}{On Syllogisms}.
\PageSep{9}

The nonexclusive character of the disjunctive conclusion of
a hypothetical syllogism, is very clearly pointed out in the
examples of this species of argument.

The class of logical problems illustrated in the chapter, \ChapRef{9}{On
the Solution of Elective Equations}, is conceived to be new: and
it is believed that the method of that chapter affords the means
of a perfect analysis of any conceivable system of propositions,
an end toward which the rules for the conversion of a single
categorical proposition are but the first step.

However, upon the originality of these or any of these views,
I am conscious that I possess too slight an acquaintance with the
literature of logical science, and especially with its older literature,
to permit me to speak with confidence.

It may not be inappropriate, before concluding these observations,
to offer a few remarks upon the general question of the
use of symbolical language in the mathematics. Objections
have lately been very strongly urged against this practice, on
the ground, that by obviating the necessity of thought, and
substituting a reference to general formul� in the room of
personal effort, it tends to weaken the reasoning faculties.

Now the question of the use of symbols may be considered
in two distinct points of view. First, it may be considered with
reference to the progress of scientific discovery, and secondly,
with reference to its bearing upon the discipline of the intellect.

And with respect to the first view, it may be observed that
as it is one fruit of an accomplished labour, that it sets us at
liberty to engage in more arduous toils, so it is a necessary
result of an advanced state of science, that we are permitted,
and even called upon, to proceed to higher problems, than those
which we before contemplated. The practical inference is
obvious. If through the advancing power of scientific methods,
we find that the pursuits on which we were once engaged,
afford no longer a sufficiently ample field for intellectual effort,
the remedy is, to proceed to higher inquiries, and, in new
tracks, to seek for difficulties yet unsubdued. And such is,
\PageSep{10}
indeed, the actual law of scientific progress. We must be
content, either to abandon the hope of further conquest, or to
employ such aids of symbolical language, as are proper to the
stage of progress, at which we have arrived. Nor need we fear
to commit ourselves to such a course. We have not yet arrived
so near to the boundaries of possible knowledge, as to suggest
the apprehension, that scope will fail for the exercise of the
inventive faculties.

In discussing the second, and scarcely less momentous question
of the influence of the use of symbols upon the discipline
of the intellect, an important distinction ought to be made. It
is of most material consequence, whether those symbols are
used with a full understanding of their meaning, with a perfect
comprehension of that which renders their use lawful, and an
ability to expand the abbreviated forms of reasoning which they
induce, into their full syllogistic \Typo{devolopment}{development}; or whether they
are mere unsuggestive characters, the use of which is suffered
to rest upon authority.

The answer which must be given to the question proposed,
will differ according as the one or the other of these suppositions
is admitted. In the former case an intellectual discipline of a
high order is provided, an exercise not only of reason, but of
the faculty of generalization. In the latter case there is no
mental discipline whatever. It were perhaps the best security
against the danger of an unreasoning reliance upon symbols,
on the one hand, and a neglect of their just claims on the other,
that each subject of applied mathematics should be treated in the
spirit of the methods which were known at the time when the
application was made, but in the best form which those methods
have assumed. The order of attainment in the individual mind
would thus bear some relation to the actual order of scientific
discovery, and the more abstract methods of the higher analysis
would be offered to such minds only, as were prepared to
receive them.

The relation in which this Essay stands at once to Logic and
\PageSep{11}
to Mathematics, may further justify some notice of the question
which has lately been revived, as to the relative value of the two
studies in a liberal education. One of the chief objections which
have been urged against the study of Mathematics in general, is
but another form of that which has been already considered with
respect to the use of symbols in particular. And it need not here
be further dwelt upon, than to notice, that if it avails anything,
it applies with an equal force against the study of Logic. The
canonical forms of the Aristotelian syllogism are really symbolical;
only the symbols are less perfect of their kind than those
of mathematics. If they are employed to test the validity of an
argument, they as truly supersede the exercise of reason, as does
a reference to a formula of analysis. Whether men do, in the
present day, make this use of the Aristotelian canons, except as
a special illustration of the rules of Logic, may be doubted; yet
it cannot be questioned that when the authority of Aristotle was
dominant in the schools of Europe, such applications were habitually
made. And our argument only requires the admission,
that the case is possible.

But the question before us has been argued upon higher
grounds. Regarding Logic as a branch of Philosophy, and defining
Philosophy as the ``science of a real existence,'' and ``the
research of causes,'' and assigning as its \emph{main} business the investigation
of the ``why, (\textgreek{t`o d'ioti}),'' while Mathematics display
only the ``that, (\textgreek{t`o <ot`i}),'' Sir W.~Hamilton has contended,
not simply, that the superiority rests with the study of Logic,
but that the study of Mathematics is at once dangerous and useless.\footnote
  {\textit{Edinburgh Review}, vol.~\textsc{lxii}. p.~409, and \textit{Letter to A. De~Morgan, Esq.}}
The pursuits of the mathematician ``have not only not
trained him to that acute scent, to that delicate, almost instinctive,
tact which, in the twilight of probability, the search and
discrimination of its finer facts demand; they have gone to cloud
his vision, to indurate his touch, to all but the blazing light, the
iron chain of demonstration, and left him out of the narrow confines
of his science, to a passive \emph{credulity} in any premises, or to
\PageSep{12}
an absolute \emph{incredulity} in all.'' In support of these and of other
charges, both argument and copious authority are adduced.\footnote
  {The arguments are in general better than the authorities. Many writers
  quoted in condemnation of mathematics (Aristo, Seneca, Jerome, Augustine,
  Cornelius Agrippa,~\etc.)\ have borne a no less explicit testimony against other
  sciences, nor least of all, against that of logic. The treatise of the last named
  writer \textit{De~Vanitate Scientiarum}, must surely have been referred to by mistake.---\textit{Vide}
  cap.~\textsc{cii}.}
I shall not attempt a complete discussion of the topics which
are suggested by these remarks. My object is not controversy,
and the observations which follow are offered not in the spirit
of antagonism, but in the hope of contributing to the formation
of just views upon an important subject. Of Sir W.~Hamilton
it is impossible to speak otherwise than with that respect which
is due to genius and learning.

Philosophy is then described as the \emph{science of a real existence}
\Pagelabel{12}%
and \emph{the research of causes}. And that no doubt may rest upon
the meaning of the word \emph{cause}, it is further said, that philosophy
``mainly investigates the \emph{why}.'' These definitions are common
among the ancient writers. Thus Seneca, one of Sir W.~Hamilton's
authorities, \textit{Epistle}~\textsc{lxxxviii}., ``The philosopher seeks
and knows the \emph{causes} of natural things, of which the mathematician
searches out and computes the numbers and the measures.''
It may be remarked, in passing, that in whatever
degree the belief has prevailed, that the business of philosophy
is immediately with \emph{causes}; in the same degree has every
science whose object is the investigation of \emph{laws}, been lightly
esteemed. Thus the Epistle to which we have referred, bestows,
by contrast with Philosophy, a separate condemnation on Music
and Grammar, on Mathematics and Astronomy, although it is
that of Mathematics only that Sir W.~Hamilton has quoted.

Now we might take our stand upon the conviction of many
thoughtful and reflective minds, that in the extent of the meaning
above stated, Philosophy is impossible. The business of
true Science, they conclude, is with laws and phenomena. The
nature of Being, the mode of the operation of Cause, the \emph{why},
\PageSep{13}
they hold to be beyond the reach of our intelligence. But we
do not require the vantage-ground of this position; nor is it
doubted that whether the aim of Philosophy is attainable or not,
the desire which impels us to the attempt is an instinct of our
higher nature. Let it be granted that the problem which has
baffled the efforts of ages, is not a hopeless one; that the
``science of a real existence,'' and ``the research of causes,''
``that kernel'' for which ``Philosophy is still militant,'' do
not transcend the limits of the human intellect. I am then
compelled to assert, that according to this view of the nature of
Philosophy, \emph{Logic forms no part of it}. On the principle of
a true classification, we ought no longer to associate Logic and
Metaphysics, but Logic and Mathematics.

Should any one after what has been said, entertain a doubt
upon this point, I must refer him to the evidence which will be
afforded in the following Essay. He will there see Logic resting
like Geometry upon axiomatic truths, and its theorems constructed
upon that general doctrine of symbols, which constitutes
the foundation of the recognised Analysis. In the Logic
of Aristotle he will be led to view a collection of the formul�
of the science, expressed by another, but, (it is thought) less
perfect scheme of symbols. I feel bound to contend for the
absolute exactness of this parallel. It is no escape from the conclusion
to which it points to assert, that Logic not only constructs
a science, but also inquires into the origin and the nature of its
own principles,---a distinction which is denied to Mathematics.
``It is wholly beyond the domain of mathematicians,'' it is said,
``to inquire into the origin and nature of their principles.''---%
\textit{Review}, page~415. But upon what ground can such a distinction
be maintained? What definition of the term Science will
be found sufficiently arbitrary to allow such differences?

The application of this conclusion to the question before us is
clear and decisive. The mental discipline which is afforded by
the study of Logic, \emph{as an exact science}, is, in species, the same
as that afforded by the study of Analysis.
\PageSep{14}

Is it then contended that either Logic or Mathematics can
supply a perfect discipline to the Intellect? The most careful
and unprejudiced examination of this question leads me to doubt
whether such a position can be maintained. The exclusive claims
of either must, I believe, be abandoned, nor can any others, partaking
of a like exclusive character, be admitted in their room.
It is an important observation, which has more than once been
made, that it is one thing to arrive at correct premises, and another
thing to deduce logical conclusions, and that the business of life
depends more upon the former than upon the latter. The study
of the exact sciences may teach us the one, and it may give us
some general preparation of knowledge and of practice for the
attainment of the other, but it is to the union of thought with
action, in the field of Practical Logic, the arena of Human Life,
that we are to look for its fuller and more perfect accomplishment.

I desire here to express my conviction, that with the advance
of our knowledge of all true science, an ever-increasing
harmony will be found to prevail among its separate branches.
The view which leads to the rejection of one, ought, if consistent,
to lead to the rejection of others. And indeed many
of the authorities which have been quoted against the study
of Mathematics, are even more explicit in their condemnation of
Logic. ``Natural science,'' says the Chian Aristo, ``is above us,
Logical science does not concern us.'' When such conclusions
are founded (as they often are) upon a deep conviction of the
preeminent value and importance of the study of Morals, we
admit the premises, but must demur to the inference. For it
has been well said by an ancient writer, that it is the ``characteristic
of the liberal sciences, not that they conduct us to Virtue,
but that they prepare us for Virtue;'' and Melancthon's sentiment,
``abeunt studia in mores,'' has passed into a proverb.
Moreover, there is a common ground upon which all sincere
votaries of truth may meet, exchanging with each other the
language of Flamsteed's appeal to Newton, ``The works of the
Eternal Providence will be better understood through your
labors and mine.''
\PageSep{15}


\Chapter{First Principles.}

\First{Let} us employ the symbol~$1$, or unity, to represent the
Universe, and let us understand it as comprehending every
conceivable class of objects whether actually existing or not,
it being premised that the same individual may be found in
more than one class, inasmuch as it may possess more than one
quality in common with other individuals. Let us employ the
letters $X$,~$Y$,~$Z$, to represent the individual members of classes,
$X$~applying to every member of one class, as members of that
particular class, and $Y$~to every member of another class as
members of such class, and so on, according to the received language
of treatises on Logic.

Further let us conceive a class of symbols $x$,~$y$,~$z$, possessed
of the following character.

The symbol~$x$ operating upon any subject comprehending
individuals or classes, shall be supposed to select from that
subject all the~$X$s which it contains. In like manner the symbol~$y$,
operating upon any subject, shall be supposed to select from
it all individuals of the class~$Y$ which are comprised in it, and
so on.

When no subject is expressed, we shall suppose~$1$ (the Universe)
to be the subject understood, so that we shall have
\[
x = x\quad (1),
\]
the meaning of either term being the selection from the Universe
of all the~$X$s which it contains, and the result of the operation
\PageSep{16}
being in common language, the class~$X$, \ie~the class of which
each member is an~$X$.

From these premises it will follow, that the product~$xy$ will
represent, in succession, the selection of the class~$Y$, and the
selection from the class~$Y$ of such individuals of the class~$X$ as
are contained in it, the result being the class whose members are
both $X$s~and~$Y$s. And in like manner the product~$xyz$ will
represent a compound operation of which the successive elements
are the selection of the class~$Z$, the selection from it of
such individuals of the class~$Y$ as are contained in it, and the
selection from the result thus obtained of all the individuals of
the class~$X$ which it contains, the final result being the class
common to $X$,~$Y$, and~$Z$.

From the nature of the operation which the symbols $x$,~$y$,~$z$,
are conceived to represent, we shall designate them as elective
symbols. An expression in which they are involved will be
called an elective function, and an equation of which the members
are elective functions, will be termed an elective equation.

It will not be necessary that we should here enter into the
analysis of that mental operation which we have represented by
the elective symbol. It is not an act of Abstraction according
to the common acceptation of that term, because we never lose
sight of the concrete, but it may probably be referred to an exercise
of the faculties of Comparison and Attention. Our present
concern is rather with the laws of combination and of succession,
by which its results are governed, and of these it will suffice to
notice the following.

1st. The result of an act of election is independent of the
grouping or classification of the subject.

Thus it is indifferent whether from a group of objects considered
as a whole, we select the class~$X$, or whether we divide
the group into two parts, select the~$X$s from them separately,
and then connect the results in one aggregate conception.

We may express this law mathematically by the equation
\[
x(u + v) = xu + xv,
\]
\PageSep{17}
$u + v$ representing the undivided subject, and $u$~and~$v$ the
component parts of it.

2nd. It is indifferent in what order two successive acts of
election are performed.

Whether from the class of animals we select sheep, and from
the sheep those which are horned, or whether from the class of
animals we select the horned, and from these such as are sheep,
the result is unaffected. In either case we arrive at the class
\emph{horned sheep}.

The symbolical expression of this law is
\[
xy = yx.
\]

3rd. The result of a given act of election performed twice,
or any number of times in succession, is the result of the same
act performed once.

If from a group of objects we select the~$X$s, we obtain a class
of which all the members are~$X$s. If we repeat the operation
on this class no further change will ensue: in selecting the~$X$s
we take the whole. Thus we have
\[
xx = x,
\]
or
\[
x^{2} = x;
\]
and supposing the same operation to be $n$~times performed, we
have
\[
x^{n} = x,
\]
which is the mathematical expression of the law above stated.\footnote
  {The office of the elective symbol~$x$, is to select individuals comprehended
  in the class~$X$. Let the class~$X$ be supposed to embrace the universe; then,
  whatever the class~$Y$ may be, we have
  \[
  xy = y.
  \]
  The office which $x$~performs is now equivalent to the symbol~$+$, in one at
  least of its interpretations, and the index law~\Eqref{(3)} gives
  \[
  +^{n} = +,
  \]
  which is the known property of that symbol.}

The laws we have established under the symbolical forms
\begin{align*}
x(u + v) &= xu + xv,
\Tag{(1)} \\
xy &= yx,
\Tag{(2)} \\
x^{n} &= x,
\Tag{(3)}
\end{align*}
\PageSep{18}
are sufficient for the basis of a Calculus. From the first of these,
it appears that elective symbols are \emph{distributive}, from the second
that they are \emph{commutative}; properties which they possess in
common with symbols of \emph{quantity}, and in virtue of which, all
the processes of common algebra are applicable to the present
system. The one and sufficient axiom involved in this application
is that equivalent operations performed upon equivalent
subjects produce equivalent results.\footnote
  {It is generally asserted by writers on Logic, that all reasoning ultimately
  depends on an application of the dictum of Aristotle, \textit{de omni et~nullo}. ``Whatever
  is predicated universally of any class of things, may be predicated in like
  manner of any thing comprehended in that class.'' But it is agreed that this
  dictum is not immediately applicable in all cases, and that in a majority of
  instances, a certain previous process of reduction is necessary. What are the
  elements involved in that process of reduction? Clearly they are as much
  a part of general reasoning as the dictum itself.

  Another mode of considering the subject resolves all reasoning into an application
  of one or other of the following canons,~viz.\

  1. If two terms agree with one and the same third, they agree with each
  other.

  2. If one term agrees, and another disagrees, with one and the same third,
  these two disagree with each other.

  But the application of these canons depends on mental acts equivalent to
  those which are involved in the before-named process of reduction. We have to
  select individuals from classes, to convert propositions,~\etc., before we can avail
  ourselves of their guidance. Any account of the process of reasoning is insufficient,
  which does not represent, as well the laws of the operation which the
  mind performs in that process, as the primary truths which it recognises and
  applies.

  It is presumed that the laws in question are adequately represented by the
  fundamental equations of the present Calculus. The proof of this will be found
  in its capability of expressing propositions, and of exhibiting in the results of
  its processes, every result that may be arrived at by ordinary reasoning.}

The third law~\Eqref{(3)} we shall denominate the index law. It is
peculiar to elective symbols, and will be found of great importance
in enabling us to reduce our results to forms meet for
interpretation.

From the circumstance that the processes of algebra may be
applied to the present system, it is not to be inferred that the
interpretation of an elective equation will be unaffected by such
processes. The expression of a truth cannot be negatived by
\PageSep{19}
a legitimate operation, but it may be limited. The equation
$y = z$ implies that the classes $Y$~and~$Z$ are equivalent, member
for member. Multiply it by a factor~$x$, and we have
\[
xy = xz,
\]
which expresses that the individuals which are common to the
classes $X$~and~$Y$ are also common to $X$~and~$Z$, and \textit{vice vers�}.
This is a perfectly legitimate inference, but the fact which it
declares is a less general one than was asserted in the original
proposition.
\PageSep{20}


\Chapter{Of Expression and Interpretation.}

\begin{Abstract}
A Proposition is a sentence which either affirms or denies, as, All men are
mortal, No creature is independent.

A Proposition has necessarily two terms, as \emph{men}, \emph{mortal}; the former of which,
or the one spoken of, is called the subject; the latter, or that which is affirmed
or denied of the subject, the predicate. These are connected together by the
copula~\emph{is}, or \emph{is not}, or by some other modification of the substantive verb.

The substantive verb is the only verb recognised in Logic; all others are
resolvable by means of the verb \emph{to be} and a participle or adjective, \eg~``The
Romans conquered''; the word conquered is both copula and predicate, being
equivalent to ``were (copula) victorious'' (predicate).

A Proposition must either be affirmative or negative, and must be also either
universal or particular. Thus we reckon in all, four kinds of pure categorical
Propositions.

1st. Universal-affirmative, usually represented by~$A$,
\[
\text{Ex. All $X$s are $Y$s.}
\]

2nd. Universal-negative, usually represented by~$E$,
\[
\text{Ex. No $X$s are $Y$s.}
\]

3rd. Particular-affirmative, usually represented by~$I$,
\[
\text{Ex. Some $X$s are $Y$s.}
\]

4th. Particular-negative, usually represented by~$O$,\footnote
  {The above is taken, with little variation, from the Treatises of Aldrich
  and Whately.}
\[
\text{Ex. Some $X$s are not $Y$s.}
\]
\end{Abstract}

1. To express the class, not-$X$, that is, the class including
all individuals that are not~$X$s.

The class~$X$ and the class not-$X$ together make the Universe.
But the Universe is~$1$, and the class~$X$ is determined by the
symbol~$x$, therefore the class not-$X$ will be determined by
the symbol~$1 - x$.
\PageSep{21}

Hence the office of the symbol $1 - x$ attached to a given
subject will be, to select from it all the not-$X$s which it
contains.

And in like manner, as the product~$xy$ expresses the entire
class whose members are both $X$s and~$Y$s, the symbol $y(1 - x)$
will represent the class whose members are $Y$s but not~$X$s,
and the symbol $(1 - x)(1 - y)$ the entire class whose members
are neither $X$s~nor~$Y$s.

2. To express the Proposition, All $X$s are~$Y$s.

As all the~$X$s which exist are found in the class~$Y$, it is
obvious that to select out of the Universe all~$Y$s, and from
these to select all~$X$s, is the same as to select at once from the
Universe all~$X$s.

Hence
\[
xy = x,
\]
or
\[
x(1 - y) = 0.
\Tag{(4)}
\]

3. To express the Proposition, No $X$s are~$Y$s.

To assert that no $X$s are~$Y$s, is the same as to assert that
there are no terms common to the classes $X$~and~$Y$. Now
all individuals common to those classes are represented by~$xy$.
Hence the Proposition that No~$X$s are~$Y$s, is represented by
the equation
\[
xy = 0.
\Tag{(5)}
\]

4. To express the Proposition, Some $X$s are~$Y$s.

If some $X$s are~$Y$s, there are some terms common to the
classes $X$~and~$Y$. Let those terms constitute a separate class~$V$,
to which there shall correspond a separate elective symbol~$v$,
then
\[
v = xy.
\Tag{(6)}
\]
And as $v$~includes all terms common to the classes $X$~and~$Y$,
we can indifferently interpret it, as Some~$X$s, or Some~$Y$s.
\PageSep{22}

5. To express the Proposition, Some $X$s are not~$Y$s.

In the last equation write $1 - y$ for~$y$, and we have
\[
v = x(1 - y),
\Tag{(7)}
\]
the interpretation of~$v$ being indifferently Some~$X$s or Some
not-$Y$s.

The above equations involve the complete theory of categorical
Propositions, and so far as respects the employment of
analysis for the deduction of logical inferences, nothing more
can be desired. But it may be satisfactory to notice some particular
forms deducible from the third and fourth equations, and
susceptible of similar application.

If we multiply the equation~\Eqref{(6)} by~$x$, we have
\[
vx = x^{2}y = xy\quad\text{by~\Eqref{(3)}.}
\]

Comparing with~\Eqref{(6)}, we find
\[
v = vx,
\]
or
\[
v(1 - x) = 0.
\Tag{(8)}
\]

And multiplying~\Eqref{(6)} by~$y$, and reducing in a similar manner,
we have
\[
v = vy,
\]
or
\[
v(1 - y) = 0.
\Tag{(9)}
\]

Comparing \Eqref{(8)} and~\Eqref{(9)},
\[
vx = vy = v.
\Tag{(10)}
\]

And further comparing \Eqref{(8)} and~\Eqref{(9)} with~\Eqref{(4)}, we have as the
equivalent of this system of equations the Propositions
\[
\begin{aligned}
&\text{All $V$s are~$X$s} \\
&\text{All $V$s are~$Y$s}
\end{aligned}
\Rbrace{2}.
\]

The system~\Eqref{(10)} might be used to replace~\Eqref{(6)}, or the single
equation
\[
vx = vy,
\Tag{(11)}
\]
might be used, assigning to~$vx$ the interpretation, Some~$X$s, and
to~$vy$ the interpretation, Some~$Y$s. But it will be observed that
\PageSep{23}
this system does not express quite so much as the single equation~\Eqref{(6)},
from which it is derived. Both, indeed, express the
Proposition, Some~$X$s are~$Y$s, but the system~\Eqref{(10)} does not
imply that the class~$V$ includes \emph{all} the terms that are common
to $X$~and~$Y$.

In like manner, from the equation~\Eqref{(7)} which expresses the
Proposition Some~$X$s are not~$Y$s, we may deduce the system
\[
vx = v(1 - y) = v,
\Tag{(12)}
\]
in which the interpretation of~$v(1 - y)$ is Some not-$Y$s. Since
in this case $vy = 0$, we must of course be careful not to interpret~$vy$
as Some~$Y$s.

If we multiply the first equation of the system~\Eqref{(12)},~viz.
\[
vx = v(1 - y),
\]
by~$y$, we have
\begin{align*}
vxy &= vy(1 - y); \\
\therefore vxy &= 0,
\Tag{(13)}
\end{align*}
which is a form that will occasionally present itself. It is not
necessary to revert to the primitive equation in order to interpret
this, for the condition that $vx$~represents Some~$X$s, shews
us by virtue of~\Eqref{(5)}, that its import will be
\[
\text{Some~$X$s are not~$Y$s,}
\]
the subject comprising \emph{all} the~$X$s that are found in the class~$V$.

Universally in these cases, difference of form implies a difference
of interpretation with respect to the auxiliary symbol~$v$,
and each form is interpretable by itself.

Further, these differences do not introduce into the Calculus
a needless perplexity. It will hereafter be seen that they give
a precision and a definiteness to its conclusions, which could not
otherwise be secured.

Finally, we may remark that all the equations by which
particular truths are expressed, are deducible from any one
general equation, expressing any one general Proposition, from
which those particular Propositions are necessary deductions.
\PageSep{24}
This has been partially shewn already, but it is much more fully
exemplified in the following scheme.

The general equation
\[
x = y,
\]
implies that the classes $X$~and~$Y$ are equivalent, member for
member; that every individual belonging to the one, belongs
to the other also. Multiply the equation by~$x$, and we have
\begin{align*}
x^{2} &= xy; \\
\therefore x &= xy,
\end{align*}
which implies, by~\Eqref{(4)}, that all~$X$s are~$Y$s. Multiply the same
equation by~$y$, and we have in like manner
\[
y = xy;
\]
the import of which is, that all~$Y$s are~$X$s. Take either of these
equations, the latter for instance, and writing it under the form
\[
(1 - x)y = 0,
\]
we may regard it as an equation in which~$y$, an unknown
quantity, is sought to be expressed in terms of~$x$. Now it
will be shewn when we come to treat of the Solution of Elective
Equations (and the result may here be verified by substitution)
that the most general solution of this equation is
\[
y = vx,
\]
which implies that All~$Y$s are~$X$s, and that Some~$X$s are~$Y$s.
Multiply by~$x$, and we have
\[
vy = vx,
\]
which indifferently implies that some~$Y$s are~$X$s and some~$X$s
are~$Y$s, being the particular form at which we before arrived.

For convenience of reference the above and some other
results have been classified in the annexed Table, the first
column of which contains propositions, the second equations,
and the third the conditions of final interpretation. It is to
be observed, that the auxiliary equations which are given in
this column are not independent: they are implied either
in the equations of the second column, or in the condition for
\PageSep{25}
the interpretation of~$v$. But it has been thought better to write
them separately, for greater ease and convenience. And it is
further to be borne in mind, that although three different forms
are given for the expression of each of the \emph{particular} propositions,
everything is really included in the first form.
\begin{table}[hbt!]
\caption{TABLE.}
\footnotesize
\begin{alignat*}{3}
&\text{The class~$X$}       &&x \\
&\text{The class not-$X$}   &&1 - x \\
%
&\!\begin{aligned}
&\text{All~$X$s are~$Y$s} \\
&\text{All~$Y$s are~$X$s}
\end{aligned}\Rbrace{2}     && x = y \\
%
&\text{All~$X$s are~$Y$s}   && x(1 - y) = 0 \\
&\text{No~$X$s are~$Y$s}    && \PadTo[r]{x(1 - y) = 0}{xy = 0} \\
%
&\!\begin{aligned}
&\text{All~$Y$s are~$X$s} \\
&\text{Some~$X$s are~$Y$s}
\end{aligned}\Rbrace{2}     && y = vx
&&\begin{aligned}
&vx = \text{Some~$X$s} \\
&v(1 - x) = 0.
\end{aligned} \\[8pt]
%
&\!\begin{aligned}
&\text{No~$Y$s are~$X$s} \\
&\text{Some not-$X$s are~$Y$s}
\end{aligned}\Rbrace{2}     && y = v(1 - x)
&&\begin{aligned}
v(1 - x) &= \text{some not-$X$s} \\
vx &= 0.
\end{aligned} \\[8pt]
%
&\text{Some~$X$s are~$Y$s}  &&
\Lbrace{3}\begin{aligned}
&v = xy \\
\text{or } &vx = vy \\
\text{or } &vx(1 - y) = 0
\end{aligned}\quad          &&
\begin{aligned}
&v = \text{some~$X$s or some~$Y$s} \\
&vx = \text{some~$X$s},\ vy = \text{some~$Y$s} \\
&v(1 - x) = 0,\ v(1 - y) = 0.
\end{aligned} \\[8pt]
%
&\text{Some~$X$s are not~$Y$s} &&
\Lbrace{3}\begin{aligned}
&v = x(1 - y) \\
\text{or } &vx = v(1 - y) \\
\text{or } &vxy = 0
\end{aligned}               &&
\begin{aligned}
&v = \text{some~$X$s, or some not-$Y$s} \\
&vx = \text{some~$X$s}, v(1 - y) = \text{some not-$Y$s} \\
&v(1 - x) = 0,\ vy = 0.
\end{aligned}
\end{alignat*}
\end{table}
\PageSep{26}


\Chapter{Of the Conversion of Propositions.}

\begin{Abstract}
A Proposition is said to be converted when its terms are transposed; when
nothing more is done, this is called simple conversion; \eg
\begin{align*}
&\text{No virtuous man is a tyrant, \emph{is converted into}} \\
&\text{No tyrant is a virtuous man.}
\intertext{\indent
Logicians also recognise conversion \textit{per accidens}, or by limitation, \eg}
&\text{All birds are animals, \emph{is converted into}} \\
&\text{Some animals are birds.}
\intertext{And conversion by \emph{contraposition} or \emph{negation}, as}
&\text{Every poet is a man of genius, \emph{converted into}} \\
&\text{He who is not a man of genius is not a poet.}
\end{align*}

In one of these three ways every Proposition may be illatively converted, viz.\
$E$~and~$I$ simply, $A$~and~$O$ by negation, $A$~and~$E$ by limitation.
\end{Abstract}

The primary canonical forms already determined for the
expression of Propositions, are
\begin{alignat*}{2}
&\text{All~$X$s are~$Y$s,}      &x(1 - y) &= 0,
\Ltag{A} \\
&\text{No~$X$s are~$Y$s,}       &xy &= 0,
\Ltag{E} \\
&\text{Some~$X$s are~$Y$s,}     &v &= xy,
\Ltag{I} \\
&\text{Some~$X$s are not~$Y$s,} &v &= x(1 - y).
\Ltag{O}
\end{alignat*}

On examining these, we perceive that $E$~and~$I$ are symmetrical
with respect to $x$~and~$y$, so that $x$~being changed into~$y$,
and $y$~into~$x$, the equations remain unchanged. Hence $E$~and~$I$
may be interpreted into
\begin{gather*}
\text{No~$Y$s are~$X$s,} \\
\text{Some~$Y$s are~$X$s,}
\end{gather*}
respectively. Thus we have the known rule of the Logicians,
that particular affirmative and universal negative Propositions
admit of simple conversion.
\PageSep{27}

The equations $A$~and~$O$ may be written in the forms
\begin{gather*}
(1 - y)\bigl\{1 - (1 - x)\bigr\} = 0, \\
v = (1 - y)\bigl\{1 - (1 - x)\bigr\}.
\end{gather*}

Now these are precisely the forms which we should have
obtained if we had in those equations changed $x$~into~$1 - y$,
and $y$~into~$1 - x$, which would have represented the changing
in the original Propositions of the~$X$s into not-$Y$s, and the~$Y$s
into not-$X$s, the resulting Propositions being
\begin{gather*}
\text{All not-$Y$s are not-$X$s,} \\
\text{Some not-$Y$s are not not-$X$s.}\atag
\end{gather*}
Or we may, by simply inverting the order of the factors in the
second member of~$O$, and writing it in the form
\[
v = (1 - y)x,
\]
interpret it by~$I$ into
\[
\text{Some not-$Y$s are~$X$s,}
\]
which is really another form of~\aref. Hence follows the rule,
that universal affirmative and particular negative Propositions
admit of negative conversion, or, as it is also termed, conversion
by contraposition.

The equations $A$~and~$E$, written in the forms
\begin{align*}
(1 - y) x &= 0, \\
yx &= 0,
\end{align*}
give on solution the respective forms
\begin{align*}
x &= vy, \\
x &= v(1 - y),
\end{align*}
the correctness of which may be shewn by substituting these
values of~$x$ in the equations to which they belong, and observing
that those equations are satisfied quite independently of the nature
of the symbol~$v$. The first solution may be interpreted into
\[
\text{Some~$Y$s are~$X$s,}
\]
and the second into
\[
\text{Some not-$Y$s are~$X$s.}
\]
\PageSep{28}
From which it appears that universal-affirmative, and universal-negative
Propositions are convertible by limitation, or, as it has
been termed, \textit{per accidens}.

The above are the laws of Conversion recognized by Abp.~Whately.
Writers differ however as to the admissibility of
negative conversion. The question depends on whether we will
consent to use such terms as not-$X$, not-$Y$. Agreeing with
those who think that such terms ought to be admitted, even
although they change the \emph{kind} of the Proposition, I am constrained
to observe that the present classification of them is
faulty and defective. Thus the conversion of No~$X$s are~$Y$s,
into All~$Y$s are not-$X$s, though perfectly legitimate, is not recognised
in the above scheme. It may therefore be proper to
examine the subject somewhat more fully.

Should we endeavour, from the system of equations we have
obtained, to deduce the laws not only of the conversion, but
also of the general transformation of propositions, we should be
led to recognise the following distinct elements, each connected
with a distinct mathematical process.

1st. The negation of a term, \ie~the changing of~$X$ into not-$X$,
or not-$X$ into~$X$.

2nd. The translation of a Proposition from one \emph{kind} to
another, as if we should change
\[
\text{All~$X$s are~$Y$s into Some~$X$s are~$Y$s,}
\Ltag{$A$~into~$I$}
\]
which would be lawful; or
\[
\text{All~$X$s are~$Y$s into No~$X$s are~$Y$\Typo{.}{s,}}
\Ltag{$A$~into~$E$}
\]
which would be unlawful.

3rd. The simple conversion of a Proposition.

The conditions in obedience to which these processes may
lawfully be performed, may be deduced from the equations by
which Propositions are expressed.

We have
\begin{alignat*}{2}
&\text{All~$X$s are~$Y$s\Add{,}}\qquad & x(1 - y) &= 0,
\Ltag{A} \\
&\text{No~$X$s are~$Y$s\Add{,}}        & xy &= 0.
\Ltag{E}
\end{alignat*}
\PageSep{29}

Write $E$ in the form
\[
x\bigl\{1 - (1 - y)\bigr\} = 0,
\]
%[** TN: "A" italicized in the original]
and it is interpretable by~$A$ into
\[
\text{All~$X$s are not-$Y$s,}
\]
so that we may change
\[
\text{No~$X$s are~$Y$s into All~$X$s are not-$Y$s.}
\]

In like manner $A$~interpreted by~$E$ gives
\[
\text{No~$X$s are not-$Y$s,}
\]
so that we may change
\[
\text{All~$X$s are~$Y$s into No~$X$s are not-$Y$s.}
\]

From these cases we have the following Rule: A universal-affirmative
Proposition is convertible into a universal-negative,
and, \textit{vice vers�}, by negation of the predicate.

Again, we have
\begin{alignat*}{2}
&\text{Some~$X$s are~$Y$s\Add{,}}          & v &= xy, \\
&\text{Some~$X$s are not~$Y$s\Add{,}}\qquad& v &= x(1 - y).
\end{alignat*}
These equations only differ from those last considered by the
presence of the term~$v$. The same reasoning therefore applies,
and we have the Rule---

A particular-affirmative proposition is convertible into a particular-negative,
and \textit{vice vers�}, by negation of the predicate.

Assuming the universal Propositions
\begin{alignat*}{2}
&\text{All~$X$s are~$Y$s\Add{,}}\qquad & x(1 - y) &= 0, \\
&\text{No~$X$s are~$Y$s\Add{,}}        & xy &= 0.
\end{alignat*}
Multiplying by~$v$, we find
\begin{align*}
vx(1 - y) &= 0, \\
vxy &= 0,
\end{align*}
which are interpretable into
\begin{align*}
&\text{Some~$X$s are~$Y$s,}
\Ltag{I} \\
&\text{Some~$X$s are not~$Y$s.}
\Ltag{O}
\end{align*}
\PageSep{30}

Hence a universal-affirmative is convertible into a particular-affirmative,
and a universal-negative into a particular-negative
without negation of subject or predicate.

Combining the above with the already proved rule of simple
conversion, we arrive at the following system of independent
laws of transformation.

1st. An affirmative Proposition may be changed into its corresponding
negative ($A$~into~$E$, or $I$~into~$O$), and \textit{\Typo{vice versa}{vice vers�}},
by negation of the predicate.

2nd. A universal Proposition may be changed into its corresponding
particular Proposition, ($A$~into~$I$, or $E$~into~$O$).

3rd. In a particular-affirmative, or universal-negative Proposition,
the terms may be mutually converted.

Wherein negation of a term is the changing of~$X$ into not-$X$,
and \textit{vice vers�}, and is not to be understood as affecting the \emph{kind}
of the Proposition.

Every lawful transformation is reducible to the above rules.
Thus we have
\begin{alignat*}{2}
&\text{All~$X$s are~$Y$s,} \\
&\text{No~$X$s are not-$Y$s} &&\text{by 1st rule,} \\
&\text{No not-$Y$s are~$X$s} &&\text{by 3rd rule,} \\
&\text{All not-$Y$s are not-$X$s } &&\text{by 1st rule,}
\end{alignat*}
which is an example of \emph{negative conversion}. Again,
\begin{alignat*}{2}
&\text{No~$X$s are~$Y$s,} \\
&\text{No~$Y$s are~$X$s} &&\text{3rd rule,} \\
&\text{All~$Y$s are not-$X$s}\quad &&\text{1st rule,}
\end{alignat*}
which is the case already deduced.
\PageSep{31}


\Chapter{Of Syllogisms.}

\begin{Abstract}
A Syllogism consists of three Propositions, the last of which, called the
conclusion, is a logical consequence of the two former, called the premises;
\Typo{e.g.}{\eg}
\begin{alignat*}{2}
&\text{\emph{Premises,}} &&
\Lbrace{2}\begin{aligned}
&\text{All~$Y$s are~$X$s.} \\
&\text{All~$Z$s are~$Y$s.}
\end{aligned} \\
&\text{\emph{Conclusion,}}\quad &&
\text{All~$Z$s are~$X$s.}
\end{alignat*}

Every syllogism has three and only three terms, whereof that which is
the subject of the conclusion is called the \emph{minor} term, the predicate of the
conclusion, the \emph{major} term, and the remaining term common to both premises,
the middle term. Thus, in \Typo{ths}{the} above formula, $Z$~is the minor term, $X$~the
major term, $Y$~the middle term.

The figure of a syllogism consists in the situation of the middle term with
respect to the terms of the conclusion. The varieties of figure are exhibited
in the annexed scheme.
\[
\begin{array}{*{3}{c<{\qquad}}c@{}}
\ColHead{1st Fig.} & \ColHead{2nd Fig.} & \ColHead{3rd Fig.} & \ColHead{4th Fig.} \\
YX & XY & YX & XY \\
ZY & ZY & YZ & YZ \\
ZX & ZX & ZX & ZX
\end{array}
\]

When we designate the three propositions of a syllogism by their usual
symbols ($A$, $E$, $I$, $O$), and in their actual order, we are said to determine
the mood of the syllogism. Thus the syllogism given above, by way of
illustration, belongs to the mood~$AAA$ in the first figure.

The moods of all syllogisms commonly received as valid, are represented
by the vowels in the following mnemonic verses.

Fig.~1.---bArbArA, cElArEnt, dArII, fErIO que prioris.

Fig.~2.---cEsArE, cAmEstrEs, \Typo{fEstIno}{fEstInO}, bArOkO, secund�.

Fig.~3.---Tertia dArAptI, dIsAmIs, dAtIsI, fElAptOn, \\
\PadTo{\text{\indent Fig.~3.---}}{}bOkArdO, fErIsO, habet: quarta insuper addit.

Fig.~4.---brAmAntIp, cAmEnEs, dImArIs, \Typo{fEsapO}{fEsApO}, frEsIsOn.
\end{Abstract}

\First{The} equation by which we express any Proposition concerning
the classes $X$~and~$Y$, is an equation between the
symbols $x$~and~$y$, and the equation by which we express any
\PageSep{32}
Proposition concerning the classes $Y$~and~$Z$, is an equation
between the symbols $y$~and~$z$. If from two such equations
we eliminate~$y$, the result, if it do not vanish, will be an
equation between $x$~and~$z$, and will be interpretable into a
Proposition concerning the classes $X$~and~$Z$. And it will then
constitute the third member, or Conclusion, of a Syllogism,
of which the two given Propositions are the premises.

The result of the elimination of~$y$ from the equations
\[
\begin{alignedat}{2}
ay  &+ b  &&= 0, \\
a'y &+ b' &&= 0,
\end{alignedat}
\Tag{(14)}
\]
is the equation
\[
ab' - a'b = 0.
\Tag{(15)}
\]

Now the equations of Propositions being of the first order
with reference to each of the variables involved, all the cases
of elimination which we shall have to consider, will be reducible
to the above case, the constants $a$,~$b$, $a'$,~$b'$, being
replaced by functions of $x$,~$z$, and the auxiliary symbol~$v$.

As to the choice of equations for the expression of our
premises, the only restriction is, that the equations must not
\emph{both} be of the form $ay = 0$, for in such cases elimination would
be impossible. When both equations are of this form, it is
necessary to solve one of them, and it is indifferent which
we choose for this purpose. If that which we select is of
the form $xy = 0$, its solution is
\[
y = v(1 - x),
\Tag{(16)}
\]
if of the form $(1 - x)y = 0$, the solution will be
\[
y = vx,
\Tag{(17)}
\]
and these are the only cases which can arise. The reason
of this exception will appear in the sequel.

For the sake of uniformity we shall, in the expression of
particular propositions, confine ourselves to the forms
\begin{alignat*}{2}
vx &= vy,           &&\text{Some~$X$s are~$Y$s,} \\
vx &= v(1 - y),\quad&&\text{Some~$X$s are not~$Y$s\Typo{,}{.}}
\end{alignat*}
\PageSep{33}
These have a closer analogy with \Eqref{(16)}~and~\Eqref{(17)}, than the other
forms which might be used.

Between the forms about to be developed, and the Aristotelian
canons, some points of difference will occasionally be observed,
of which it may be proper to forewarn the reader.

To the right understanding of these it is proper to remark,
that the essential structure of a Syllogism is, in some measure,
arbitrary. Supposing the order of the premises to be fixed,
and the distinction of the major and the minor term to be
thereby determined, it is purely a matter of choice which of
the two shall have precedence in the Conclusion. Logicians
have settled this question in favour of the minor term, but
it is clear, that this is a convention. Had it been agreed
that the major term should have the first place in the conclusion,
a logical scheme might have been constructed, less
convenient in some cases than the existing one, but superior
in others. What it lost in \textit{barbara}, it would gain in \textit{bramantip}.
Convenience is \emph{perhaps} in favour of the adopted arrangement,\footnote
  {The contrary view was maintained by Hobbes. The question is very
  fairly discussed in Hallam's \textit{Introduction to the Literature of Europe}, vol.~\textsc{iii}.
  p.~309. In the rhetorical use of Syllogism, the advantage appears to rest
  with the rejected form.}
but it is to be remembered that it is \emph{merely} an arrangement.

Now the method we shall exhibit, not having reference
to one scheme of arrangement more than to another, will
always give the more general conclusion, regard being paid
only to its abstract lawfulness, considered as a result of pure
reasoning. And therefore we shall sometimes have presented
to us the spectacle of conclusions, which a logician would
pronounce informal, but never of such as a reasoning being
would account false.

The Aristotelian canons, however, beside restricting the \emph{order}
of the terms of a conclusion, limit their nature also;---and
this limitation is of more consequence than the former. We
may, by a change of figure, replace the particular conclusion
\PageSep{34}
of \textit{bramantip} by the general conclusion of~\textit{barbara}; but we
cannot thus reduce to rule such inferences, as
\[
\text{Some not-$X$s are not~$Y$s.}
\]

Yet there are cases in which such inferences may lawfully
be drawn, and in unrestricted argument they are of frequent
occurrence. Now if an inference of this, or of any other
kind, is lawful in itself, it will be exhibited in the results
of our method.

We may by restricting the canon of interpretation confine
our expressed results within the limits of the scholastic logic;
but this would only be to restrict ourselves to the use of a part
of the conclusions to which our analysis entitles us.

The classification we shall adopt will be purely mathematical,
and we shall afterwards consider the logical arrangement to
which it corresponds. It will be sufficient, for reference, to
name the premises and the Figure in which they are found.

\textsc{Class} 1st.---Forms in which $v$~does not enter.

Those which admit of an inference are $AA$,~$EA$, Fig.~1;
$AE$,~$EA$, Fig.~2; $AA$,~$AE$, Fig.~4.

Ex. $AA$, Fig.~1, and, by mutation of premises (change of
order), $AA$,~Fig.~4.
\begin{alignat*}{4}
&\text{All~$Y$s are~$X$s,}\qquad&
y(1 - x) &= 0,\qquad&& \text{or }& (1 - x) y &= 0, \\
&\text{All~$Z$s are~$Y$s,} &
z(1 - y) &= 0, &&\text{or }& zy - z &= 0.
\end{alignat*}

Eliminating~$y$ by~\Eqref{(13)} we have
\begin{gather*}
z(1 - x) = 0, \\
\therefore\ \text{All~$Z$s are~$X$s.}
\end{gather*}

A convenient mode of effecting the elimination, is to write
the equation of the premises, so that $y$~shall appear only as
a factor of one member in the first equation, and only as
a factor of the opposite member in the second equation, and
then to multiply the equations, omitting the~$y$. This method
we shall adopt.
\PageSep{35}

Ex. $AE$, Fig.~2, and, by mutation of premises, $EA$, Fig\Typo{,}{.}~2.
\[
\begin{alignedat}[t]{2}
&\text{All~$X$s are~$Y$s,}\qquad& x(1 - y) &= 0, \\
&\text{No~$Z$s are~$Y$s,}       &       zy &= 0,
\end{alignedat}\quad
\begin{array}[t]{rr@{\,}c@{\,}l@{}}
\text{or } & x &=& xy\Add{,} \\
           &zy &=&  0\Add{,} \\
\cline{2-4}
           &zx &=&  0\Add{,} \\
\multicolumn{4}{l}{\therefore\ \rlap{No~$Z$s are~$X$s.}}
\end{array}
\]

The only case in which there is no inference is~$AA$, Fig.~2,
\[
\begin{alignedat}[t]{2}
&\text{All~$X$s are~$Y$s,}\qquad& x(1 - y) &= 0, \\
&\text{All~$Z$s are~$Y$s,}      & z(1 - y) &= 0,
\end{alignedat}\quad
\begin{array}[t]{rr@{\,}c@{\,}l@{}}
& x &=& xy\Add{,} \\
&zy &=&  z\Add{,} \\
\cline{2-4}
&xz &=& xz\Add{,} \\
\multicolumn{4}{l}{\rlap{$\therefore\ 0 = 0$.}}
\end{array}
\]

\textsc{Class} 2nd.---When $v$~is introduced by the solution of an
equation.

The lawful cases directly or indirectly\footnote
  {We say \emph{directly} or \emph{indirectly}, mutation or conversion of premises being
  in some instances required. Thus, $AE$ (fig.~1) is resolvable by \Chg{Fesapo}{\textit{fesapo}} (fig.~4),
  or by \Chg{Ferio}{\textit{ferio}} (fig.~1). Aristotle and his followers rejected the fourth figure
  as only a modification of the first, but this being a mere question of form,
  either scheme may be termed Aristotelian.}
determinable by the
Aristotelian Rules are~$AE$, Fig.~1; $AA$, $AE$, $EA$, Fig.~3;
$EA$, Fig.~4.

The lawful cases not so determinable, are $EE$, Fig.~1; $EE$,
Fig.~2; $EE$, Fig.~3; $EE$, Fig.~4.

Ex. $AE$, Fig.~1, and, by mutation of premises, $EA$, Fig.~4.
\[
\begin{alignedat}[t]{2}
&\text{All~$Y$s are~$X$s,}\qquad& y(1 - x) &= 0, \\
&\text{No~$Z$s are~$Y$s,}       &       zy &= 0,
\end{alignedat}\quad
\begin{array}[t]{rr@{\,}c@{\,}l@{}}
&y &=& vx\Add{,}\atag \\
&0 &=& zy\Add{,} \\
\cline{2-4}
&0 &=& vzx\Add{,} \\
\multicolumn{4}{l}{\therefore\ \rlap{Some~$X$s are not~$Z$s.}}
\end{array}
\]

The reason why we cannot interpret $vzx = 0$ into Some~$Z$s
are not-$X$s, is that by the very terms of the first equation~\aref\
the interpretation of~$vx$ is fixed, as Some~$X$s; $v$~is regarded
as the representative of Some, only with reference to the
class~$X$.
\PageSep{36}

For the reason of our employing a solution of one of the
primitive equations, see the remarks on \Eqref{(16)}~and~\Eqref{(17)}. Had
we solved the second equation instead of the first, we should
have had
\begin{gather*}
\begin{aligned}
(1 - x)y &= 0, \\
v(1 - z) &= y,\atag \\
v(1 - z)(1 - x) &= 0,\btag
\end{aligned} \\
\therefore\ \text{Some not-$Z$s are~$X$s.}
\end{gather*}

Here it is to be observed, that the second equation~\aref\ fixes
the meaning of~$v(1 - z)$, as Some not-$Z$s. The full meaning
of the result~\bref\ is, that all the not-$Z$s which are found in
the class~$Y$ are found in the class~$X$, and it is evident that
this could not have been expressed in any other way.

Ex.~2. $AA$, Fig.~3.
\[
\begin{alignedat}[t]{2}
&\text{All~$Y$s are~$X$s,}\qquad& y(1 - x) &= 0, \\
&\text{All~$Y$s are~$Z$s,}      & y(1 - z) &= 0,
\end{alignedat}\quad
\begin{array}[t]{rr@{\,}c@{\,}l@{}}
&y &=& vx\Add{,} \\
&0 &=& y(1 - z)\Add{,} \\
\cline{2-4}
&0 &=& vx(1 - z)\Add{,} \\
\multicolumn{4}{l}{\therefore\ \rlap{Some~$X$s are~$Z$s.}}
\end{array}
\]

Had we solved the second equation, we should have had
as our result, Some~$Z$s are~$X$s. The form of the final equation
particularizes what~$X$s or what~$Z$s are referred to, and this
remark is general.

The following, $EE$, Fig.~1, and, by mutation, $EE$, Fig.~4,
is an example of a lawful case not determinable by the Aristotelian
Rules.
\[
\begin{alignedat}[t]{2}
&\text{No~$Y$s are~$X$s,}\qquad& xy &= 0, \\
&\text{No~$Z$s are~$Y$s,}      & zy &= 0,
\end{alignedat}\quad
\begin{array}[t]{rr@{\,}c@{\,}l@{}}
&0 &=& xy\Add{,} \\
&y &=& v(1 - z)\Add{,} \\
\cline{2-4}
&0 &=& v(1 - z)x\Add{,} \\
\multicolumn{4}{l}{\therefore\ \rlap{Some not-$Z$s are not~$X$s.}}
\end{array}
\]

\textsc{Class} 3rd.---When $v$~is met with in one of the equations,
but not introduced by solution.
\PageSep{37}

The lawful cases determinable \emph{directly} or \emph{indirectly} by the
Aristotelian Rules, are $AI$,~$EI$, Fig.~1; $AO$, $EI$, $OA$, $IE$,
Fig.~2; $AI$, $AO$, $EI$, $EO$, $IA$, $IE$, $OA$, $OE$, Fig.~3; $IA$, $IE$,
Fig.~4.

Those not so determinable are~$OE$, Fig.~1; $EO$, Fig.~4.

The cases in which no inference is possible, are $AO$, $EO$,
$IA$, $IE$, $OA$, Fig.~1; $AI$, $EO$, $IA$, $OE$, Fig.~2; $OA$, $OE$,
$AI$, $EI$, $AO$, Fig.~4.

Ex.~1. $AI$, Fig.~1, and, by mutation, $IA$, Fig.~4.
\[
\begin{aligned}[t]
&\text{All~$Y$s are~$X$s,} \\
&\text{Some~$Z$s are~$Y$s,}
\end{aligned}
\begin{array}[t]{>{\qquad}rr@{\,}c@{\,}l@{}}
&y(1 - x) &=& 0\Add{,} \\
&vz &=& vy\Add{,} \\
\cline{2-4}
&vz(1 - x) &=& 0\Add{,} \\
\therefore\ &
\multicolumn{3}{l}{\rlap{Some~$Z$s are~$X$s.}}
\end{array}
\]

Ex.~2. $AO$, Fig.~2, and, by mutation, $OA$, Fig.~2.
\[
\begin{alignedat}[t]{2}
&\text{All~$X$s are~$Y$s,}\qquad& x(1 - y) &= 0, \\
&\text{Some~$Z$s are not~$Y$s,} & vz &= v(1 - y),
\end{alignedat}\quad
\begin{array}[t]{rr@{\,}c@{\,}l@{}}
&x &=& xy\Add{,} \\
&vy &=& v(1 - z)\Add{,} \\
\cline{2-4}
&vx &=& vx(1 - z)\Add{,} \\
&vxz&=& 0\Add{,} \\
\multicolumn{4}{r}{\llap{$\therefore\ \text{Some~$Z$s are not~$X$s.}$}}
\end{array}
\]

The interpretation of~$vz$ as Some~$Z$s, is implied, it will be
observed, in the equation $vz = v(1 - y)$ considered as representing
the proposition Some~$Z$s are not~$Y$s.

The cases not determinable by the Aristotelian Rules are
$OE$, Fig.~1, and, by mutation, $EO$, Fig.~4.
\[
\begin{aligned}[t]
&\text{Some~$Y$s are not~$X$s,} \\
&\text{No~$Z$s are~$Y$s,}
\end{aligned}\qquad
\begin{array}[t]{>{\qquad}rr@{\,}c@{\,}l@{}}
&vy &=& v(1 - x)\Add{,} \\
& 0 &=& zy\Add{,} \\
\cline{2-4}
& 0 &=& v(1 - x)z\Add{,} \\
\multicolumn{4}{c}{\makebox[0pt][c]{$\therefore$\ Some not-$X$s are not~$Z$s.}}
\end{array}
\]

The equation of the first premiss here permits us to interpret
$v(1 - x)$, but it does not enable us to interpret~$vz$.
\PageSep{38}

Of cases in which no inference is possible, we take as
examples---

$AO$, Fig.~1, and, by mutation, $OA$, Fig.~4\Typo{,}{.}
\[
\begin{alignedat}[t]{2}
&\text{All~$Y$s are~$X$s,}\qquad& y(1 - x) &= 0, \\
&\text{Some~$Z$s are not~$Y$s,} & vz &= v(1 - y)\Add{,}\atag
\end{alignedat}\qquad
\begin{array}[t]{r@{\,}c@{\,}l@{}}
y(1 - x) &=& 0\Add{,} \\
v(1 - z) &=& vy\Add{,} \\
\cline{1-3}
v(1 - z)(1 - x) &=& 0\Add{,}\btag \\
0&=& 0\Add{,}
\end{array}
\]
since the auxiliary equation in this case is $v(1 - z) = 0$.

Practically it is not necessary to perform this reduction, but
it is satisfactory to do so. The equation~\aref, it is seen, defines~$vz$
as Some~$Z$s, but it does not define $v(1 - z)$, so that we might
stop at the result of elimination~\bref, and content ourselves with
saying, that it is not interpretable into a relation between the
classes $X$~and~$Z$.

Take as a second example $AI$, Fig.~2, and, by mutation,
$IA$, Fig.~2\Typo{,}{.}
\[
\begin{alignedat}[t]{2}
&\text{All~$X$s are~$Y$s,}\qquad& x(1 - y) &= 0, \\
&\text{Some~$Z$s are~$Y$s,} & vz &= vy,
\end{alignedat}\qquad
\begin{array}[t]{r@{\,}c@{\,}l@{}}
x &=& xy\Add{,} \\
vy &=& vz\Add{,} \\
\cline{1-3}
vx &=& vxz\Add{,} \\
\llap{$v(1 - z)x$}&=& 0\Add{,} \\
0&=& 0,
\end{array}
\]
the auxiliary equation in this case being $v(1 - z)= 0$.

Indeed in every case in this class, in which no inference
is possible, the result of elimination is reducible to the form
$0 = 0$. Examples therefore need not be multiplied.

\textsc{Class} 4th.---When $v$~enters into both equations.

No inference is possible in any case, but there exists a distinction
among the unlawful cases which is peculiar to this
class. The two divisions are,

1st. When the result of elimination is reducible by the
auxiliary equations to the form $0 = 0$. The cases are $II$, $OI$,
\PageSep{39}
Fig.~1; $II$, $OO$, Fig.~2; $II$, $IO$, $OI$, $OO$, Fig.~3; $II$, $IO$,
Fig.~4.

2nd. When the result of elimination is not reducible by the
auxiliary equations to the form $0 = 0$.

The cases are $IO$, $OO$, Fig.~1; $IO$, $OI$, Fig.~2; $OI$, $OO$,
Fig.~4.

Let us take as an example of the former case,~$II$, Fig.~3.
\[
\begin{alignedat}[t]{2}
&\text{Some~$X$s are~$Y$s,}\qquad& vx &= vy, \\
&\text{Some~$Z$s are~$Y$s,} & v'z &= v'y,
\end{alignedat}\qquad
\begin{array}[t]{r@{\,}c@{\,}l@{}}
vx &=& vy\Add{,} \\
v'y &=& v'z\Add{,} \\
\cline{1-3}
vv'x &=& vv'z\Add{.}
\end{array}
\]

Now the auxiliary equations $v(1 - x) = 0$, $v'(1 - z) = 0$,
%[** TN: Next word anomalously displayed in the original]
give
\[
vx = v,\quad v'z = v'.
\]
Substituting we have
\begin{align*}
vv' &= vv', \\
\therefore 0 &= 0.
\end{align*}

As an example of the latter case, let us take $IO$, Fig.~1\Typo{,}{.}
\[
\begin{alignedat}[t]{2}
&\text{Some~$Y$s are~$X$s,}          & vy  &= vx, \\
&\text{Some~$Z$s are not~$Y$s,}\qquad& v'z &= v'(1 - y),
\end{alignedat}\quad
\begin{array}[t]{r@{\,}c@{\,}l@{}}
vy &=& vx\Add{,} \\
v'(1 - z) &=& v'y\Add{,} \\
\cline{1-3}
vv'(1 - z) &=& vv'x\Add{.}
\end{array}
\]

Now the auxiliary equations being $v(1 - x) = 0$, $v'(1 - z) = 0$,
the above reduces to $vv' = 0$. It is to this form that all similar
cases are reducible. Its interpretation is, that the classes $v$
and~$v'$ have no common member, as is indeed evident.

The above classification is purely founded on mathematical
distinctions. We shall now inquire what is the logical division
to which it corresponds.

The lawful cases of the first class comprehend all those in
which, from two universal premises, a universal conclusion
may be drawn. We see that they include the premises of
\textit{barbara} and \textit{celarent} in the first figure, of \textit{cesare} and \textit{camestres}
in the second, and of \textit{bramantip} and \textit{camenes} in the fourth.
\PageSep{40}
The premises of \textit{bramantip} are included, because they admit
of an universal conclusion, although not in the same figure.

The lawful cases of the second class are those in which
a particular conclusion only is deducible from two universal
premises.

The lawful cases of the third class are those in which a
conclusion is deducible from two premises, one of which is
universal and the other particular.

The fourth class has no lawful cases.

Among the cases in which no inference of any kind is possible,
we find six in the fourth class distinguishable from the
others by the circumstance, that the result of elimination does
not assume the form $0 = 0$. The cases are
{\small
\[
\Lbrace{2}\begin{aligned}
&\text{Some~$Y$s are~$X$s,} \\
&\text{Some~$Z$s are not~$Y$s,}
\end{aligned}\Rbrace{2}\quad
%
\Lbrace{2}\begin{aligned}
&\text{Some~$Y$s are not~$X$s,} \\
&\text{Some~$Z$s are not~$Y$s,}
\end{aligned}\Rbrace{2}\quad
%
\Lbrace{2}\begin{aligned}
&\text{Some~$X$s are~$Y$s,} \\
&\text{Some~$Z$s are not~$Y$s,}
\end{aligned}\Rbrace{2}
\]
}%
and the three others which are obtained by mutation of
premises.

It might be presumed that some logical peculiarity would
be found to answer to the mathematical peculiarity which we
have noticed, and in fact there exists a very remarkable one.
If we examine each pair of premises in the above scheme, we
shall find that there \emph{is virtually} no middle term, \emph{\ie~no medium
of comparison}, in any of them. Thus, in the first example,
the individuals spoken of in the first premiss are asserted to
belong to the class~$Y$, but those spoken of in the second
premiss are \emph{virtually} asserted to belong to the class not-$Y$:
nor can we by any lawful transformation or conversion alter
this state of things. The comparison will still be made with
the class~$Y$ in one premiss, and with the class not-$Y$ in the
other.

Now in every case beside the above six, there will be found
a middle term, either expressed or implied. I select two
of the most difficult cases.
\PageSep{41}

In $AO$, Fig.~1, viz.
\begin{align*}
&\text{All~$Y$s are~$X$s,} \\
&\text{Some~$Z$s are not~$Y$s,}
\end{align*}
we have, by \emph{negative conversion} of the first premiss,
\begin{align*}
&\text{All not-$X$s are not-$Y$s,} \\
&\text{Some~$Z$s are not~$Y$s,}
\end{align*}
and the middle term is now seen to be not-$Y$.

Again, in $EO$, Fig.~1,
\begin{align*}
&\text{No~$Y$s are~$X$s,} \\
&\text{Some~$Z$s are not~$Y$s,}
\end{align*}
a proved conversion of the first premiss (see \ChapRef{5}{Conversion of
Propositions}), gives
\begin{align*}
&\text{All~$X$s are not-$Y$s,} \\
&\text{Some~$Z$s are not-$Y$s,}
\end{align*}
and the middle term, the true medium of comparison, is plainly
\Pagelabel{41}%
not-$Y$, although as the not-$Y$s in the one premiss \emph{may be}
different from those in the other, no conclusion can be drawn.

The mathematical condition in question, therefore,---the irreducibility
of the final equation to the form $0 = 0$,---adequately
represents the logical condition of there being no middle term,
or common medium of comparison, in the given premises.

I am not aware that the distinction occasioned by the
presence or absence of a middle term, in the strict sense here
understood, has been noticed by logicians before. The distinction,
though real and deserving attention, is indeed by
no means an obvious one, and it would have been unnoticed
in the present instance but for the peculiarity of its mathematical
expression.

What appears to be novel in the above case is the proof
of the existence of combinations of premises in which there
\PageSep{42}
is absolutely no medium of comparison. When such a medium
of comparison, or true middle term, does exist, the condition
that its quantification in both premises together shall exceed
its quantification as a single whole, has been ably and
\Pagelabel{42}%
clearly shewn by Professor De~Morgan to be necessary to
lawful inference (\textit{Cambridge Memoirs}, Vol.~\textsc{viii}.\ Part~3). And
this is undoubtedly the true principle of the Syllogism, viewed
from the standing-point of Arithmetic.

I have said that it would be possible to impose conditions
of interpretation which should restrict the results of this calculus
to the Aristotelian forms. Those conditions would be,

1st. That we should agree not to interpret the forms $v(1 - x)$,
$v(1 - z)$.

2ndly. That we should agree to reject every interpretation in
which the order of the terms should violate the Aristotelian rule.

Or, instead of the second condition, it might be agreed that,
the conclusion being determined, the order of the premises
should, if necessary, be changed, so as to make the syllogism
formal.

From the \emph{general} character of the system it is indeed plain,
that it may be made to represent any conceivable scheme of
logic, by imposing the conditions proper to the case contemplated.

We have found it, in a certain class of cases, to be necessary
to replace the two equations expressive of universal Propositions,
by their solutions; and it may be proper to remark,
that it would have been allowable in all instances to have
done this,\footnote
  {It may be satisfactory to illustrate this statement by an example. In
  \textit{\Chg{Barbara}{barbara}}, we should have
  \[
  \begin{aligned}[t]
    &\text{All~$Y$s are~$X$s,} \\
    &\text{All~$Z$s are~$Y$s,}
  \end{aligned}\qquad
  \begin{array}[t]{>{\qquad}r@{\,}c@{\,}l@{}}
    y &=& vx\Add{,} \\
    z &=& v'y\Add{,} \\
    \cline{1-3}
    z &=& vv'x\Add{,} \\
    \multicolumn{3}{c}{\makebox[0pt][c]{$\therefore$\ All~$Z$s are~$X$s.}}
  \end{array}
  \]
%[** TN: Footnote continues]
  Or, we may multiply the resulting equation by~$1 - x$, which gives
  \[
  z(1 - x) = 0,
  \]
  whence the same conclusion, All~$Z$s are~$X$s.

  Some additional examples of the application of the system of equations in
  the text to the demonstration of general theorems, may not be inappropriate.

  Let $y$ be the term to be eliminated, and let $x$ stand indifferently for either of
  the other symbols, then each of the equations of the premises of any given
  syllogism may be put in the form
  \[
  ay + bx = 0,
  \GrTag[a]{(\alpha)}
  \]
  if the premiss is affirmative, and in the form
  \[
  ay + b(1 - x) = 0,
  \GrTag[b]{(\beta)}
  \]
  if it is negative, $a$~and~$b$ being either constant, or of the form~$�v$. To prove
  this in detail, let us examine each kind of proposition, making $y$~successively
  subject and predicate.
  \begin{alignat*}{2}
    A,\ &\text{All~$Y$s are~$X$s,} & y - vx &= 0,
    \GrTag[c]{(\gamma)} \\
       &\text{All~$X$s are~$Y$s,} & x - vy &= 0,
    \GrTag[d]{(\delta)} \\
%
    E,\ &\text{No~$Y$s are~$X$s,}  & xy &= 0, \\
       &\text{No~$X$s are~$Y$s,}  & y - v(1 - x) &= 0,
    \GrTag[e]{(\epsilon)} \\
%
    I,\ &\text{Some~$X$s are~$Y$s,} && \\
       &\text{Some~$Y$s are~$X$s,} &vx - vy &= 0,
    \GrTag[f]{(\zeta)} \\
%
    O,\ &\text{Some~$Y$s are not~$X$s,}\qquad& vy - v(1 - x) &= 0,
    \GrTag[g]{(\eta)} \\
       &\text{Some~$X$s are not~$Y$s,} &      vx &= v(1 - y), \\
       && \therefore vy - v(1 - x) &= 0.
    \GrTag[h]{(\theta)}
  \end{alignat*}

  The affirmative equations \GrEq[c]{(\gamma)},~\GrEq[d]{(\delta)} and~\GrEq[f]{(\zeta)}, belong to~\GrEq[a]{(\alpha)}, and the negative
  equations \GrEq[e]{(\epsilon)},~\GrEq[g]{(\eta)} and~\GrEq[h]{(\theta)}, to~\GrEq[b]{(\beta)}. It is seen that the two last negative equations
  are alike, but there is a difference of interpretation. In the former
  \[
  v(1 - x) = \text{Some not-$X$s,}
  \]
  in the latter,
  \[
  v(1 - x) = 0.
  \]

  The utility of the two general forms of reference, \GrEq[a]{(\alpha)}~and~\GrEq[b]{(\beta)}, will appear
  from the following application.

  1st. \emph{A conclusion drawn from two affirmative propositions} is itself affirmative.

  By \GrEq[a]{(\alpha)} we have for the given propositions,
  \begin{alignat*}{2}
    ay  &+ bx  &&= 0, \\
    a'y &+ b'z &&= 0,
  \end{alignat*}
%[** TN: Footnote continues]
  and eliminating
  \[
  ab'z - a'bx = 0,
  \]
  which is of the form~\GrEq[a]{(\alpha)}. Hence, if there is a conclusion, it is affirmative.

  2nd. \emph{A conclusion drawn from an affirmative and a negative proposition is
negative.}

  By \GrEq[a]{(\alpha)}~and~\GrEq[b]{(\beta)}, we have for the given propositions
  \begin{align*}
    ay + bx &= 0, \\
    a'y + b'(1 - z) &= 0, \\
    \therefore\ a'bx - ab'(1 - z) &= 0,
  \end{align*}
  which is of the form~\GrEq[b]{(\beta)}. Hence the conclusion, if there is one, is negative.

  3rd. \emph{A conclusion drawn from two negative premises will involve a negation,
  \(not-$X$, not-$Z$\) in both subject and predicate, and will therefore be inadmissible in
  the Aristotelian system, though just in itself.}

  For the premises being
  \begin{alignat*}{2}
     ay &+ b (1 - x) &&= 0, \\
    a'y &+ b'(1 - z) &&= 0,
  \end{alignat*}
  the conclusion will be
  \[
  ab'(1 - z) - a'b(1 - x) = 0,
  \]
  which is only interpretable into a proposition that has a negation in each term.

  4th. \emph{Taking into account those syllogisms only, in which the conclusion is the
  most general, that can be deduced from the premises,---if, in an Aristotelian
  syllogism, the minor premises be changed in quality \(from affirmative to negative
  or from negative to affirmative\), whether it be changed in quantity or not, no conclusion
  will be deducible in the same figure.}

  An Aristotelian proposition does not admit a term of the form not-$Z$ in the
  subject,---Now on changing the quantity of the minor proposition of a syllogism,
  we transfer it from the general form
  \begin{align*}
    ay + bz &= 0, \\
  \intertext{to the general form}
    a'y + b'(1 - z) &= 0,
  \end{align*}
  see \GrEq[a]{(\alpha)}~\emph{and}~\GrEq[b]{(\beta)}, or \textit{vice vers�}. And therefore, in the equation of the conclusion,
  there will be a change from~$z$ to~$1 - z$, or \textit{vice vers�}. But this is equivalent to
  the change of~$Z$ into not-$Z$, or not-$Z$ into~$Z$. Now the subject of the original
  conclusion must have involved a~$Z$ and not a not-$Z$, therefore the subject of the
  new conclusion will involve a not-$Z$, and the conclusion will not be admissible
  in the Aristotelian forms, except by conversion, which would render necessary
  a change of Figure.

  Now the conclusions of this calculus are always the most general that can be
  drawn, and therefore the above demonstration must not be supposed to extend
  to a syllogism, in which a particular conclusion is deduced, when a universal
  one is possible. This is the case with \textit{bramantip} only, among the Aristotelian
  forms, and therefore the transformation of \textit{bramantip} into \textit{camenes}, and \textit{vice vers�},
  is the case of restriction contemplated in the preliminary statement of the
  theorem.

  5th. \emph{If for the minor premiss of an Aristotelian syllogism, we substitute its contradictory,
  no conclusion is deducible in the same figure.}

  It is here only necessary to examine the case of \textit{bramantip}, all the others
  being determined by the last proposition.

  On changing the minor of \textit{bramantip} to its contradictory, we have $AO$,
  Fig.~4, and this admits of no legitimate inference.

  Hence the theorem is true without exception. Many other general theorems
  may in like manner be proved.}
%[** TN: End of 3.5-page footnote]
so that every case of the Syllogism, without exception,
\PageSep{43}
might have been treated by equations comprised in
the general forms
\Pagelabel{43}%
\begin{alignat*}{3}
 y &= vx,            &&\text{or}      & y - vx &= 0,
\Ltag{A} \\
 y &= v(1 - x),\qquad&&\text{or}\quad & y + vx - v &= 0,
\Ltag{E} \\
vy &= vx,            &&&            vy - vx &= 0,
\Ltag{I} \\
vy &= v(1 - x),      &&&            vy + vx - v &= 0.
\Ltag{O}
\end{alignat*}
\PageSep{44}

Perhaps the system we have actually employed is better,
as distinguishing the cases in which $v$~only \emph{may} be employed,
\PageSep{45}
from those in which it \emph{must}. But for the demonstration of
certain general properties of the Syllogism, the above system
is, from its simplicity, and from the mutual analogy of its
forms, very convenient. We shall apply it to the following
theorem.\footnote
  {This elegant theorem was communicated by the Rev.\ Charles Graves,
  Fellow and Professor of Mathematics in Trinity College, Dublin, to whom the
  Author desires further to record his grateful acknowledgments for a very
  judicious examination of the former portion of this work, and for some new
  applications of the method. The following example of Reduction \textit{ad~impossibile}
  is among the number:
  \[
  \begin{array}{rl<{\quad}r@{\,}c@{\,}l@{}}
    \text{Reducend Mood,} &
    \text{All~$X$s are~$Y$s,} &
    1 - y &=& v'(1 - x)\Add{,} \\
    \PadTxt{Reducend Mood,}{\textit{\Chg{Baroko}{baroko}}} &
    \text{Some~$Z$s are not~$Y$s\Add{,}} &
    vz &=& v(1 - y)\Add{,} \\
    \cline{3-5}
%
    &\text{Some~$Z$s are not~$X$s\Add{,}} &
    vz &=& vv'(1 - x)\Add{,} \\
%
    \text{Reduct Mood,} &
    \text{All~$X$s are~$Y$s\Add{,}} &
    1 - y &=& v'(1 - x)\Add{,} \\
    \PadTxt{Reduct Mood,}{\textit{\Chg{Barbara}{barbara}}} &
    \text{All~$Z$s are~$X$s\Add{,}} &
    z(1 - x) &=& 0\Add{,} \\
    \cline{2-5}
    &\text{All~$Z$s are~$Y$s\Add{,}} &
    z(1 - y) &=& 0.
  \end{array}
  \]

  The conclusion of the reduct mood is seen to be the contradictory of the
  suppressed minor premiss. Whence,~\etc. It may just be remarked that the
  mathematical test of contradictory propositions is, that on eliminating one
  elective symbol between their equations, the other elective symbol vanishes.
  The \emph{ostensive} reduction of \textit{\Chg{Baroko}{baroko}} and \textit{\Chg{Bokardo}{bokardo}} involves no difficulty.

  Professor Graves suggests the employment of the equation $x = vy$ for the
  primary expression of the Proposition All~$X$s are~$Y$s, and remarks, that on
  multiplying both members by~$1 - y$, we obtain $x(1 - y) = 0$, the equation from
  which we set out in the text, and of which the previous one is a solution.}

Given the three propositions of a Syllogism, prove that there
is but one order in which they can be legitimately arranged,
and determine that order.

All the forms above given for the expression of propositions,
are particular cases of the general form,
\[
a + bx + cy = 0.
\]
\PageSep{46}

Assume then for the premises of the given syllogism, the
equations
\begin{alignat*}{3}
a  &+ bx &&+ cy &&= 0,
\Tag{(18)} \\
a' &+ b'z &&+ c'y &&= 0,
\Tag{(19)}
\end{alignat*}
then, eliminating~$y$, we shall have for the conclusion
\[
ac' - a'c + bc'x - b'cz = 0.
\Tag{(20)}
\]

Now taking this as one of our premises, and either of the
original equations, suppose~\Eqref{(18)}, as the other, if by elimination
of a common term~$x$, between them, we can obtain a result
equivalent to the remaining premiss~\Eqref{(19)}, it will appear that
there are more than one order in which the Propositions may
be lawfully written; but if otherwise, one arrangement only
is lawful.

Effecting then the elimination, we have
\[
bc(a' + b'z + c'y) = 0,
\Tag{(21)}
\]
which is equivalent to~\Eqref{(19)} multiplied by a factor~$bc$. Now on
examining the value of this factor in the equations $A$,~$E$, $I$,~$O$,
we find it in each case to be $v$~or~$-v$. But it is evident,
that if an equation expressing a given Proposition be multiplied
by an extraneous factor, derived from another equation,
its interpretation will either be limited or rendered
impossible. Thus there will either be no result at all, or the
result will be a \emph{limitation} of the remaining Proposition.

If, however, one of the original equations were
\[
x = y,\quad\text{or}\quad x - y = 0,
\]
the factor~$bc$ would be~$-1$, and would \emph{not} limit the interpretation
of the other premiss. Hence if the first member of
a syllogism should be understood to represent the double
proposition All~$X$s are~$Y$s, and All~$Y$s are~$X$s, it would be
indifferent in what order the remaining Propositions were
written.
\PageSep{47}

A more general form of the above investigation would be,
to express the premises by the equations
\begin{alignat*}{4}
a  &+ bx  &&+ cy  &&+ dxy  &&= 0,
\Tag{(22)} \\
a' &+ b'z &&+ c'y &&+ d'zy &&= 0.
\Tag{(23)}
\end{alignat*}

After the double elimination of $y$~and~$x$ we should find
\[
(bc - ad)(a' + b'z + c'y + d'zy) = 0;
\]
and it would be seen that the factor $bc - ad$ must in every
case either vanish or express a limitation of meaning.

The determination of the order of the Propositions is sufficiently
obvious.
\PageSep{48}


\Chapter{Of Hypotheticals.}

\begin{Abstract}
A hypothetical Proposition is defined to be \emph{two or more categoricals united by
a copula} (or conjunction), and the different kinds of hypothetical Propositions
are named from their respective conjunctions, viz.\ conditional (if), disjunctive
(either, or),~\etc.

In conditionals, that categorical Proposition from which the other results
is called the \emph{antecedent}, that which results from it the \emph{consequent}.

Of the conditional syllogism there are two, and only two formul�.

1st. The constructive,
\begin{gather*}
\text{If $A$~is~$B$, then $C$~is~$D$,} \\
\text{But $A$~is~$B$, therefore $C$~is~$D$.}
\end{gather*}

2nd. The Destructive,
\begin{gather*}
\text{If $A$~is~$B$, then $C$~is~$D$,} \\
\text{But $C$~is not~$D$, therefore $A$~is not~$B$.}
\end{gather*}

A dilemma is a complex conditional syllogism, with several antecedents
in the major, and a disjunctive minor.
\end{Abstract}

\First{If} we examine either of the forms of conditional syllogism
above given, we shall see that the validity of the argument
does not depend upon any considerations which have reference
to the terms $A$,~$B$,~$C$,~$D$, considered as the representatives
of individuals or of classes. We may, in fact, represent the
Propositions $A$~is~$B$, $C$~is~$D$, by the arbitrary symbols $X$~and~$Y$
respectively, and express our syllogisms in such forms as the
following:
\begin{gather*}
\text{If $X$ is true, then $Y$ is true,} \\
\text{But $X$ is true, therefore $Y$ is true.}
\end{gather*}

Thus, what we have to consider is not objects and classes
of objects, but the truths of Propositions, namely, of those
\PageSep{49}
elementary Propositions which are embodied in the terms of
our hypothetical premises.

To the symbols $X$,~$Y$,~$Z$, representative of Propositions, we
may appropriate the elective symbols $x$,~$y$,~$z$, in the following
sense.

The hypothetical Universe,~$1$, shall comprehend all conceivable
cases and conjunctures of circumstances.

The elective symbol~$x$ attached to any subject expressive of
such cases shall select those cases in which the Proposition~$X$
is true, and similarly for $Y$~and~$Z$.

If we confine ourselves to the contemplation of a given proposition~$X$,
and hold in abeyance every other consideration,
then two cases only are conceivable, viz.\ first that the given
Proposition is true, and secondly that it is false.\footnote
  {It was upon the obvious principle that a Proposition is either true or false,
  that the Stoics, applying it to assertions respecting future events, endeavoured
  to establish the doctrine of Fate. It has been replied to their argument, that it
%[** TN: Italicized entire Latin phrase; only "est" italicized in original]
  involves ``an abuse of the word \emph{true}, the precise meaning of which is \textit{id quod
  res est}. An assertion respecting the future is neither true nor false.''---\textit{Copleston
  on Necessity and Predestination}, p.~36. Were the Stoic axiom, however, presented
  under the form, It is either certain that a given event will take place,
  or certain that it will not; the above reply would fail to meet the difficulty.
  The proper answer would be, that no merely verbal definition can settle the
  question, what is the actual course and constitution of Nature. When we
  affirm that it is either certain that an event will take place, or certain that
  it will not take place, we tacitly assume that the order of events is necessary,
  that the Future is but an evolution of the Present; so that the state of things
  which is, completely determines that which shall be. But this (at least as respects
  the conduct of moral agents) is the very question at issue. Exhibited
  under its proper form, the Stoic reasoning does not involve an abuse of terms,
  but a \textit{petitio principii}.

  It should be added, that enlightened advocates of the doctrine of Necessity
  in the present day, viewing the end as appointed only in and through the
  means, justly repudiate those practical ill consequences which are the reproach
  of Fatalism.}
As these
cases together make up the Universe of the Proposition, and
as the former is determined by the elective symbol~$x$, the latter
is determined by the symbol~$1 - x$.

But if other considerations are admitted, each of these cases
will be resolvable into others, individually less extensive, the
\PageSep{50}
number of which will depend upon the number of foreign considerations
admitted. Thus if we associate the Propositions $X$
and~$Y$, the total number of conceivable cases will be found as
exhibited in the following scheme.
\[
\begin{array}[b]{*{2}{l@{\ }}>{\qquad}c@{}}
\multicolumn{2}{c}{\ColHead{Cases.}} &
\multicolumn{1}{>{\qquad}c}{\ColHead{Elective expressions.}} \\
\text{1st}& \text{$X$ true,  $Y$ true\Add{,}} & xy\Add{,} \\
\text{2nd}& \text{$X$ true,  $Y$ false\Add{,}}& x(1 - y)\Add{,} \\
\text{3rd}& \text{$X$ false, $Y$ true\Add{,}} & (1 - x)y\Add{,} \\
\text{4th}& \text{$X$ false, $Y$ false\Add{,}}& (1 - x)(1 - y)\Add{.}
\end{array}
\Tag{(24)}
\]

If we add the elective expressions for the two first of the
above cases the sum is~$x$, which is the elective symbol appropriate
to the more general case of $X$~being true independently
of any consideration of~$Y$; and if we add the elective expressions
in the two last cases together, the result is~$1 - x$, which
is the elective expression appropriate to the more general case
of $X$~being false.

Thus the extent of the hypothetical Universe does not at
all depend upon the number of circumstances which are taken
into account. And it is to be noted that however few or many
those circumstances may be, the sum of the elective expressions
representing every conceivable case will be unity. Thus let
us consider the three Propositions, $X$,~It rains, $Y$,~It hails,
$Z$,~It freezes. The possible cases are the following:
\[
\begin{array}{*{2}{l@{\ }}l@{}}
&\multicolumn{1}{c}{\ColHead{Cases.}} &
\multicolumn{1}{c}{\ColHead{Elective expressions.}} \\
\text{1st}& \text{It rains, hails, and freezes,} & xyz\Add{,} \\
\text{2nd}& \text{It rains and hails, but does not freeze\Add{,}}& xy(1 - z)\Add{,} \\
\text{3rd}& \text{It rains and freezes, but does not hail\Add{,}}& xz(1 - y)\Add{,} \\
\text{4th}& \text{It freezes and hails, but does not rain\Add{,}}& yz(1 - x)\Add{,} \\
\text{5th}& \text{It rains, but neither hails nor freezes\Add{,}}& x(1 - y)(1 - z)\Add{,} \\
\text{6th}& \text{It hails, but neither rains nor freezes\Add{,}}& y(1 - x)(1 - z)\Add{,} \\
\text{7th}& \text{It freezes, but neither hails nor rains\Add{,}}& z(1 - x)(1 - y)\Add{,} \\
\text{8th}& \text{It neither rains, hails, nor freezes\Add{,}}& (1 - x)(1 - y)(1 - z)\Add{,} \\
\cline{3-3}
&&\multicolumn{1}{c}{1 = \text{sum\Add{.}}}
\end{array}
\]
\PageSep{51}


\Section{Expression of Hypothetical Propositions.}

To express that a given Proposition~$X$ is true.

The symbol $1 - x$ selects those cases in which the Proposition~$X$
is false. But if the Proposition is true, there are no
such cases in its hypothetical Universe, therefore
\begin{align*}
1 - x &= 0, \\
\intertext{or}
x &= 1.
\Tag{(25)}
\end{align*}

To express that a given Proposition~$X$ is false.

The elective symbol~$x$ selects all those cases in which the
Proposition is true, and therefore if the Proposition is false,
\[
x = 0.
\Tag{(26)}
\]

And in every case, having determined the elective expression
appropriate to a given Proposition, we assert the truth of that
Proposition by equating the elective expression to unity, and
its falsehood by equating the same expression to~$0$.

To express that two Propositions, $X$~and~$Y$, are simultaneously
true.

The elective symbol appropriate to this case is~$xy$, therefore
the equation sought is
\[
xy = 1.
\Tag{(27)}
\]

To express that two Propositions, $X$~and~$Y$, are simultaneously
false.

The condition will obviously be
\begin{align*}
(1 - x)(1 - y) &= 1, \\
\intertext{or}
x + y - xy &= 0.
\Tag{(28)}
\end{align*}

To express that either the Proposition~$X$ is true, or the
Proposition~$Y$ is true.

To assert that either one or the other of two Propositions
is true, is to assert that it is not true, that they are both false.
Now the elective expression appropriate to their both being
false is~$(1 - x)(1 - y)$, therefore the equation required is
\begin{align*}
(1 - x)(1 - y) &= 0, \\
\intertext{or}
x + y - xy &= 1.
\Tag{(29)}
\end{align*}
\PageSep{52}

And, by indirect considerations of this kind, may every disjunctive
Proposition, however numerous its members, be expressed.
But the following general Rule will usually be
preferable.

\begin{Rule}
Consider what are those distinct and mutually exclusive
cases of which it is implied in the statement of the given Proposition,
that some one of them is true, and equate the sum of their
elective expressions to unity. This will give the equation of the
given Proposition.
\end{Rule}

For the sum of the elective expressions for all distinct conceivable
cases will be unity. Now all these cases being mutually
exclusive, and it being asserted in the given Proposition that
some one case out of a given set of them is true, it follows that
all which are not included in that set are false, and that their
elective expressions are severally equal to~$0$. Hence the sum
of the elective expressions for the remaining cases, viz.\ those
included in the given set, will be unity. Some one of those
cases will therefore be true, and as they are mutually exclusive,
it is impossible that more than one should be true. Whence
the Rule in question.

And in the application of this Rule it is to be observed, that
if the cases contemplated in the given disjunctive Proposition
are not mutually exclusive, they must be resolved into an equivalent
series of cases which are mutually exclusive.

Thus, if we take the Proposition of the preceding example,
viz.\ Either $X$~is true, or $Y$~is true, and assume that the two
members of this Proposition are not exclusive, insomuch that
in the enumeration of possible cases, we must reckon that of
the Propositions $X$~and~$Y$ being both true, then the mutually
exclusive cases which fill up the Universe of the Proposition,
with their elective expressions, are
\[
\begin{array}{l@{\ }l<{\qquad}c@{}}
\text{1st,}& \text{$X$~true and $Y$~false,}& x(1 - y), \\
\text{2nd,}& \text{$Y$~true and $X$~false,}& y(1 - x), \\
\text{3rd,}& \text{$X$~true and $Y$~true,} & xy,
\end{array}
\]
\PageSep{53}
and the sum of these elective expressions equated to unity gives
\[
x + y - xy = 1\Typo{.}{,}
\Tag{(30)}
\]
as before. But if we suppose the members of the disjunctive
Proposition to be exclusive, then the only cases to be considered
are
\[
\begin{array}{l@{\ }l<{\qquad}c@{}}
\text{1st,}& \text{$X$~true, $Y$~false,}& x(1 - y), \\
\text{2nd,}& \text{$Y$~true, $X$~false,}& y(1 - x),
\end{array}
\]
and the sum of these elective expressions equated to~$0$, gives
\[
x - 2xy + y = 1.
\Tag{(31)}
\]

The subjoined examples will further illustrate this method.

To express the Proposition, Either $X$~is not true, or $Y$~is not
true, the members being exclusive.

The mutually exclusive cases are
\[
\begin{array}{l@{\ }l<{\qquad}c@{}}
\text{1st,}& \text{$X$~not true, $Y$~true,}& y(1 - x), \\
\text{2nd,}& \text{$Y$~not true, $X$~true,}& x(1 - y),
\end{array}
\]
and the sum of these equated to unity gives
\[
x - 2xy + y = 1,
\Tag{(32)}
\]
which is the same as~\Eqref{(31)}, and in fact the Propositions which
they represent are equivalent.

To express the Proposition, Either $X$~is not true, or $Y$~is not
true, the members not being exclusive.

To the cases contemplated in the last Example, we must add
the following, viz.
\[
\text{$X$~not true, $Y$~not true,}\qquad (1 - x)(1 - y).
\]

The sum of the elective expressions gives
\begin{gather*}
x(1 - y) + y(1 - x) + (1 - x)(1 - y) = 1, \\
\intertext{or}
xy = 0.
\Tag{(33)}
\end{gather*}

To express the disjunctive Proposition, Either $X$~is true, or
$Y$~is true, or $Z$~is true, the members being exclusive.
\PageSep{54}

Here the mutually exclusive cases are
\[
\begin{array}{l@{\ }l<{\qquad}c@{}}
\text{1st,}& \text{$X$~true, $Y$~false, $Z$~false,}& x(1 - y)(1 - z), \\
\text{2nd,}& \text{$Y$~true, $Z$~false, $X$~false,}& y(1 - z)(1 - x), \\
\text{3rd,}& \text{$Z$~true, $X$~false, $Y$~false,}& z(1 - x)(1 - y),
\end{array}
\]
and the sum of the elective expressions equated to~$1$, gives,
upon reduction,
\[
x + y + z - 2(xy + yz + zx) + 3xyz = 1.
\Tag{(34)}
\]

The expression of the same Proposition, when the members
are in no sense exclusive, will be
\[
(1 - x)(1 - y)(1 - z) = 0.
\Tag{(35)}
\]

And it is easy to see that our method will apply to the
expression of any similar Proposition, whose members are
subject to any specified amount and character of exclusion.

To express the conditional Proposition, If $X$~is true, $Y$~is
true.

Here it is implied that all the cases of $X$~being true, are
cases of $Y$~being true. The former cases being determined
by the elective symbol~$x$, and the latter by~$y$, we have, in
virtue of~\Eqref{(4)},
\[
x(1 - y) = 0.
\Tag{(36)}
\]

To express the conditional Proposition, If $X$~be true, $Y$~is
not true.

The equation is obviously
\[
xy = 0;
\Tag{(37)}
\]
this is equivalent to~\Eqref{(33)}, and in fact the disjunctive Proposition,
Either $X$~is not true, or $Y$~is not true, and the conditional
Proposition, If $X$~is true, $Y$~is not true, are equivalent.

To express that If $X$~is not true, $Y$~is not true.

In~\Eqref{(36)} write $1 - x$ for~$x$, and $1 - y$ for~$y$, we have
\[
(1 - x)y = 0.
\]
\PageSep{55}

The results which we have obtained admit of verification
in many different ways. Let it suffice to take for more particular
examination the equation
\[
x - 2xy + y = 1,
\Tag{(38)}
\]
which expresses the conditional Proposition, Either $X$~is true,
or $Y$~is true, the members being in this case exclusive.

First, let the Proposition~$X$ be true, then $x = 1$, and substituting,
we have
\[
1 - 2y + y = 1,\qquad
\therefore -y = 0,\quad\text{or}\quad y = 0,
\]
which implies that $Y$~is not true.

Secondly, let $X$~be not true, then $x = 0$, and the equation
gives
\[
y = 1,
\Tag{(39)}
\]
which implies that $Y$~is true. In like manner we may proceed
with the assumptions that $Y$~is true, or that $Y$~is false.

Again, in virtue of the property $x^{2} = x$, $y^{2} = y$, we may write
the equation in the form
\[
x^{2} - 2xy + y^{2} = 1,
\]
and extracting the square root, we have
\[
x - y = �1,
\Tag{(40)}
\]
and this represents the actual case; for, as when $X$~is true
or false, $Y$~is respectively false or true, we have
\begin{gather*}
x = 1\quad\text{or}\quad 0, \\
y = 0\quad\text{or}\quad 1, \\
\therefore x - y = 1\quad\text{or}\quad -1.
\end{gather*}

There will be no difficulty in the analysis of other cases.


\Section{Examples of Hypothetical Syllogism.}

The treatment of every form of hypothetical Syllogism will
consist in forming the equations of the premises, and eliminating
the symbol or symbols which are found in more than one of
them. The result will express the conclusion.
\PageSep{56}

1st. Disjunctive Syllogism.
\begin{align*}
&\begin{array}{l<{\qquad}@{}c@{}}
\text{Either $X$~is true, or $Y$~is true (exclusive),} &
x + y - 2xy = 1\Add{,} \\
\text{But $X$~is true,} & x = 1\Add{,} \\
\cline{2-2}
\text{Therefore $Y$~is not true,} & \therefore y = 0\Add{.}
\end{array} \\
&\begin{array}{l<{\quad}@{}c@{}}
\text{Either $X$~is true, or $Y$~is true (not exclusive),}&
x + y - xy = 1\Add{,} \\
\text{But $X$~is not true,}& x = 0\Add{,} \\
\cline{2-2}
\text{Therefore $Y$~is true,}& \therefore y = 1\Add{.}
\end{array}
\end{align*}

2nd. Constructive Conditional Syllogism.
\[
\begin{array}{l<{\qquad}@{}c@{}}
\text{If $X$~is true, $Y$~is true,}& x(1 - y) = 0\Add{,} \\
\text{But $X$~is true,}& x = 1\Add{,} \\
\text{Therefore $Y$~is true,}& \therefore 1 - y = 0\quad\text{or}\quad y = 1.
\end{array}
\]

3rd. Destructive Conditional Syllogism.
\[
\begin{array}{l<{\qquad}@{}r@{}}
\text{If $X$~is true, $Y$~is true,}& x(1 - y) = 0\Add{,} \\
\text{But $Y$~is not true,}& y = 0\Add{,} \\
\text{Therefore $X$~is not true,}& \therefore x = 0\Add{.}
\end{array}
\]

4th. Simple Constructive Dilemma, the minor premiss exclusive.
\begin{alignat*}{2}
&\text{If $X$~is true, $Y$~is true,}& x(1 - y) &= 0,
\Tag{(41)} \\
&\text{If $Z$~is true, $Y$~is true,}& z(1 - y) &= 0,
\Tag{(42)} \\
&\text{But Either $X$~is true, or $Z$~is true,}\quad&
x + z - 2xz &= 1.
\Tag{(43)}
\end{alignat*}

From the equations \Eqref{(41)},~\Eqref{(42)},~\Eqref{(43)}, we have to eliminate
$x$~and~$z$. In whatever way we effect this, the result is
\[
y = 1;
\]
whence it appears that the Proposition~$Y$ is true.

5th. Complex Constructive Dilemma, the minor premiss not
exclusive.
\[
\begin{array}{l<{\qquad}@{}r@{}}
\text{If $X$~is true, $Y$~is true,}& x(1 - y) = 0, \\
\text{If $W$~is true, $Z$~is true,}& w(1 - z) = 0, \\
\text{Either $X$~is true, or $W$~is true,}& x + w - xw = 1.
\end{array}
\]

From these equations, eliminating~$x$, we have
\[
y + z - yz = 1,
\]
\PageSep{57}
which expresses the Conclusion, Either $Y$~is true, or $Z$~is true,
the members being \Chg{non-exclusive}{nonexclusive}.

6th. Complex Destructive Dilemma, the minor premiss exclusive.
\[
\begin{array}{l<{\qquad}@{}r@{}}
\text{If $X$~is true, $Y$~is true,}& x(1 - y) = 0\Add{,} \\
\text{If $W$~is true, $Z$~is true,}& w(1 - z) = 0\Add{,} \\
\text{Either $Y$~is not true, or $Z$~is not true,}& y + z - 2yz = 1.
\end{array}
\]

From these equations we must eliminate $y$~and~$z$. The
result is
\[
xw = 0,
\]
which expresses the Conclusion, Either $X$~is not true, or $Y$~is
not true, the members \emph{not being exclusive}.

7th. Complex Destructive Dilemma, the minor premiss not
exclusive.
\[
\begin{array}{l<{\qquad}@{}r@{}}
\text{If $X$~is true, $Y$~is true,}& x(1 - y) = 0\Add{,} \\
\text{If $W$~is true, $Z$~is true,}& w(1 - z) = 0\Add{,} \\
\text{Either $Y$~is not true, or $Z$~is not true,}& yz = 0.
\end{array}
\]

On elimination of $y$~and~$z$, we have
\[
xw = 0,
\]
which indicates the same Conclusion as the previous example.

It appears from these and similar cases, that whether the
members of the minor premiss of a Dilemma are exclusive
or not, the members of the (disjunctive) Conclusion are never
exclusive. This fact has perhaps escaped the notice of logicians.

The above are the principal forms of hypothetical Syllogism
which logicians have recognised. It would be easy, however,
to extend the list, especially by the blending of the disjunctive
and the conditional character in the same Proposition, of which
the following is an example.
\[
\begin{array}{l<{\qquad}@{}c@{}}
\multicolumn{2}{l}{%
  \text{If $X$~is true, then either $Y$~is true, or $Z$~is true,}} \\
                           & x(1 - y - z + yz) = 0\Add{,} \\
\text{But $Y$~is not true,}& y = 0\Add{,} \\
\text{Therefore If $X$~is true, $Z$~is true,}& \therefore x(1 - z) = 0.
\end{array}
\]
\PageSep{58}

That which logicians term a \emph{Causal} Proposition is properly
a conditional Syllogism, the major premiss of which is suppressed.

The assertion that the Proposition~$X$ is true, \emph{because} the
Proposition~$Y$ is true, is equivalent to the assertion,
\begin{align*}
&\text{The Proposition~$Y$ is true,} \\
&\text{\emph{Therefore} the Proposition X is true;}
\end{align*}
and these are the minor premiss and conclusion of the conditional
Syllogism,
\begin{align*}
&\text{If $Y$~is true, $X$~is true,} \\
&\text{But $Y$~is true,} \\
&\text{Therefore $X$~is true.}
\end{align*}
And thus causal Propositions are seen to be included in the
applications of our general method.

Note, that there is a family of disjunctive and conditional
Propositions, which do not, of right, belong to the class considered
in this Chapter. Such are those in which the force
of the disjunctive or conditional particle is expended upon the
predicate of the Proposition, as if, speaking of the inhabitants
of a particular island, we should say, that they are all \emph{either
Europeans or Asiatics}; meaning, that it is true of each individual,
that he is either a European or an Asiatic. If we
appropriate the elective symbol~$x$ to the inhabitants, $y$~to
Europeans, and $z$~to Asiatics, then the equation of the above
Proposition is
\[
x = xy + xz,\quad\text{or}\quad x(1 - y - z) = 0;\atag
\]
to which we might add the condition $yz = 0$, since no Europeans
are Asiatics. The nature of the symbols $x$,~$y$,~$z$, indicates that
the Proposition belongs to those which we have before designated
as \emph{Categorical}. Very different from the above is the
Proposition, Either all the inhabitants are Europeans, or they
are all Asiatics. Here the disjunctive particle separates Propositions.
The case is that contemplated in~\Eqref{(31)} of the present
Chapter; and the symbols by which it is expressed,
\PageSep{59}
although subject to the same laws as those of~\aref, have a totally
different interpretation.\footnote
  {Some writers, among whom is Dr.\ Latham (\textit{First Outlines}), regard it as
  the exclusive office of a conjunction to connect \emph{Propositions}, not \emph{words}. In this
  view I am not able to agree. The Proposition, Every animal is \emph{either} rational
  \emph{or} irrational, cannot be resolved into, \emph{Either} every animal is rational, \emph{or} every
  animal is irrational. The former belongs to pure categoricals, the latter to
  hypotheticals. In \emph{singular} Propositions, such conversions would seem to be
  allowable. This animal is \emph{either} rational \emph{or} irrational, is equivalent to, \emph{Either}
  this animal is rational, \emph{or} it is irrational. This peculiarity of \emph{singular} Propositions
  would almost justify our ranking them, though truly universals, in
  a separate class, as Ramus and his followers did.}

The distinction is real and important. Every Proposition
which language can express may be represented by elective
symbols, and the laws of combination of those symbols are in
all cases the same; but in one class of instances the symbols
have reference to collections of objects, in the other, to the
truths of constituent Propositions.
\PageSep{60}


\Chapter{Properties of Elective Functions.}

\First{Since} elective symbols combine according to the laws of
quantity, we may, by Maclaurin's theorem, expand a given
function~$\phi(x)$, in ascending powers of~$x$, known cases of failure
excepted. Thus we have
\[
\phi(x) = \phi(0) + \phi'(0)x + \frac{\phi''(0)}{1�2}x^{2} + \etc.
\Tag{(44)}
\]

Now $x^{2} = x$, $x^{3} = x$,~\etc., whence
\[
\phi(x) = \phi(0) + x\bigl\{\phi'(0) + \frac{\phi''(0)}{1�2} + \etc.\bigr\}.
\Tag{(45)}
\]

Now if in~\Eqref{(44)} we make $x = 1$, we have
\[
\phi(1) = \phi(0) + \phi'(0) + \frac{\phi''(0)}{1�2} + \etc.,
\]
whence
\[
\phi'(0) + \frac{\phi''(0)}{1�2} + \frac{\phi'''(0)}{1�2�3} + \etc.
  = \phi(1) - \phi(0).
\]

Substitute this value for the coefficient of~$x$ in the second
member of~\Eqref{(45)}, and we have\footnote
  {Although this and the following theorems have only been proved for those
  forms of functions which are expansible by Maclaurin's theorem, they may be
  regarded as true for all forms whatever; this will appear from the applications.
  The reason seems to be that, as it is only through the one form of expansion
  that elective functions become interpretable, no conflicting interpretation is
  possible.

  The development of~$\phi(x)$ may also be determined thus. By the known formula
  for expansion in factorials,
  \[
  \phi(x) = \phi(0) + \Delta\phi(0)x
    + \frac{\Delta^{2}\phi(0)}{1�2}x(x - 1) + \etc.
  \]
%[** TN: Footnote continues]
  Now $x$~being an elective symbol, $x(x - 1) = 0$, so that all the terms after the
  second, vanish. Also $\Delta\phi(0) = \phi(1) - \phi(0)$, whence
  \[
  \phi\bigl\{x = \phi(0)\bigr\} + \bigl\{\phi(1) - \phi(0)\bigr\}x.
  \]

  The mathematician may be interested in the remark, that this is not the
  only case in which an expansion stops at the second term. The expansions of
  the compound operative functions $\phi\left(\dfrac{d}{dx} + x^{-1}\right)$ and $\phi\left\{x + \left(\dfrac{d}{dx}\right)^{-1}\right\}$ are,
  respectively,
  \[
  \phi\left(\frac{d}{dx}\right) + \phi'\left(\frac{d}{dx}\right)x^{-1},
  \]
  and
  \[
  \phi(x) + \phi'(x)\left(\frac{d}{dx}\right)^{-1}.
  \]

  See \textit{Cambridge Mathematical Journal}, Vol.~\textsc{iv}. p.~219.}
\[
\phi(x) = \phi(0) + \bigl\{\phi(1) - \phi(0)\bigr\}x,
\Tag{(46)}
\]
\PageSep{61}
which we shall also employ under the form
\[
\phi(x) = \phi(1)x + \phi(0)(1 - x).
\Tag{(47)}
\]

Every function of~$x$, in which integer powers of that symbol
are alone involved, is by this theorem reducible to the first
order. The quantities $\phi(0)$,~$\phi(1)$, we shall call the moduli
of the function~$\phi(x)$. They are of great importance in the
theory of elective functions, as will appear from the succeeding
Propositions.

\Prop{1.} Any two functions $\phi(x)$,~$\psi(x)$, are equivalent,
whose corresponding moduli are equal.

This is a plain consequence of the last Proposition. For since
\begin{align*}
\phi(x) &= \phi(0) + \bigl\{\phi(1) - \phi(0)\bigr\}x, \\
\psi(x) &= \psi(0) + \bigl\{\psi(1) - \psi(0)\bigr\}x,
\end{align*}
it is evident that if $\phi(0) = \psi(0)$, $\phi(1) = \psi(1)$, the two
expansions will be equivalent, and therefore the functions which
they represent will be equivalent also.

The converse of this Proposition is equally true, viz.

If two functions are equivalent, their corresponding moduli
are equal.

Among the most important applications of the above theorem,
we may notice the following.

Suppose it required to determine for what forms of the
function~$\phi(x)$, the following equation is satisfied, viz.
\[
\bigl\{\phi(x)\bigr\}^{n} = \phi(x).
\]
\PageSep{62}
Here we at once obtain for the expression of the conditions
in question,
\[
\bigl\{\phi(0)\bigr\}^{n} = \phi(0)\Typo{.}{,}\quad
\bigl\{\phi(1)\bigr\}^{n} = \phi(1).
\Tag{(48)}
\]

Again, suppose it required to determine the conditions under
which the following equation is satisfied, viz.
\[
\phi(x)\psi(x) = \chi(x)\Typo{,}{.}
\]

The general theorem at once gives
\[
\phi(0)\psi(0) = \chi(0)\Typo{.}{,}\quad
\phi(1)\psi(1) = \chi(1).
\Tag{(49)}
\]

This result may also be proved by substituting for~$\phi(x)$,
$\psi(x)$, $\chi(x)$, their expanded forms, and equating the coefficients
of the resulting equation properly reduced.

All the above theorems may be extended to functions of more
than one symbol. For, as different elective symbols combine
with each other according to the same laws as symbols of quantity,
we can first expand a given function with reference to any
particular symbol which it contains, and then expand the result
with reference to any other symbol, and so on in succession, the
order of the expansions being quite indifferent.

Thus the given function being~$\phi(xy)$ we have
\[
\phi(xy) = \phi(x0) + \bigl\{\phi(x1) - \phi(x0)\bigr\}y,
\]
and expanding the coefficients with reference to~$x$, and reducing
\begin{align*}
\phi(xy) = \phi(00)
  &+ \bigl\{\phi(10) - \phi(00)\bigr\}x
   + \bigl\{\phi(01) - \phi(00)\bigr\}y \\
  &+ \bigl\{\phi(11) - \phi(10) - \phi(01) + \phi(00)\bigr\}xy,
\Tag{(50)}
\end{align*}
to which we may give the elegant symmetrical form
\begin{align*}
%[** TN: Not aligned in the original]
\phi(xy) = \phi(00)(1 - x)(1 - y) &+ \phi(01)y(1 - x) \\
  &+ \phi(10)x(1 - y) + \phi(11)xy,
\Tag{(51)}
\end{align*}
wherein we shall, in accordance with the language already
employed, designate $\phi(00)$, $\phi(01)$, $\phi(10)$, $\phi(11)$, as the
moduli of the function~$\phi(xy)$.

By inspection of the above general form, it will appear that
any functions of two variables are equivalent, whose corresponding
moduli are all equal.
\PageSep{63}

Thus the conditions upon which depends the satisfaction of
the equation,
\[
\bigl\{\phi(xy)\bigr\}^{n} = \phi(xy)
\]
are seen to be
\[
\begin{alignedat}{2}
\bigl\{\phi(00)\bigr\}^{n} &= \phi(00),\qquad&
\bigl\{\phi(01)\bigr\}^{n} &= \phi(01), \\
\bigl\{\phi(10)\bigr\}^{n} &= \phi(10), &
\bigl\{\phi(11)\bigr\}^{n} &= \phi(11).
\end{alignedat}
\Tag{(52)}
\]

And the conditions upon which depends the satisfaction of
the equation
\[
\phi(xy)\psi(xy) = \chi(xy),
\]
are
\[
\begin{alignedat}{2}
\phi(00)\psi(00) &= \chi(00),\qquad&
\phi(01)\psi(01) &= \chi(01), \\
\phi(10)\psi(10) &= \chi(10),\qquad&
\phi(11)\psi(11) &= \chi(11).
\end{alignedat}
\Tag{(53)}
\]

It is very easy to assign by induction from \Eqref{(47)}~and~\Eqref{(51)}, the
general form of an expanded elective function. It is evident
that if the number of elective symbols is~$m$, the number of the
moduli will be~$2^{m}$, and that their separate values will be obtained
by interchanging in every possible way the values $1$~and~$0$ in the
places of the elective symbols of the given function. The several
terms of the expansion of which the moduli serve as coefficients,
will then be formed by writing for each~$1$ that recurs under the
functional sign, the elective symbol~$x$,~\etc., which it represents,
and for each~$0$ the corresponding~$1 - x$,~\etc., and regarding these
as factors, the product of which, multiplied by the modulus from
which they are obtained, constitutes a term of the expansion.

Thus, if we represent the moduli of any elective function
$\phi(xy\dots)$ by $a_{1}$,~$a_{2}$, $\dots,~a_{r}$, the function itself, when expanded
and arranged with reference to the moduli, will assume the form
\[
\phi(xy) = a_{1}t_{1} + a_{2}t_{2} \dots + a_{r}t_{r},
\Tag{(54)}
\]
in which $t_{1}t_{2}\dots t_{r}$ are functions of $x$,~$y$,~$\dots$, resolved into factors
of the forms $x$,~$y$,~$\dots$ $1 - x$, $1 - y$,~$\dots$~\etc. These functions satisfy
individually the index relations
\[
t_{1}^{n} = t_{1},\quad
t_{2}^{n} = t_{2},\quad \etc.,
\Tag{(55)}
\]
and the further relations,
\[
t_{1}t_{2} = 0\dots t_{1}t_{2} = 0,~\etc.,
\Tag{(56)}
\]
\PageSep{64}
the product of any two of them vanishing. This will at once
be inferred from inspection of the particular forms \Eqref{(47)}~and~\Eqref{(51)}.
Thus in the latter we have for the values of $t_{1}$,~$t_{2}$,~\etc., the forms
\[
xy,\quad
x(1 - y),\quad
(1 - x)y,\quad
(1 - x)(1 - y);
\]
and it is evident that these satisfy the index relation, and that
their products all vanish. We shall designate $t_{1}t_{2}\dots$ as the constituent
functions of~$\phi(xy)$, and we shall define the peculiarity
of the vanishing of the binary products, by saying that those
functions are \emph{exclusive}. And indeed the classes which they
represent are mutually exclusive.

The sum of all the constituents of an expanded function is
unity. An elegant proof of this Proposition will be obtained
by expanding~$1$ as a function of any proposed elective symbols.
Thus if in~\Eqref{(51)} we assume $\phi(xy) = 1$, we have $\phi(11) = 1$,
$\phi(10) = 1$, $\phi(01) = 1$, $\phi(00) = 1$, and \Eqref{(51)}~gives
\[
1 = xy + x(1 - y) + (1 - x)y + (1 - x)(1 - y).
\Tag{(57)}
\]

It is obvious indeed, that however numerous the symbols
involved, all the moduli of unity are unity, whence the sum
of the constituents is unity.

We are now prepared to enter upon the question of the
general interpretation of elective equations. For this purpose
we shall find the following Propositions of the greatest service.

\Prop{2.} If the first member of the general equation
$\phi(xy\dots) = 0$, be expanded in a series of terms, each of which
is of the form~$at$, $a$~being a modulus of the given function, then
for every numerical modulus~$a$ which does not vanish, we shall
have the equation
\[
at = 0,
\]
and the combined interpretations of these several equations will
express the full significance of the original equation.

For, representing the equation under the form
\[
a_{1}t_{1} + a_{2}t_{2} \dots + a_{r}t_{r}  = 0.
\Tag{(58)}
\]

Multiplying by~$t_{1}$ we have, by~\Eqref{(56)},
\[
a_{1}t_{1} = 0,
\Tag{(59)}
\]
\PageSep{65}
whence if $a_{1}$~is a numerical constant which does not vanish,
\[
t_{1} = 0,
\]
and similarly for all the moduli which do not vanish. And
inasmuch as from these constituent equations we can form the
given equation, their interpretations will together express its
entire significance.

Thus if the given equation were
\[
x - y = 0,\quad \text{$X$s~and~$Y$s are identical,}
\Tag{(60)}
\]
we should have $\phi(11) = 0$, $\phi(10) = 1$, $\phi(01) = -1$, $\phi(00) = 0$,
so that the expansion~\Eqref{(51)} would assume the form
\[
x(1 - y) - y(1 - x) = 0,
\]
whence, by the above theorem,
\begin{alignat*}{2}
x(1 - y) &= 0,\qquad& \text{All~$X$s are~$Y$s,} \\
y(1 - x) &= 0,      & \text{All~$Y$s are~$X$s,}
\end{alignat*}
results which are together equivalent to~\Eqref{(60)}.

It may happen that the simultaneous satisfaction of equations
thus deduced, may require that one or more of the elective
symbols should vanish. This would only imply the nonexistence
of a class: it may even happen that it may lead to a final
result of the form
\[
1 = 0,
\]
which would indicate the nonexistence of the logical Universe.
Such cases will only arise when we attempt to unite contradictory
Propositions in a single equation. The manner in which
the difficulty seems to be evaded in the result is characteristic.

It appears from this Proposition, that the differences in the
interpretation of elective functions depend solely upon the
number and position of the vanishing moduli. No change in
the value of a modulus, but one which causes it to vanish,
produces any change in the interpretation of the equation in
which it is found. If among the infinite number of different
values which we are thus permitted to give to the moduli which
do not vanish in a proposed equation, any one value should be
\PageSep{66}
preferred, it is unity, for when the moduli of a function are all
either $0$~or~$1$, the function itself satisfies the condition
\[
\bigl\{\phi(xy\dots)\bigr\}^{n} = \phi(xy\dots),
\]
and this at once introduces symmetry into our Calculus, and
provides us with fixed standards for reference.

\Prop{3.} If $w = \phi(xy\dots)$, $w$,~$x$,~$y$,~$\dots$ being elective symbols,
and if the second member be completely expanded and arranged
in a series of terms of the form~$at$, we shall be permitted
to equate separately to~$0$ every term in which the modulus~$a$
does not satisfy the condition
\[
a^{n} = a,
\]
and to leave for the value of~$w$ the sum of the remaining terms.

As the nature of the demonstration of this Proposition is
quite unaffected by the number of the terms in the second
member, we will for simplicity confine ourselves to the supposition
of there being four, and suppose that the moduli of the
two first only, satisfy the index law.

We have then
\[
w = a_{1}t_{1} + a_{2}t_{2} + a_{3}t_{3} + a_{4}t_{4},
\Tag{(61)}
\]
with the relations
\[
a_{1}^{n} = a_{1},\quad
a_{2}^{n} = a_{2},
\]
in addition to the two sets of relations connecting $t_{1}$,~$t_{2}$, $t_{3}$,~$t_{4}$,
in accordance with \Eqref{(55)}~and~\Eqref{(56)}.

Squaring~\Eqref{(61)}, we have
\[
w = a_{1}t_{1} + a_{2}t_{2} + a_{3}^{2}t_{3} + a_{4}^{2}t_{4},
\]
and subtracting~\Eqref{(61)} from this,
\[
(a_{3}^{2} - a_{3})t_{3} + (a_{4}^{2} - a_{4})t_{4} = 0;
\]
and it being an hypothesis, that the coefficients of these terms
do not vanish, we have, by \PropRef{2},
\[
t_{3} = 0,\quad
t_{4} = 0,
\Tag{(62)}
\]
whence \Eqref{(61)}~becomes
\[
w = a_{1}t_{1} + a_{2}t_{2}.
\]
The utility of this Proposition will hereafter appear.
\PageSep{67}

\Prop{4.} The functions $t_{1}t_{2}\dots t_{r}$ being mutually exclusive, we
shall always have
\[
\psi(a_{1}t_{1} + a_{2}t_{2} \dots + a_{r}t_{r})
  = \psi(a_{1})t_{1} + \psi(a_{2})t_{2} \dots + \psi(a_{r})t_{r},
\Tag{(63)}
\]
whatever may be the values of $a_{1}a_{2}\dots a_{r}$ or the form of~$\psi$.

%[** TN: Paragraph not indented in the original]
Let the function $a_{1}t_{1} + a_{2}t_{2} \dots + a_{r}t_{r}$ be represented by~$\phi(xy\dots)$,
then the moduli $a_{1}a_{2}\dots a_{r}$ will be given by the expressions
\[
\phi(11\dots),\quad
\phi(10\dots),\quad
(\dots)\ \phi(00\dots).
\]

Also
\begin{align*}
&\phantom{{}={}}\psi(a_{1}t_{1} + a_{2}t_{2} \dots + a_{r}t_{r})
   = \psi\bigl\{\phi(xy\dots)\bigr\} \\
  &= \psi\bigl\{\phi(11\dots)\bigr\}xy\dots
   + \psi\bigl\{\phi(10\dots)\bigr\}x(1 - y)\dots \\
  &\qquad+ \psi\bigl\{\phi(00\dots)\bigr\}(1 - x)(1 - y)\dots \\
  &= \psi(a_{1})xy\dots + \psi(a_{2})x(1 - y)\dots + \psi(a_{r})(1 - x)(1 - y)\dots \\
  &= \psi(a_{1})t_{1} + \psi(a_{2})t_{2}\dots + \psi(a_{r})t_{r}.
\Tag{(64)}
\end{align*}

It would not be difficult to extend the list of interesting
properties, of which the above are examples. But those which
we have noticed are sufficient for our present requirements.
The following Proposition may serve as an illustration of their
utility.

\Prop{5.} Whatever process of reasoning we apply to a single
given Proposition, the result will either be the same Proposition
or a limitation of it.

Let us represent the equation of the given Proposition under
its most general form,
\[
a_{1}t_{1} + a_{2}t_{2} \dots + a_{r}t_{r} = 0,
\Tag{(65)}
\]
resolvable into as many equations of the form $t = 0$ as there are
moduli which do not vanish.

Now the most general transformation of this equation is
\[
\psi(a_{1}t_{1} + a_{2}t_{2} \dots + a_{r}t_{r}) = \psi(0),
\Tag{(66)}
\]
provided that we attribute to~$\psi$ a perfectly arbitrary character,
allowing it even to involve new elective symbols, having \emph{any
proposed relation} to the original ones.
\PageSep{68}

The development of~\Eqref{(66)} gives, by the last Proposition,
\[
\psi(a_{1})t_{1} + \psi(a_{2})t_{2}\dots + \psi(a_{r})t_{r} = \psi(0).
\]
To reduce this to the general form of reference, it is only necessary
to observe that since
\[
t_{1} + t_{2} \dots + t_{r} = 1,
\]
we may write for~$\psi(0)$,
\[
\psi(0)(t_{1} + t_{2} \dots + t_{r}),
\]
whence, on substitution and transposition,
\[
\bigl\{\psi(a_{1}) - \psi(0)\bigr\}t_{1} +
\bigl\{\psi(a_{2}) - \psi(0)\bigr\}t_{2} \dots +
\bigl\{\psi(a_{r}) - \psi(0)\bigr\}t_{r} = 0.
\]

From which it appears, that if $a$~be any modulus of the
original equation, the corresponding modulus of the transformed
equation will be
\[
\psi(a) - \psi(0).
\]

If $a = 0$, then $\psi(a) - \psi(0) = \psi(0) - \psi(0) = 0$, whence
there are no \emph{new terms} in the transformed equation, and therefore
there are no \emph{new Propositions} given by equating its constituent
members to~$0$.

Again, since $\psi(a) - \psi(0)$ may vanish without $a$~vanishing,
terms may be wanting in the transformed equation which existed
in the primitive. Thus some of the constituent truths of the
original Proposition may entirely disappear from the interpretation
of the final result.

Lastly, if $\psi(a) - \psi(0)$ do not vanish, it must either be
a numerical constant, or it must involve new elective symbols.
In the former case, the term in which it is found will give
\[
t = 0,
\]
which is one of the constituents of the original equation: in the
latter case we shall have
\[
\bigl\{\psi(a\Typo{}{)} - \psi(0)\bigr\}t = 0,
\]
in which $t$~has a limiting factor. The interpretation of this
equation, therefore, is a limitation of the interpretation of~\Eqref{(65)}.
\PageSep{69}

The purport of the last investigation will be more apparent
to the mathematician than to the logician. As from any mathematical
equation an infinite number of others may be deduced,
it seemed to be necessary to shew that when the original
equation expresses a logical Proposition, every member of the
derived series, even when obtained by expansion under a functional
sign, admits of exact and consistent interpretation.
\PageSep{70}


\Chapter{Of the Solution of Elective Equations.}

\First{In} whatever way an elective symbol, considered as unknown,
may be involved in a proposed equation, it is possible to assign
its complete value in terms of the remaining elective symbols
considered as known. It is to be observed of such equations,
that from the very nature of elective symbols, they are necessarily
linear, and that their solutions have a very close analogy
with those of linear differential equations, arbitrary elective
symbols in the one, occupying the place of arbitrary constants
in the other. The method of solution we shall in the first place
illustrate by particular examples, and, afterwards, apply to the
investigation of general theorems.

Given $(1 - x)y = 0$, (All~$Y$s are~$X$s), to determine~$y$ in
terms of~$x$.

As $y$~is a function of~$x$, we may assume $y = vx + v'(1 - x)$,
(such being the expression of an arbitrary function of~$x$), the
moduli $v$~and~$v'$ remaining to be determined. We have then
\[
(1 - x)\bigl\{vx + v'(1 - x)\bigr\} = 0,
\]
or, on actual multiplication,
\[
v'(1 - x) = 0\Typo{:}{;}
\]
that this may be generally true, without imposing any restriction
upon~$x$, we must assume $v' = 0$, and there being no condition to
limit~$v$, we have
\[
y = vx.
\Tag{(67)}
\]

This is the complete solution of the equation. The condition
that $y$~is an elective symbol requires that $v$~should be an elective
\PageSep{71}
symbol also (since it must satisfy the index law), its interpretation
in other respects being arbitrary.

Similarly the solution of the equation, $xy = 0$, is
\[
y = v(1 - x).
\Tag{(68)}
\]

Given $(1 - x)zy = 0$, (All~$Y$s which are~$Z$s are~$X$s), to determine~$y$.

As $y$~is a function of $x$~and~$z$, we may assume
\[
y = v(1 - x) (1 - z) + v'(1 - x)z + v''x(1 - z) + v'''zx.
\]
And substituting, we get
\[
v'(1 - x)z = 0,
\]
whence $v' = 0$. The complete solution is therefore
\[
y = v(1 - x)(1 - z) + v''x(1 - z) + v'''xz,
\Tag{(69)}
\]
$v'$,~$v''$,~$v'''$, being arbitrary elective symbols, and the rigorous
interpretation of this result is, that Every~$Y$ is \emph{either} a not-$X$
and not-$Z$, or an~$X$ and not-$Z$, or an~$X$ and~$Z$.

It is deserving of note that the above equation may, in consequence
of its linear form, be solved by adding the two
particular solutions with reference to $x$~and~$z$; and replacing
the arbitrary constants which each involves by an arbitrary
function of the other symbol, the result is
\[
y = x\phi(z) + (1 - z)\psi(x).
\Tag{(70)}
\]

To shew that this solution is equivalent to the other, it is
only necessary to substitute for the arbitrary functions $\phi(z)$,
$\psi(x)$, their equivalents
\[
wz + w'(1 - z)\quad\text{and}\quad w''x + w'''(1 - x),
\]
we get
\[
y = wxz + (w + w'')x(1 - z) + w'''(1 - x)(1 - z).
\]

In consequence of the perfectly arbitrary character of $w'$~and~$w''$,
we may replace their sum by a single symbol~$w$, whence
\[
y = wxz + w'x(1 - z) + w'''(1 - x)(1 - z),
\]
which agrees with~\Eqref{(69)}.
\PageSep{72}

The solution of the equation $wx(1 - y)z = 0$, expressed by
arbitrary functions, is
\[
z = (1 - w) \phi(xy) + (1 - x)\psi(wy) + y\chi(wx).
\Tag{(71)}
\]

These instances may serve to shew the analogy which exists
between the solutions of elective equations and those of the
corresponding order of linear differential equations. Thus the
expression of the integral of a partial differential equation,
either by arbitrary functions or by a series with arbitrary coefficients,
is in strict analogy with the case presented in the two
last examples. To pursue this comparison further would minister
to curiosity rather than to utility. We shall prefer to contemplate
the problem of the solution of elective equations under
its most general aspect, which is the object of the succeeding
investigations.

To solve the general equation $\phi(xy) = 0$, with reference to~$y$.

If we expand the given equation with reference to $x$~and~$y$,
we have
\[
%[** TN: Equation broken across two lines in the original
\phi(00)(1 - x)(1 - y) + \phi(01)(1 - x)y + \phi(10)x(1 - y)
  + \phi(11)xy = 0,
\Tag{(72)}
\]
the coefficients $\phi(00)$~\etc.\ being numerical constants.

Now the general expression of~$y$, as a function of~$x$, is
\[
y = vx + v'(1 - x),
\]
$v$~and~$v'$ being unknown symbols to be determined. Substituting
this value in~\Eqref{(72)}, we obtain a result which may be written in
the following form,
\[
%[** TN: Equation broken across two lines in the original
\bigl[\phi(10) + \bigl\{\phi(11) - \phi(10)\bigr\}v\bigr]x
  + \bigl[\phi(00) + \bigl\{\phi(00) - \phi(00)\bigr\} v'\bigr](1 - x) = 0;
\]
and in order that this equation may be satisfied without any
way restricting the generality of~$x$, we must have
\begin{alignat*}{2}
\phi(10) &+ \bigl\{\phi(11) - \phi(10)\bigr\}v  &&= 0, \\
\phi(00) &+ \bigl\{\phi(01) - \phi(00)\bigr\}v' &&= 0,
\end{alignat*}
\PageSep{73}
from which we deduce
\[
v = \frac{\phi(10)}{\phi(10) - \phi(11)}\;,\qquad
v' = \frac{\phi(00)}{\phi(01) - \phi(00)}\;,
\]
wherefore
\[
y = \frac{\phi(10)}{\phi(10) - \phi(11)}\, x
  + \frac{\phi(00)}{\phi(00) - \phi(01)}\, (1 - x).
\Tag{(73)}
\]

Had we expanded the original equation with respect to $y$~only,
we should have had
\[
\phi(x0) + \bigl\{\phi(x1) - \phi(x0)\bigr\}y = 0;
\]
but it might have startled those who are unaccustomed to the
processes of Symbolical Algebra, had we from this equation
deduced
\[
y = \frac{\phi(x0)}{\phi(x0) - \phi(x1)}\;,
\]
because of the apparently meaningless character of the second
member. Such a result would however have been perfectly
lawful, and the expansion of the second member would have
given us the solution above obtained. I shall in the following
example employ this method, and shall only remark that those
to whom it may appear doubtful, may verify its conclusions by
the previous method.

To solve the general equation $\phi(xyz) = 0$, or in other words
to determine the value of~$z$ as a function of $x$~and~$y$.

Expanding the given equation with reference to~$z$, we have
\begin{gather*}
\phi(xy0) + \bigl\{\phi(xy1) - \phi(xy0)\bigr\}\Chg{�}{}z = 0; \\
\therefore z = \frac{\phi(xy0)}{\phi(xy0) - \phi(xy1)}\;,
\Tag{(74)}
\end{gather*}
and expanding the second member as a function of $x$~and~$y$ by
aid of the general theorem, we have
\begin{multline*}
z = \frac{\phi(110)}{\phi(110) - \phi(111)}\, xy
  + \frac{\phi(100)}{\phi(100) - \phi(101)}\, x(1 - y) \\
  + \frac{\phi(010)}{\phi(010) - \phi(011)}\, (1 - x)y
  + \frac{\phi(000)}{\phi(000) - \phi(001)}\, (1 - x)(1 - y),
\Tag{(75)}
\end{multline*}
\PageSep{74}
and this is the complete solution required. By the same
method we may resolve an equation involving any proposed
number of elective symbols.

In the interpretation of any general solution of this nature,
the following cases may present themselves.

The values of the moduli $\phi(00)$, $\phi(01)$,~\etc.\ being constant,
one or more of the coefficients of the solution may assume
the form $\frac{0}{0}$~or~$\frac{1}{0}$. In the former case, the indefinite symbol~$\frac{0}{0}$
must be replaced by an arbitrary elective symbol~$v$. In the
latter case, the term, which is multiplied by a factor~$\frac{1}{0}$ (or by
any numerical constant except~$1$), must be separately equated
to~$0$, and will indicate the existence of a subsidiary Proposition.
This is evident from~\Eqref{(62)}.

Ex. Given $x(1 - y) = 0$, All~$X$s are~$Y$s, to determine~$y$ as
a function of~$x$.

Let $\phi(xy) = x(1 - y)$, then $\phi(10) = 1$, $\phi(11) = 0$, $\phi(01) = 0$,
$\phi(00) = 0$; whence, by~\Eqref{(73)},
\begin{align*}
y &= \frac{1}{1 - 0}\, x + \frac{0}{0 - 0}\, (1 - x) \\
  &= x + \tfrac{0}{0}(1 - x) \\
  &= x + v(1 - x),
\Tag{(76)}
\end{align*}
$v$~being an arbitrary elective symbol. The interpretation of this
result is that the class~$Y$ consists of the entire class~$X$ with an
indefinite remainder of not-$X$s. This remainder is indefinite in
the highest sense, \ie~it may vary from~$0$ up to the entire class
of not-$X$s.

Ex. Given $x(1 - z) + z = y$, (the class~$Y$ consists of the
entire class~$Z$, with such not-$Z$s as are~$X$s), to find~$Z$.

Here $\phi(xyz) = x(1 - z) - y + z$, whence we have the following
set of values for the moduli,
\begin{alignat*}{4}
\phi(110) &= 0,\quad& \phi(111) &= 0,\quad& \phi(100) &= 1,\quad& \phi(101) &= 1, \\
\phi(010) &=-1,\quad& \phi(011) &= 0,\quad& \phi(000) &= 0,\quad& \phi(001) &= 1,
\end{alignat*}
and substituting these in the general formula~\Eqref{(75)}, we have
\[
z = \tfrac{0}{0}xy + \tfrac{1}{0}x(1 - y) + (1 - x)y,
\Tag{(77)}
\]
\PageSep{75}
the infinite coefficient of the second term indicates the equation
\[
x(1 - y) = 0,\quad\text{All~$X$s are~$Y$s;}
\]
and the indeterminate coefficient of the first term being replaced
by~$v$, an arbitrary elective symbol, we have
\[
z = (1 - x)y + vxy,
\]
the interpretation of which is, that the class~$Z$ consists of all the~$Y$s
which are not~$X$s, and an \emph{indefinite} remainder of~$Y$s which
are~$X$s. Of course this indefinite remainder may vanish. The
two results we have obtained are logical inferences (not very
obvious ones) from the original Propositions, and they give us
all the information which it contains respecting the class~$Z$, and
its constituent elements.

Ex. Given $x = y(1 - z) + z(1 - y)$. The class~$X$ consists of
all~$Y$s which are not-$Z$s, and all~$Z$s which are not-$Y$s: required
the class~$Z$.

We have
\begin{alignat*}{4}
\phi(xyz) &= \rlap{$x - y(1 - z) - z(1 - y)$,} \\
\phi(110) &= 0,\quad& \phi(111) &= 1,\quad&
\phi(100) &= 1,\quad& \phi(101) &= 0, \\
%
\phi(010) &= -1,\quad& \phi(011) &= 0, &
\phi(000) &=  0,     & \phi(001) &= -1;
\end{alignat*}
whence, by substituting in~\Eqref{(75)},
\[
z = x(1 - y) + y(1 - x),
\Tag{(78)}
\]
the interpretation of which is, the class~$Z$ consists of all~$X$s
which are not~$Y$s, and of all~$Y$s which are not~$X$s; an inference
strictly logical.

Ex. Given $y\bigl\{1 - z(1 - x)\bigr\} = 0$, All~$Y$s are~$Z$s and not-$X$s.

Proceeding as before to form the moduli, we have, on substitution
in the general formul�,
\[
z = \tfrac{1}{0}xy
  + \tfrac{0}{0}x(1 - y)
  + y(1 - x)
  + \tfrac{0}{0}(1 - x)(1 - y),
\]
or
\begin{align*}
%[** TN: Unaligned in the original]
z &= y(1 - x) + vx(1 - y) + v'(1 - x)(1 - y) \\
  &= y(1 - x) + (1 - y)\phi(x),
\Tag{(79)}
\end{align*}
with the relation
\[
xy = 0\Typo{:}{;}
\]
from these it appears that No~$Y$s are~$X$s, and that the class~$Z$
\PageSep{76}
consists of all~$Y$s which are not~$X$s, and of an indefinite remainder
of not-$Y$s.

This method, in combination with Lagrange's method of
indeterminate multipliers, may be very elegantly applied to the
treatment of simultaneous equations. Our limits only permit us
to offer a single example, but the subject is well deserving of
further investigation.

Given the equations $x(1 - z) = 0$, $z(1 - y) = 0$, All~$X$s are~$Z$s,
All~$Z$s are~$Y$s, to determine the complete value of~$z$ with
any subsidiary relations connecting $x$~and~$y$.

Adding the second equation multiplied by an indeterminate
constant~$\lambda$, to the first, we have
\[
x(1 - z) + \lambda z(1 - y) = 0,
\]
whence determining the moduli, and substituting in~\Eqref{(75)},
\[
z = xy + \frac{1}{1 - \lambda}\, x(1 - y) + \tfrac{0}{0}(1 - x)y,
\Tag{(80)}
\]
from which we derive
\[
z = xy + v(1 - x)y,
\]
with the subsidiary relation
\[
x(1 - y) = 0\Typo{:}{;}
\]
the former of these expresses that the class~$Z$ consists of all~$X$s
that are~$Y$s, with an indefinite remainder of not-$X$s that are~$Y$s;
the latter, that All~$X$s are~$Y$s, being in fact the conclusion
of the syllogism of which the two given Propositions are the
premises.

By assigning an appropriate meaning to our symbols, all the
equations we have discussed would admit of interpretation in
hypothetical, but it may suffice to have considered them as
examples of categoricals.

That peculiarity of elective symbols, in virtue of which every
elective equation is reducible to a system of equations $t_{1} = 0$,
$t_{2} = 0$,~\etc., so constituted, that all the binary products $t_{1}t_{2}$, $t_{1}t_{3}$,
\etc., vanish, represents a general doctrine in Logic with reference
to the ultimate analysis of Propositions, of which it
may be desirable to offer some illustration.
\PageSep{77}

Any of these constituents $t_{1}$,~$t_{2}$,~\etc.\ consists only of factors
of the forms $x$,~$y$,~$\dots$ $1 - w$,~$1 - z$,~\etc. In categoricals it therefore
represents a compound class, \ie~a class defined by the
presence of certain qualities, and by the absence of certain
other qualities.

Each constituent equation $t_{1} = 0$,~\etc.\ expresses a denial of the
existence of some class so defined, and the different classes are
mutually exclusive.

\begin{Rule}[]
Thus all categorical Propositions are resolvable into a denial of
the existence of certain compound classes, no member of one such
class being a member of another.
\end{Rule}

The Proposition, All~$X$s are~$Y$s, expressed by the equation
$x(1 - y) = 0$, is resolved into a denial of the existence of a
class whose members are~$X$s and not-$Y$s.

The Proposition Some~$X$s are~$Y$s, expressed by $v = xy$, is
resolvable as follows. On expansion,
\begin{gather*}
v - xy = vx(1 - y) + vy(1 - x) + v(1 - x)(1 - y) - xy(1 - v); \\
\therefore
vx(1 - y) = 0,\quad
vy(1 - x) = 0,\quad
v(1 - x)(1 - y) = 0,\quad
(1 - v)xy = 0.
\end{gather*}

The three first imply that there is no class whose members
belong to a certain unknown Some, and are~1st, $X$s~and not~$Y$s;
2nd, $Y$s~and not~$X$s; 3rd, not-$X$s and not-$Y$s. The fourth
implies that there is no class whose members are $X$s~and~$Y$s
without belonging to this unknown Some.

From the same analysis it appears that \begin{Rule}[]all hypothetical Propositions
may be resolved into denials of the coexistence of the truth
or falsity of certain assertions.
\end{Rule}

Thus the Proposition, If $X$~is true, $Y$~is true, is resolvable
by its equation $x(1 - y) = 0$, into a denial that the truth of~$X$
and the falsity of~$Y$ coexist.

And the Proposition Either $X$~is true, or $Y$~is true, members
exclusive, is resolvable into a denial, first, that $X$~and~$Y$ are
both true; secondly, that $X$~and~$Y$ are both false.

But it may be asked, is not something more than a system of
negations necessary to the constitution of an affirmative Proposition?
is not a positive element required? Undoubtedly
\PageSep{78}
there is need of one; and this positive element is supplied
in categoricals by the assumption (which may be regarded as
a prerequisite of reasoning in such cases) that there \emph{is} a Universe
of conceptions, and that each individual it contains either
belongs to a proposed class or does not belong to it; in hypotheticals,
by the assumption (equally prerequisite) that there
is a Universe of conceivable cases, and that any given Proposition
is either true or false. Indeed the question of the
existence of conceptions (\textgreek{e>i >'esti}) is preliminary to any statement
%[** TN: Should be \textgreek{t'i >esti}? Not sufficiently certain to change.]
of their qualities or relations (\textgreek{t'i >'esti}).---\textit{Aristotle, Anal.\ Post.}\
lib.~\textsc{ii}.\ cap.~2.

It would appear from the above, that Propositions may be
regarded as resting at once upon a positive and upon a negative
foundation. Nor is such a view either foreign to the spirit
of Deductive Reasoning or inappropriate to its Method; the
latter ever proceeding by limitations, while the former contemplates
the particular as derived from the general.


%[** TN: Equation numbering restarts]
\Section{Demonstration of the Method of Indeterminate Multipliers, as
applied to Simultaneous Elective Equations.}

To avoid needless complexity, it will be sufficient to consider
the case of three equations involving three elective symbols,
those equations being the most general of the kind. It will
be seen that the case is marked by every feature affecting
the character of the demonstration, which would present itself
in the discussion of the more general problem in which the
number of equations and the number of variables are both
unlimited.

Let the given equations be
\[
\phi(xyz) = 0,\quad
\psi(xyz) = 0,\quad
\chi(xyz) = 0.
\Tag[app]{(1)}
\]

Multiplying the second and third of these by the arbitrary
constants $h$~and~$k$, and adding to the first, we have
\[
\phi(xyz) + h\psi(xyz) + k\chi(xyz) = 0;
\Tag[app]{(2)}
\]
\PageSep{79}
and we are to shew, that in solving this equation with reference
to any variable~$z$ by the general theorem~\Eqref{(75)}, we shall obtain
not only the general value of~$z$ independent of $h$~and~$k$, but
also any subsidiary relations which may exist between $x$~and~$y$
independently of~$z$.

%[xref]
If we represent the general equation~\Eqref[app]{(2)} under the form
$F(xyz) = 0$, its solution may by~\Eqref{(75)} be written in the form
\[
z = \frac{xy}{1 - \dfrac{F(111)}{F(110)}}
  + \frac{x(1 - y)}{1 - \dfrac{F(101)}{F(100)}}
  + \frac{y(1 - x)}{1 - \dfrac{F(011)}{F(010)}}
  + \frac{(1 - x)(1 - y)}{1 - \dfrac{F(001)}{F(000)}};
\]
and we have seen, that any one of these four terms is to be
equated to~$0$, whose modulus, which we may represent by~$M$,
does not satisfy the condition $M^{n} = M$, or, which is here the
same thing, whose modulus has any other value than $0$~or~$1$.

Consider the modulus (suppose~$M_{1}$) of the first term, viz.
$\dfrac{1}{1 - \dfrac{F(111)}{F(110)}}$, and giving to the symbol~$F$ its full meaning,
we have
\[
M_{1} = \frac{1}{1 - \dfrac{\phi(111) + h\psi(111) + k\chi(111)}
                           {\phi(110) + h\psi(110) + k\chi(110)}}.
\]

It is evident that the condition $M_{1}^{n} = M_{1}$ cannot be satisfied
unless the right-hand member be independent of $h$~and~$k$; and
in order that this may be the case, we must have the function
$\dfrac{\phi(111) + h\psi(111) + k\chi(111)}
       {\phi(110) + h\psi(110) + k\chi(110)}$ independent of $h$~and~$k$.

Assume then
\[
\frac{\phi(111) + h\psi(111) + k\chi(111)}
     {\phi(110) + h\psi(110) + k\chi(110)} = c,
\]
$c$~being independent of $h$~and~$k$; we have, on clearing of fractions
and equating coefficients,
\[
\phi(111) = c\phi(110),\quad
\psi(111) = c\psi(110),\quad
\chi(111) = c\chi(110);
\]
whence, eliminating~$c$,
\[
\frac{\phi(111)}{\phi(110)}
  = \frac{\psi(111)}{\psi(110)}
  = \frac{\chi(111)}{\chi(110)},
\]
\PageSep{80}
being equivalent to the triple system
\[
\left.\begin{alignedat}{3}
&\phi(111)\psi(110) &&- \phi(110)\psi(111) &&= 0\Add{,} \\
&\psi(111)\chi(110) &&- \psi(110)\chi(111) &&= 0\Add{,} \\
&\chi(111)\phi(110) &&- \chi(110)\Typo{\psi}{\phi}(111) &&= 0\Add{;}
\end{alignedat}
\right\}
\Tag[app]{(3)}
\]
and it appears that if any one of these equations is not satisfied,
the modulus~$M_{1}$ will not satisfy the condition $M_{1}^{n} = M_{1}$, whence
the first term of the value of~$z$ must be equated to~$0$, and
we shall have
\[
xy = 0,
\]
a relation between $x$~and~$y$ independent of~$z$.

Now if we expand in terms of~$z$ each pair of the primitive
equations~\Eqref[app]{(1)}, we shall have
\begin{alignat*}{3}
&\phi(xy0) &&+ \bigl\{\phi(xy1) - \phi(xy0)\bigr\}z &&= 0, \\
&\psi(xy0) &&+ \bigl\{\psi(xy1) - \psi(xy0)\bigr\}z &&= 0, \\
&\chi(xy0) &&+ \bigl\{\chi(xy1) - \chi(xy0)\bigr\}z &&= 0,
\end{alignat*}
and successively eliminating~$z$ between each pair of these equations,
we have
\begin{alignat*}{3}
&\phi(xy1)\psi(xy0) &&- \phi(xy0)\psi(xy1) &&= 0, \\
&\psi(xy1)\chi(xy0) &&- \psi(xy0)\chi(xy1) &&= 0, \\
&\chi(xy1)\phi(xy0) &&- \chi(xy0)\phi(xy1) &&= 0,
\end{alignat*}
which express all the relations between $x$~and~$y$ that are formed
by the elimination of~$z$. Expanding these, and writing in full
the first term, we have
\begin{alignat*}{3}
&\bigl\{\phi(111)\psi(110) &&- \phi(110)\psi(111)\bigr\}xy &&+ \etc. = 0, \\
&\bigl\{\psi(111)\chi(110) &&- \psi(110)\chi(111)\bigr\}xy &&+ \etc. = 0, \\
&\bigl\{\chi(111)\phi(110) &&- \chi(110)\phi(111)\bigr\}xy &&+ \etc. = 0\Typo{:}{;}
\end{alignat*}
and it appears from \PropRef{2}.\ that if the coefficient of~$xy$ in any
of these equations does not vanish, we shall have the equation
\[
xy = 0;
\]
but the coefficients in question are the same as the first members
of the system~\Eqref[app]{(3)}, and the two sets of conditions exactly agree.
Thus, as respects the first term of the expansion, the method of
indeterminate coefficients leads to the same result as ordinary
elimination; and it is obvious that from their similarity of form,
the same reasoning will apply to all the other terms.
\PageSep{81}

Suppose, in the second place, that the conditions~\Eqref[app]{(3)} are satisfied
so that $M_{1}$~is independent of $h$~and~$k$. It will then indifferently
assume the equivalent forms
\[
M_{1} = \frac{1}{1 - \dfrac{\phi(111)}{\phi(110)}}
      = \frac{1}{1 - \dfrac{\psi(111)}{\psi(110)}}
      = \frac{1}{1 - \dfrac{\chi(111)}{\chi(110)}}\Add{.}
\]

These are the exact forms of the first modulus in the expanded
values of~$z$, deduced from the solution of the three
primitive equations singly. If this common value of~$M_{1}$ is $1$
or $\frac{0}{0} = v$, the term will be retained in~$z$; if any other constant
value (except~$0$), we have a relation $xy = 0$, not given by elimination,
but deducible from the primitive equations singly, and
similarly for all the other terms. Thus in every case the expression
of the subsidiary relations is a necessary accompaniment
of the process of solution.

It is evident, upon consideration, that a similar proof will
apply to the discussion of a system indefinite as to the number
both of its symbols and of its equations.
%[** TN: No page break in the original]


\Chapter{Postscript.}

\First{Some} additional explanations and references which have
occurred to me during the printing of this work are subjoined.

The remarks on the connexion between Logic and Language,
\Pageref{5}, are scarcely sufficiently explicit. Both the one and the
other I hold to depend very materially upon our ability to form
general notions by the faculty of abstraction. Language is an
instrument of Logic, but not an indispensable instrument.

To the remarks on Cause, \Pageref{12}, I desire to add the following:
Considering Cause as an invariable antecedent in Nature, (which
is Brown's view), whether associated or not with the idea of
Power, as suggested by Sir~John Herschel, the knowledge of its
existence is a knowledge which is properly expressed by the word
\emph{that} (\textgreek{t`o <ot`i}), not by \emph{why} (\textgreek{t`o di<ot`i}). It is very remarkable that
the two greatest authorities in Logic, modern and ancient, agreeing
in the latter interpretation, differ most widely in its application
to Mathematics. Sir W.~Hamilton says that Mathematics
\PageSep{82}
exhibit only the \emph{that} (\textgreek{t`o <ot`i}): Aristotle says, The \emph{why} belongs
to mathematicians, for they have the demonstrations of Causes.
\textit{Anal.\ Post.}\ lib.~\textsc{i}., cap.~\textsc{xiv}. It must be added that Aristotle's
view is consistent with the sense (albeit an erroneous one)
which in various parts of his writings he virtually assigns to the
word Cause, viz.\ an antecedent in Logic, a sense according to
which the premises might be said to be the cause of the conclusion.
This view appears to me to give even to his physical
inquiries much of their peculiar character.

Upon reconsideration, I think that the view on \Pageref{41}, as to the
presence or absence of a medium of comparison, would readily
follow from Professor De~Morgan's doctrine, and I therefore
relinquish all claim to a discovery. The mode in which it
appears in this treatise is, however, remarkable.

I have seen reason to change the opinion expressed in
\Pagerefs{42}{43}. The system of equations there given for the expression
of Propositions in Syllogism is \emph{always} preferable to the one
before employed---first, in generality---secondly, in facility of
interpretation.

In virtue of the principle, that a Proposition is either true or
false, every elective symbol employed in the expression of
hypotheticals admits only of the values $0$~and~$1$, which are the
only quantitative forms of an elective symbol. It is in fact
possible, setting out from the theory of Probabilities (which is
purely quantitative), to arrive at a system of methods and processes
for the treatment of hypotheticals exactly similar to those
which have been given. The two systems of elective symbols
and of quantity osculate, if I may use the expression, in the
points $0$~and~$1$. It seems to me to be implied by this, that
unconditional truth (categoricals) and probable truth meet together
in the constitution of contingent truth\Typo{;}{} (hypotheticals).
The general doctrine of elective symbols and all the more characteristic
applications are quite independent of any quantitative
origin.
\vfil
\begin{center}
\small
THE END.
\end{center}
\vfil\vfil
%%%%%%%%%%%%%%%%%%%%%%%%% GUTENBERG LICENSE %%%%%%%%%%%%%%%%%%%%%%%%%%
\PGLicense
\begin{PGtext}
End of Project Gutenberg's The Mathematical Analysis of Logic, by George Boole

*** END OF THIS PROJECT GUTENBERG EBOOK THE MATHEMATICAL ANALYSIS OF LOGIC ***

***** This file should be named 36884-pdf.pdf or 36884-pdf.zip *****
This and all associated files of various formats will be found in:
        http://www.gutenberg.org/3/6/8/8/36884/

Produced by Andrew D. Hwang

Updated editions will replace the previous one--the old editions
will be renamed.

Creating the works from public domain print editions means that no
one owns a United States copyright in these works, so the Foundation
(and you!) can copy and distribute it in the United States without
permission and without paying copyright royalties.  Special rules,
set forth in the General Terms of Use part of this license, apply to
copying and distributing Project Gutenberg-tm electronic works to
protect the PROJECT GUTENBERG-tm concept and trademark.  Project
Gutenberg is a registered trademark, and may not be used if you
charge for the eBooks, unless you receive specific permission.  If you
do not charge anything for copies of this eBook, complying with the
rules is very easy.  You may use this eBook for nearly any purpose
such as creation of derivative works, reports, performances and
research.  They may be modified and printed and given away--you may do
practically ANYTHING with public domain eBooks.  Redistribution is
subject to the trademark license, especially commercial
redistribution.



*** START: FULL LICENSE ***

THE FULL PROJECT GUTENBERG LICENSE
PLEASE READ THIS BEFORE YOU DISTRIBUTE OR USE THIS WORK

To protect the Project Gutenberg-tm mission of promoting the free
distribution of electronic works, by using or distributing this work
(or any other work associated in any way with the phrase "Project
Gutenberg"), you agree to comply with all the terms of the Full Project
Gutenberg-tm License (available with this file or online at
http://gutenberg.org/license).


Section 1.  General Terms of Use and Redistributing Project Gutenberg-tm
electronic works

1.A.  By reading or using any part of this Project Gutenberg-tm
electronic work, you indicate that you have read, understand, agree to
and accept all the terms of this license and intellectual property
(trademark/copyright) agreement.  If you do not agree to abide by all
the terms of this agreement, you must cease using and return or destroy
all copies of Project Gutenberg-tm electronic works in your possession.
If you paid a fee for obtaining a copy of or access to a Project
Gutenberg-tm electronic work and you do not agree to be bound by the
terms of this agreement, you may obtain a refund from the person or
entity to whom you paid the fee as set forth in paragraph 1.E.8.

1.B.  "Project Gutenberg" is a registered trademark.  It may only be
used on or associated in any way with an electronic work by people who
agree to be bound by the terms of this agreement.  There are a few
things that you can do with most Project Gutenberg-tm electronic works
even without complying with the full terms of this agreement.  See
paragraph 1.C below.  There are a lot of things you can do with Project
Gutenberg-tm electronic works if you follow the terms of this agreement
and help preserve free future access to Project Gutenberg-tm electronic
works.  See paragraph 1.E below.

1.C.  The Project Gutenberg Literary Archive Foundation ("the Foundation"
or PGLAF), owns a compilation copyright in the collection of Project
Gutenberg-tm electronic works.  Nearly all the individual works in the
collection are in the public domain in the United States.  If an
individual work is in the public domain in the United States and you are
located in the United States, we do not claim a right to prevent you from
copying, distributing, performing, displaying or creating derivative
works based on the work as long as all references to Project Gutenberg
are removed.  Of course, we hope that you will support the Project
Gutenberg-tm mission of promoting free access to electronic works by
freely sharing Project Gutenberg-tm works in compliance with the terms of
this agreement for keeping the Project Gutenberg-tm name associated with
the work.  You can easily comply with the terms of this agreement by
keeping this work in the same format with its attached full Project
Gutenberg-tm License when you share it without charge with others.

1.D.  The copyright laws of the place where you are located also govern
what you can do with this work.  Copyright laws in most countries are in
a constant state of change.  If you are outside the United States, check
the laws of your country in addition to the terms of this agreement
before downloading, copying, displaying, performing, distributing or
creating derivative works based on this work or any other Project
Gutenberg-tm work.  The Foundation makes no representations concerning
the copyright status of any work in any country outside the United
States.

1.E.  Unless you have removed all references to Project Gutenberg:

1.E.1.  The following sentence, with active links to, or other immediate
access to, the full Project Gutenberg-tm License must appear prominently
whenever any copy of a Project Gutenberg-tm work (any work on which the
phrase "Project Gutenberg" appears, or with which the phrase "Project
Gutenberg" is associated) is accessed, displayed, performed, viewed,
copied or distributed:

This eBook is for the use of anyone anywhere at no cost and with
almost no restrictions whatsoever.  You may copy it, give it away or
re-use it under the terms of the Project Gutenberg License included
with this eBook or online at www.gutenberg.org

1.E.2.  If an individual Project Gutenberg-tm electronic work is derived
from the public domain (does not contain a notice indicating that it is
posted with permission of the copyright holder), the work can be copied
and distributed to anyone in the United States without paying any fees
or charges.  If you are redistributing or providing access to a work
with the phrase "Project Gutenberg" associated with or appearing on the
work, you must comply either with the requirements of paragraphs 1.E.1
through 1.E.7 or obtain permission for the use of the work and the
Project Gutenberg-tm trademark as set forth in paragraphs 1.E.8 or
1.E.9.

1.E.3.  If an individual Project Gutenberg-tm electronic work is posted
with the permission of the copyright holder, your use and distribution
must comply with both paragraphs 1.E.1 through 1.E.7 and any additional
terms imposed by the copyright holder.  Additional terms will be linked
to the Project Gutenberg-tm License for all works posted with the
permission of the copyright holder found at the beginning of this work.

1.E.4.  Do not unlink or detach or remove the full Project Gutenberg-tm
License terms from this work, or any files containing a part of this
work or any other work associated with Project Gutenberg-tm.

1.E.5.  Do not copy, display, perform, distribute or redistribute this
electronic work, or any part of this electronic work, without
prominently displaying the sentence set forth in paragraph 1.E.1 with
active links or immediate access to the full terms of the Project
Gutenberg-tm License.

1.E.6.  You may convert to and distribute this work in any binary,
compressed, marked up, nonproprietary or proprietary form, including any
word processing or hypertext form.  However, if you provide access to or
distribute copies of a Project Gutenberg-tm work in a format other than
"Plain Vanilla ASCII" or other format used in the official version
posted on the official Project Gutenberg-tm web site (www.gutenberg.org),
you must, at no additional cost, fee or expense to the user, provide a
copy, a means of exporting a copy, or a means of obtaining a copy upon
request, of the work in its original "Plain Vanilla ASCII" or other
form.  Any alternate format must include the full Project Gutenberg-tm
License as specified in paragraph 1.E.1.

1.E.7.  Do not charge a fee for access to, viewing, displaying,
performing, copying or distributing any Project Gutenberg-tm works
unless you comply with paragraph 1.E.8 or 1.E.9.

1.E.8.  You may charge a reasonable fee for copies of or providing
access to or distributing Project Gutenberg-tm electronic works provided
that

- You pay a royalty fee of 20% of the gross profits you derive from
     the use of Project Gutenberg-tm works calculated using the method
     you already use to calculate your applicable taxes.  The fee is
     owed to the owner of the Project Gutenberg-tm trademark, but he
     has agreed to donate royalties under this paragraph to the
     Project Gutenberg Literary Archive Foundation.  Royalty payments
     must be paid within 60 days following each date on which you
     prepare (or are legally required to prepare) your periodic tax
     returns.  Royalty payments should be clearly marked as such and
     sent to the Project Gutenberg Literary Archive Foundation at the
     address specified in Section 4, "Information about donations to
     the Project Gutenberg Literary Archive Foundation."

- You provide a full refund of any money paid by a user who notifies
     you in writing (or by e-mail) within 30 days of receipt that s/he
     does not agree to the terms of the full Project Gutenberg-tm
     License.  You must require such a user to return or
     destroy all copies of the works possessed in a physical medium
     and discontinue all use of and all access to other copies of
     Project Gutenberg-tm works.

- You provide, in accordance with paragraph 1.F.3, a full refund of any
     money paid for a work or a replacement copy, if a defect in the
     electronic work is discovered and reported to you within 90 days
     of receipt of the work.

- You comply with all other terms of this agreement for free
     distribution of Project Gutenberg-tm works.

1.E.9.  If you wish to charge a fee or distribute a Project Gutenberg-tm
electronic work or group of works on different terms than are set
forth in this agreement, you must obtain permission in writing from
both the Project Gutenberg Literary Archive Foundation and Michael
Hart, the owner of the Project Gutenberg-tm trademark.  Contact the
Foundation as set forth in Section 3 below.

1.F.

1.F.1.  Project Gutenberg volunteers and employees expend considerable
effort to identify, do copyright research on, transcribe and proofread
public domain works in creating the Project Gutenberg-tm
collection.  Despite these efforts, Project Gutenberg-tm electronic
works, and the medium on which they may be stored, may contain
"Defects," such as, but not limited to, incomplete, inaccurate or
corrupt data, transcription errors, a copyright or other intellectual
property infringement, a defective or damaged disk or other medium, a
computer virus, or computer codes that damage or cannot be read by
your equipment.

1.F.2.  LIMITED WARRANTY, DISCLAIMER OF DAMAGES - Except for the "Right
of Replacement or Refund" described in paragraph 1.F.3, the Project
Gutenberg Literary Archive Foundation, the owner of the Project
Gutenberg-tm trademark, and any other party distributing a Project
Gutenberg-tm electronic work under this agreement, disclaim all
liability to you for damages, costs and expenses, including legal
fees.  YOU AGREE THAT YOU HAVE NO REMEDIES FOR NEGLIGENCE, STRICT
LIABILITY, BREACH OF WARRANTY OR BREACH OF CONTRACT EXCEPT THOSE
PROVIDED IN PARAGRAPH 1.F.3.  YOU AGREE THAT THE FOUNDATION, THE
TRADEMARK OWNER, AND ANY DISTRIBUTOR UNDER THIS AGREEMENT WILL NOT BE
LIABLE TO YOU FOR ACTUAL, DIRECT, INDIRECT, CONSEQUENTIAL, PUNITIVE OR
INCIDENTAL DAMAGES EVEN IF YOU GIVE NOTICE OF THE POSSIBILITY OF SUCH
DAMAGE.

1.F.3.  LIMITED RIGHT OF REPLACEMENT OR REFUND - If you discover a
defect in this electronic work within 90 days of receiving it, you can
receive a refund of the money (if any) you paid for it by sending a
written explanation to the person you received the work from.  If you
received the work on a physical medium, you must return the medium with
your written explanation.  The person or entity that provided you with
the defective work may elect to provide a replacement copy in lieu of a
refund.  If you received the work electronically, the person or entity
providing it to you may choose to give you a second opportunity to
receive the work electronically in lieu of a refund.  If the second copy
is also defective, you may demand a refund in writing without further
opportunities to fix the problem.

1.F.4.  Except for the limited right of replacement or refund set forth
in paragraph 1.F.3, this work is provided to you 'AS-IS' WITH NO OTHER
WARRANTIES OF ANY KIND, EXPRESS OR IMPLIED, INCLUDING BUT NOT LIMITED TO
WARRANTIES OF MERCHANTIBILITY OR FITNESS FOR ANY PURPOSE.

1.F.5.  Some states do not allow disclaimers of certain implied
warranties or the exclusion or limitation of certain types of damages.
If any disclaimer or limitation set forth in this agreement violates the
law of the state applicable to this agreement, the agreement shall be
interpreted to make the maximum disclaimer or limitation permitted by
the applicable state law.  The invalidity or unenforceability of any
provision of this agreement shall not void the remaining provisions.

1.F.6.  INDEMNITY - You agree to indemnify and hold the Foundation, the
trademark owner, any agent or employee of the Foundation, anyone
providing copies of Project Gutenberg-tm electronic works in accordance
with this agreement, and any volunteers associated with the production,
promotion and distribution of Project Gutenberg-tm electronic works,
harmless from all liability, costs and expenses, including legal fees,
that arise directly or indirectly from any of the following which you do
or cause to occur: (a) distribution of this or any Project Gutenberg-tm
work, (b) alteration, modification, or additions or deletions to any
Project Gutenberg-tm work, and (c) any Defect you cause.


Section  2.  Information about the Mission of Project Gutenberg-tm

Project Gutenberg-tm is synonymous with the free distribution of
electronic works in formats readable by the widest variety of computers
including obsolete, old, middle-aged and new computers.  It exists
because of the efforts of hundreds of volunteers and donations from
people in all walks of life.

Volunteers and financial support to provide volunteers with the
assistance they need, are critical to reaching Project Gutenberg-tm's
goals and ensuring that the Project Gutenberg-tm collection will
remain freely available for generations to come.  In 2001, the Project
Gutenberg Literary Archive Foundation was created to provide a secure
and permanent future for Project Gutenberg-tm and future generations.
To learn more about the Project Gutenberg Literary Archive Foundation
and how your efforts and donations can help, see Sections 3 and 4
and the Foundation web page at http://www.pglaf.org.


Section 3.  Information about the Project Gutenberg Literary Archive
Foundation

The Project Gutenberg Literary Archive Foundation is a non profit
501(c)(3) educational corporation organized under the laws of the
state of Mississippi and granted tax exempt status by the Internal
Revenue Service.  The Foundation's EIN or federal tax identification
number is 64-6221541.  Its 501(c)(3) letter is posted at
http://pglaf.org/fundraising.  Contributions to the Project Gutenberg
Literary Archive Foundation are tax deductible to the full extent
permitted by U.S. federal laws and your state's laws.

The Foundation's principal office is located at 4557 Melan Dr. S.
Fairbanks, AK, 99712., but its volunteers and employees are scattered
throughout numerous locations.  Its business office is located at
809 North 1500 West, Salt Lake City, UT 84116, (801) 596-1887, email
business@pglaf.org.  Email contact links and up to date contact
information can be found at the Foundation's web site and official
page at http://pglaf.org

For additional contact information:
     Dr. Gregory B. Newby
     Chief Executive and Director
     gbnewby@pglaf.org


Section 4.  Information about Donations to the Project Gutenberg
Literary Archive Foundation

Project Gutenberg-tm depends upon and cannot survive without wide
spread public support and donations to carry out its mission of
increasing the number of public domain and licensed works that can be
freely distributed in machine readable form accessible by the widest
array of equipment including outdated equipment.  Many small donations
($1 to $5,000) are particularly important to maintaining tax exempt
status with the IRS.

The Foundation is committed to complying with the laws regulating
charities and charitable donations in all 50 states of the United
States.  Compliance requirements are not uniform and it takes a
considerable effort, much paperwork and many fees to meet and keep up
with these requirements.  We do not solicit donations in locations
where we have not received written confirmation of compliance.  To
SEND DONATIONS or determine the status of compliance for any
particular state visit http://pglaf.org

While we cannot and do not solicit contributions from states where we
have not met the solicitation requirements, we know of no prohibition
against accepting unsolicited donations from donors in such states who
approach us with offers to donate.

International donations are gratefully accepted, but we cannot make
any statements concerning tax treatment of donations received from
outside the United States.  U.S. laws alone swamp our small staff.

Please check the Project Gutenberg Web pages for current donation
methods and addresses.  Donations are accepted in a number of other
ways including checks, online payments and credit card donations.
To donate, please visit: http://pglaf.org/donate


Section 5.  General Information About Project Gutenberg-tm electronic
works.

Professor Michael S. Hart is the originator of the Project Gutenberg-tm
concept of a library of electronic works that could be freely shared
with anyone.  For thirty years, he produced and distributed Project
Gutenberg-tm eBooks with only a loose network of volunteer support.


Project Gutenberg-tm eBooks are often created from several printed
editions, all of which are confirmed as Public Domain in the U.S.
unless a copyright notice is included.  Thus, we do not necessarily
keep eBooks in compliance with any particular paper edition.


Most people start at our Web site which has the main PG search facility:

     http://www.gutenberg.org

This Web site includes information about Project Gutenberg-tm,
including how to make donations to the Project Gutenberg Literary
Archive Foundation, how to help produce our new eBooks, and how to
subscribe to our email newsletter to hear about new eBooks.
\end{PGtext}

% %%%%%%%%%%%%%%%%%%%%%%%%%%%%%%%%%%%%%%%%%%%%%%%%%%%%%%%%%%%%%%%%%%%%%%% %
%                                                                         %
% End of Project Gutenberg's The Mathematical Analysis of Logic, by George Boole
%                                                                         %
% *** END OF THIS PROJECT GUTENBERG EBOOK THE MATHEMATICAL ANALYSIS OF LOGIC ***
%                                                                         %
% ***** This file should be named 36884-t.tex or 36884-t.zip *****        %
% This and all associated files of various formats will be found in:      %
%         http://www.gutenberg.org/3/6/8/8/36884/                         %
%                                                                         %
% %%%%%%%%%%%%%%%%%%%%%%%%%%%%%%%%%%%%%%%%%%%%%%%%%%%%%%%%%%%%%%%%%%%%%%% %

\end{document}
###
@ControlwordReplace = (
  ['\\begin{Quote}', ''],
  ['\\end{Quote}', ''],
  ['\\begin{Abstract}', ''],
  ['\\end{Abstract}', ''],
  ['\\end{Rule}', ''],
  ['\\etc', 'etc'],
  ['\\ie', 'i.e.'],
  ['\\eg', 'e.g.']
  );

@ControlwordArguments = (
  ['\\Signature', 1, 1, '', ''],
  ['\\tb', 0, 0, '', ''],
  ['\\BookMark', 1, 0, '', '', 1, 0, '', ''],
  ['\\First', 1, 1, '', ''],
  ['\\Chapter', 1, 1, '', ''],
  ['\\ChapRef', 1, 0, '', '', 1, 1, '', ''],
  ['\\Pagelabel', 1, 0, '', ''],
  ['\\Pageref', 1, 1, 'p. ', ''],
  ['\\Pagerefs', 1, 1, 'pp. ', ', ', 1, 1, '', ''],
  ['\\begin{Rule}', 0, 0, '', ''],
  ['\\Prop', 1, 1, 'Prop. ', ''],
  ['\\Eqref', 0, 0, '', '', 1, 1, '', ''],
  ['\\PropRef', 1, 1, 'Prop. ', ''],
  ['\\Typo', 1, 0, '', '', 1, 1, '', ''],
  ['\\Add', 1, 1, '', ''],
  ['\\Chg', 1, 0, '', '', 1, 1, '', '']
  );
$PageSeparator = qr/^\\PageSep/;
$CustomClean = 'print "\\nCustom cleaning in progress...";
my $cline = 0;
 while ($cline <= $#file) {
   $file[$cline] =~ s/--------[^\n]*\n//; # strip page separators
   $cline++
 }
 print "done\\n";';
###
This is pdfTeXk, Version 3.141592-1.40.3 (Web2C 7.5.6) (format=pdflatex 2010.5.6)  28 JUL 2011 14:43
entering extended mode
 %&-line parsing enabled.
**36884-t.tex
(./36884-t.tex
LaTeX2e <2005/12/01>
Babel <v3.8h> and hyphenation patterns for english, usenglishmax, dumylang, noh
yphenation, arabic, farsi, croatian, ukrainian, russian, bulgarian, czech, slov
ak, danish, dutch, finnish, basque, french, german, ngerman, ibycus, greek, mon
ogreek, ancientgreek, hungarian, italian, latin, mongolian, norsk, icelandic, i
nterlingua, turkish, coptic, romanian, welsh, serbian, slovenian, estonian, esp
eranto, uppersorbian, indonesian, polish, portuguese, spanish, catalan, galicia
n, swedish, ukenglish, pinyin, loaded.
(/usr/share/texmf-texlive/tex/latex/base/book.cls
Document Class: book 2005/09/16 v1.4f Standard LaTeX document class
(/usr/share/texmf-texlive/tex/latex/base/bk12.clo
File: bk12.clo 2005/09/16 v1.4f Standard LaTeX file (size option)
)
\c@part=\count79
\c@chapter=\count80
\c@section=\count81
\c@subsection=\count82
\c@subsubsection=\count83
\c@paragraph=\count84
\c@subparagraph=\count85
\c@figure=\count86
\c@table=\count87
\abovecaptionskip=\skip41
\belowcaptionskip=\skip42
\bibindent=\dimen102
) (/usr/share/texmf-texlive/tex/latex/base/inputenc.sty
Package: inputenc 2006/05/05 v1.1b Input encoding file
\inpenc@prehook=\toks14
\inpenc@posthook=\toks15
(/usr/share/texmf-texlive/tex/latex/base/latin1.def
File: latin1.def 2006/05/05 v1.1b Input encoding file
)) (/usr/share/texmf-texlive/tex/generic/babel/babel.sty
Package: babel 2005/11/23 v3.8h The Babel package
(/usr/share/texmf-texlive/tex/generic/babel/greek.ldf
Language: greek 2005/03/30 v1.3l Greek support from the babel system
(/usr/share/texmf-texlive/tex/generic/babel/babel.def
File: babel.def 2005/11/23 v3.8h Babel common definitions
\babel@savecnt=\count88
\U@D=\dimen103
) Loading the definitions for the Greek font encoding (/usr/share/texmf-texlive
/tex/generic/babel/lgrenc.def
File: lgrenc.def 2001/01/30 v2.2e Greek Encoding
)) (/usr/share/texmf-texlive/tex/generic/babel/english.ldf
Language: english 2005/03/30 v3.3o English support from the babel system
\l@british = a dialect from \language\l@english 
\l@UKenglish = a dialect from \language\l@english 
\l@canadian = a dialect from \language\l@american 
\l@australian = a dialect from \language\l@british 
\l@newzealand = a dialect from \language\l@british 
)) (/usr/share/texmf-texlive/tex/latex/base/ifthen.sty
Package: ifthen 2001/05/26 v1.1c Standard LaTeX ifthen package (DPC)
) (/usr/share/texmf-texlive/tex/latex/amsmath/amsmath.sty
Package: amsmath 2000/07/18 v2.13 AMS math features
\@mathmargin=\skip43
For additional information on amsmath, use the `?' option.
(/usr/share/texmf-texlive/tex/latex/amsmath/amstext.sty
Package: amstext 2000/06/29 v2.01
(/usr/share/texmf-texlive/tex/latex/amsmath/amsgen.sty
File: amsgen.sty 1999/11/30 v2.0
\@emptytoks=\toks16
\ex@=\dimen104
)) (/usr/share/texmf-texlive/tex/latex/amsmath/amsbsy.sty
Package: amsbsy 1999/11/29 v1.2d
\pmbraise@=\dimen105
) (/usr/share/texmf-texlive/tex/latex/amsmath/amsopn.sty
Package: amsopn 1999/12/14 v2.01 operator names
)
\inf@bad=\count89
LaTeX Info: Redefining \frac on input line 211.
\uproot@=\count90
\leftroot@=\count91
LaTeX Info: Redefining \overline on input line 307.
\classnum@=\count92
\DOTSCASE@=\count93
LaTeX Info: Redefining \ldots on input line 379.
LaTeX Info: Redefining \dots on input line 382.
LaTeX Info: Redefining \cdots on input line 467.
\Mathstrutbox@=\box26
\strutbox@=\box27
\big@size=\dimen106
LaTeX Font Info:    Redeclaring font encoding OML on input line 567.
LaTeX Font Info:    Redeclaring font encoding OMS on input line 568.
\macc@depth=\count94
\c@MaxMatrixCols=\count95
\dotsspace@=\muskip10
\c@parentequation=\count96
\dspbrk@lvl=\count97
\tag@help=\toks17
\row@=\count98
\column@=\count99
\maxfields@=\count100
\andhelp@=\toks18
\eqnshift@=\dimen107
\alignsep@=\dimen108
\tagshift@=\dimen109
\tagwidth@=\dimen110
\totwidth@=\dimen111
\lineht@=\dimen112
\@envbody=\toks19
\multlinegap=\skip44
\multlinetaggap=\skip45
\mathdisplay@stack=\toks20
LaTeX Info: Redefining \[ on input line 2666.
LaTeX Info: Redefining \] on input line 2667.
) (/usr/share/texmf-texlive/tex/latex/amsfonts/amssymb.sty
Package: amssymb 2002/01/22 v2.2d
(/usr/share/texmf-texlive/tex/latex/amsfonts/amsfonts.sty
Package: amsfonts 2001/10/25 v2.2f
\symAMSa=\mathgroup4
\symAMSb=\mathgroup5
LaTeX Font Info:    Overwriting math alphabet `\mathfrak' in version `bold'
(Font)                  U/euf/m/n --> U/euf/b/n on input line 132.
)) (/usr/share/texmf-texlive/tex/latex/base/alltt.sty
Package: alltt 1997/06/16 v2.0g defines alltt environment
) (/usr/share/texmf-texlive/tex/latex/tools/array.sty
Package: array 2005/08/23 v2.4b Tabular extension package (FMi)
\col@sep=\dimen113
\extrarowheight=\dimen114
\NC@list=\toks21
\extratabsurround=\skip46
\backup@length=\skip47
) (/usr/share/texmf-texlive/tex/latex/tools/indentfirst.sty
Package: indentfirst 1995/11/23 v1.03 Indent first paragraph (DPC)
) (/usr/share/texmf-texlive/tex/latex/footmisc/footmisc.sty
Package: footmisc 2005/03/17 v5.3d a miscellany of footnote facilities
\FN@temptoken=\toks22
\footnotemargin=\dimen115
\c@pp@next@reset=\count101
\c@@fnserial=\count102
Package footmisc Info: Declaring symbol style bringhurst on input line 817.
Package footmisc Info: Declaring symbol style chicago on input line 818.
Package footmisc Info: Declaring symbol style wiley on input line 819.
Package footmisc Info: Declaring symbol style lamport-robust on input line 823.

Package footmisc Info: Declaring symbol style lamport* on input line 831.
Package footmisc Info: Declaring symbol style lamport*-robust on input line 840
.
) (/usr/share/texmf-texlive/tex/latex/caption/caption.sty
Package: caption 2007/01/07 v3.0k Customising captions (AR)
(/usr/share/texmf-texlive/tex/latex/caption/caption3.sty
Package: caption3 2007/01/07 v3.0k caption3 kernel (AR)
(/usr/share/texmf-texlive/tex/latex/graphics/keyval.sty
Package: keyval 1999/03/16 v1.13 key=value parser (DPC)
\KV@toks@=\toks23
)
\captionmargin=\dimen116
\captionmarginx=\dimen117
\captionwidth=\dimen118
\captionindent=\dimen119
\captionparindent=\dimen120
\captionhangindent=\dimen121
)) (/usr/share/texmf-texlive/tex/latex/tools/calc.sty
Package: calc 2005/08/06 v4.2 Infix arithmetic (KKT,FJ)
\calc@Acount=\count103
\calc@Bcount=\count104
\calc@Adimen=\dimen122
\calc@Bdimen=\dimen123
\calc@Askip=\skip48
\calc@Bskip=\skip49
LaTeX Info: Redefining \setlength on input line 75.
LaTeX Info: Redefining \addtolength on input line 76.
\calc@Ccount=\count105
\calc@Cskip=\skip50
) (/usr/share/texmf-texlive/tex/latex/fancyhdr/fancyhdr.sty
\fancy@headwidth=\skip51
\f@ncyO@elh=\skip52
\f@ncyO@erh=\skip53
\f@ncyO@olh=\skip54
\f@ncyO@orh=\skip55
\f@ncyO@elf=\skip56
\f@ncyO@erf=\skip57
\f@ncyO@olf=\skip58
\f@ncyO@orf=\skip59
) (/usr/share/texmf-texlive/tex/latex/geometry/geometry.sty
Package: geometry 2002/07/08 v3.2 Page Geometry
\Gm@cnth=\count106
\Gm@cntv=\count107
\c@Gm@tempcnt=\count108
\Gm@bindingoffset=\dimen124
\Gm@wd@mp=\dimen125
\Gm@odd@mp=\dimen126
\Gm@even@mp=\dimen127
\Gm@dimlist=\toks24
(/usr/share/texmf-texlive/tex/xelatex/xetexconfig/geometry.cfg)) (/usr/share/te
xmf-texlive/tex/latex/hyperref/hyperref.sty
Package: hyperref 2007/02/07 v6.75r Hypertext links for LaTeX
\@linkdim=\dimen128
\Hy@linkcounter=\count109
\Hy@pagecounter=\count110
(/usr/share/texmf-texlive/tex/latex/hyperref/pd1enc.def
File: pd1enc.def 2007/02/07 v6.75r Hyperref: PDFDocEncoding definition (HO)
) (/etc/texmf/tex/latex/config/hyperref.cfg
File: hyperref.cfg 2002/06/06 v1.2 hyperref configuration of TeXLive
) (/usr/share/texmf-texlive/tex/latex/oberdiek/kvoptions.sty
Package: kvoptions 2006/08/22 v2.4 Connects package keyval with LaTeX options (
HO)
)
Package hyperref Info: Option `hyperfootnotes' set `false' on input line 2238.
Package hyperref Info: Option `bookmarks' set `true' on input line 2238.
Package hyperref Info: Option `linktocpage' set `false' on input line 2238.
Package hyperref Info: Option `pdfdisplaydoctitle' set `true' on input line 223
8.
Package hyperref Info: Option `pdfpagelabels' set `true' on input line 2238.
Package hyperref Info: Option `bookmarksopen' set `true' on input line 2238.
Package hyperref Info: Option `colorlinks' set `true' on input line 2238.
Package hyperref Info: Hyper figures OFF on input line 2288.
Package hyperref Info: Link nesting OFF on input line 2293.
Package hyperref Info: Hyper index ON on input line 2296.
Package hyperref Info: Plain pages OFF on input line 2303.
Package hyperref Info: Backreferencing OFF on input line 2308.
Implicit mode ON; LaTeX internals redefined
Package hyperref Info: Bookmarks ON on input line 2444.
(/usr/share/texmf-texlive/tex/latex/ltxmisc/url.sty
\Urlmuskip=\muskip11
Package: url 2005/06/27  ver 3.2  Verb mode for urls, etc.
)
LaTeX Info: Redefining \url on input line 2599.
\Fld@menulength=\count111
\Field@Width=\dimen129
\Fld@charsize=\dimen130
\Choice@toks=\toks25
\Field@toks=\toks26
Package hyperref Info: Hyper figures OFF on input line 3102.
Package hyperref Info: Link nesting OFF on input line 3107.
Package hyperref Info: Hyper index ON on input line 3110.
Package hyperref Info: backreferencing OFF on input line 3117.
Package hyperref Info: Link coloring ON on input line 3120.
\Hy@abspage=\count112
\c@Item=\count113
)
*hyperref using driver hpdftex*
(/usr/share/texmf-texlive/tex/latex/hyperref/hpdftex.def
File: hpdftex.def 2007/02/07 v6.75r Hyperref driver for pdfTeX
\Fld@listcount=\count114
)
\TmpLen=\skip60
\c@ChapNo=\count115
(./36884-t.aux
LaTeX Font Info:    Try loading font information for LGR+cmr on input line 22.
(/usr/share/texmf-texlive/tex/generic/babel/lgrcmr.fd
File: lgrcmr.fd 2001/01/30 v2.2e Greek Computer Modern
))
\openout1 = `36884-t.aux'.

LaTeX Font Info:    Checking defaults for OML/cmm/m/it on input line 457.
LaTeX Font Info:    ... okay on input line 457.
LaTeX Font Info:    Checking defaults for T1/cmr/m/n on input line 457.
LaTeX Font Info:    ... okay on input line 457.
LaTeX Font Info:    Checking defaults for OT1/cmr/m/n on input line 457.
LaTeX Font Info:    ... okay on input line 457.
LaTeX Font Info:    Checking defaults for OMS/cmsy/m/n on input line 457.
LaTeX Font Info:    ... okay on input line 457.
LaTeX Font Info:    Checking defaults for OMX/cmex/m/n on input line 457.
LaTeX Font Info:    ... okay on input line 457.
LaTeX Font Info:    Checking defaults for U/cmr/m/n on input line 457.
LaTeX Font Info:    ... okay on input line 457.
LaTeX Font Info:    Checking defaults for LGR/cmr/m/n on input line 457.
LaTeX Font Info:    ... okay on input line 457.
LaTeX Font Info:    Checking defaults for PD1/pdf/m/n on input line 457.
LaTeX Font Info:    ... okay on input line 457.
(/usr/share/texmf-texlive/tex/latex/ragged2e/ragged2e.sty
Package: ragged2e 2003/03/25 v2.04 ragged2e Package (MS)
(/usr/share/texmf-texlive/tex/latex/everysel/everysel.sty
Package: everysel 1999/06/08 v1.03 EverySelectfont Package (MS)
LaTeX Info: Redefining \selectfont on input line 125.
)
\CenteringLeftskip=\skip61
\RaggedLeftLeftskip=\skip62
\RaggedRightLeftskip=\skip63
\CenteringRightskip=\skip64
\RaggedLeftRightskip=\skip65
\RaggedRightRightskip=\skip66
\CenteringParfillskip=\skip67
\RaggedLeftParfillskip=\skip68
\RaggedRightParfillskip=\skip69
\JustifyingParfillskip=\skip70
\CenteringParindent=\skip71
\RaggedLeftParindent=\skip72
\RaggedRightParindent=\skip73
\JustifyingParindent=\skip74
)
Package caption Info: hyperref package v6.74m (or newer) detected on input line
 457.
-------------------- Geometry parameters
paper: class default
landscape: --
twocolumn: --
twoside: true
asymmetric: --
h-parts: 9.03374pt, 379.4175pt, 9.03375pt
v-parts: 1.26749pt, 538.85623pt, 1.90128pt
hmarginratio: 1:1
vmarginratio: 2:3
lines: --
heightrounded: --
bindingoffset: 0.0pt
truedimen: --
includehead: true
includefoot: true
includemp: --
driver: pdftex
-------------------- Page layout dimensions and switches
\paperwidth  397.48499pt
\paperheight 542.025pt
\textwidth  379.4175pt
\textheight 476.98244pt
\oddsidemargin  -63.23625pt
\evensidemargin -63.23624pt
\topmargin  -71.0025pt
\headheight 12.0pt
\headsep    19.8738pt
\footskip   30.0pt
\marginparwidth 98.0pt
\marginparsep   7.0pt
\columnsep  10.0pt
\skip\footins  10.8pt plus 4.0pt minus 2.0pt
\hoffset 0.0pt
\voffset 0.0pt
\mag 1000
\@twosidetrue \@mparswitchtrue 
(1in=72.27pt, 1cm=28.45pt)
-----------------------
(/usr/share/texmf-texlive/tex/latex/graphics/color.sty
Package: color 2005/11/14 v1.0j Standard LaTeX Color (DPC)
(/etc/texmf/tex/latex/config/color.cfg
File: color.cfg 2007/01/18 v1.5 color configuration of teTeX/TeXLive
)
Package color Info: Driver file: pdftex.def on input line 130.
(/usr/share/texmf-texlive/tex/latex/pdftex-def/pdftex.def
File: pdftex.def 2007/01/08 v0.04d Graphics/color for pdfTeX
\Gread@gobject=\count116
(/usr/share/texmf/tex/context/base/supp-pdf.tex
[Loading MPS to PDF converter (version 2006.09.02).]
\scratchcounter=\count117
\scratchdimen=\dimen131
\scratchbox=\box28
\nofMPsegments=\count118
\nofMParguments=\count119
\everyMPshowfont=\toks27
\MPscratchCnt=\count120
\MPscratchDim=\dimen132
\MPnumerator=\count121
\everyMPtoPDFconversion=\toks28
)))
Package hyperref Info: Link coloring ON on input line 457.
(/usr/share/texmf-texlive/tex/latex/hyperref/nameref.sty
Package: nameref 2006/12/27 v2.28 Cross-referencing by name of section
(/usr/share/texmf-texlive/tex/latex/oberdiek/refcount.sty
Package: refcount 2006/02/20 v3.0 Data extraction from references (HO)
)
\c@section@level=\count122
)
LaTeX Info: Redefining \ref on input line 457.
LaTeX Info: Redefining \pageref on input line 457.
(./36884-t.out) (./36884-t.out)
\@outlinefile=\write3
\openout3 = `36884-t.out'.


Overfull \hbox (40.57884pt too wide) in paragraph at lines 483--483
[]\OT1/cmtt/m/n/10 *** START OF THIS PROJECT GUTENBERG EBOOK THE MATHEMATICAL A
NALYSIS OF LOGIC ***[] 
 []

LaTeX Font Info:    Try loading font information for U+msa on input line 485.
(/usr/share/texmf-texlive/tex/latex/amsfonts/umsa.fd
File: umsa.fd 2002/01/19 v2.2g AMS font definitions
)
LaTeX Font Info:    Try loading font information for U+msb on input line 485.
(/usr/share/texmf-texlive/tex/latex/amsfonts/umsb.fd
File: umsb.fd 2002/01/19 v2.2g AMS font definitions
) [1

{/var/lib/texmf/fonts/map/pdftex/updmap/pdftex.map}] [2] [1


] [2] [1



] [2] [3

] [4] [5] [6] [7] [8] [9] [10] [11] [12] [13] [14

] [15] [16] [17] [18] [19

] [20] [21] [22] [23] [24] [25] [26

] [27] [28] [29] [30] [31] [32

] [33] [34] [35] [36] [37] [38] [39] [40] [41] [42] [43] [44] [45] [46] [47] [4
8] [49

] [50] [51] [52] [53] [54] [55] [56] [57] [58] [59] [60] [61] [62

] [63] [64] [65] [66] [67] [68] [69] [70] [71] [72] [73

] [74] [75] [76] [77] [78] [79] [80] [81] [82] [83] [84] [85] [86

] [87]
Overfull \hbox (30.07893pt too wide) in paragraph at lines 4612--4612
[]\OT1/cmtt/m/n/10 End of Project Gutenberg's The Mathematical Analysis of Logi
c, by George Boole[] 
 []


Overfull \hbox (30.07893pt too wide) in paragraph at lines 4614--4614
[]\OT1/cmtt/m/n/10 *** END OF THIS PROJECT GUTENBERG EBOOK THE MATHEMATICAL ANA
LYSIS OF LOGIC ***[] 
 []

[1

]
Overfull \hbox (3.82916pt too wide) in paragraph at lines 4681--4681
[]\OT1/cmtt/m/n/10 1.C.  The Project Gutenberg Literary Archive Foundation ("th
e Foundation"[] 
 []


Overfull \hbox (3.82916pt too wide) in paragraph at lines 4686--4686
[]\OT1/cmtt/m/n/10 located in the United States, we do not claim a right to pre
vent you from[] 
 []

[2]
Overfull \hbox (3.82916pt too wide) in paragraph at lines 4691--4691
[]\OT1/cmtt/m/n/10 freely sharing Project Gutenberg-tm works in compliance with
 the terms of[] 
 []

[3]
Overfull \hbox (3.82916pt too wide) in paragraph at lines 4754--4754
[]\OT1/cmtt/m/n/10 posted on the official Project Gutenberg-tm web site (www.gu
tenberg.org),[] 
 []

[4] [5] [6] [7] [8] [9] [10] (./36884-t.aux)

 *File List*
    book.cls    2005/09/16 v1.4f Standard LaTeX document class
    bk12.clo    2005/09/16 v1.4f Standard LaTeX file (size option)
inputenc.sty    2006/05/05 v1.1b Input encoding file
  latin1.def    2006/05/05 v1.1b Input encoding file
   babel.sty    2005/11/23 v3.8h The Babel package
   greek.ldf    2005/03/30 v1.3l Greek support from the babel system
  lgrenc.def    2001/01/30 v2.2e Greek Encoding
 english.ldf    2005/03/30 v3.3o English support from the babel system
  ifthen.sty    2001/05/26 v1.1c Standard LaTeX ifthen package (DPC)
 amsmath.sty    2000/07/18 v2.13 AMS math features
 amstext.sty    2000/06/29 v2.01
  amsgen.sty    1999/11/30 v2.0
  amsbsy.sty    1999/11/29 v1.2d
  amsopn.sty    1999/12/14 v2.01 operator names
 amssymb.sty    2002/01/22 v2.2d
amsfonts.sty    2001/10/25 v2.2f
   alltt.sty    1997/06/16 v2.0g defines alltt environment
   array.sty    2005/08/23 v2.4b Tabular extension package (FMi)
indentfirst.sty    1995/11/23 v1.03 Indent first paragraph (DPC)
footmisc.sty    2005/03/17 v5.3d a miscellany of footnote facilities
 caption.sty    2007/01/07 v3.0k Customising captions (AR)
caption3.sty    2007/01/07 v3.0k caption3 kernel (AR)
  keyval.sty    1999/03/16 v1.13 key=value parser (DPC)
    calc.sty    2005/08/06 v4.2 Infix arithmetic (KKT,FJ)
fancyhdr.sty    
geometry.sty    2002/07/08 v3.2 Page Geometry
geometry.cfg
hyperref.sty    2007/02/07 v6.75r Hypertext links for LaTeX
  pd1enc.def    2007/02/07 v6.75r Hyperref: PDFDocEncoding definition (HO)
hyperref.cfg    2002/06/06 v1.2 hyperref configuration of TeXLive
kvoptions.sty    2006/08/22 v2.4 Connects package keyval with LaTeX options (HO
)
     url.sty    2005/06/27  ver 3.2  Verb mode for urls, etc.
 hpdftex.def    2007/02/07 v6.75r Hyperref driver for pdfTeX
  lgrcmr.fd    2001/01/30 v2.2e Greek Computer Modern
ragged2e.sty    2003/03/25 v2.04 ragged2e Package (MS)
everysel.sty    1999/06/08 v1.03 EverySelectfont Package (MS)
   color.sty    2005/11/14 v1.0j Standard LaTeX Color (DPC)
   color.cfg    2007/01/18 v1.5 color configuration of teTeX/TeXLive
  pdftex.def    2007/01/08 v0.04d Graphics/color for pdfTeX
supp-pdf.tex
 nameref.sty    2006/12/27 v2.28 Cross-referencing by name of section
refcount.sty    2006/02/20 v3.0 Data extraction from references (HO)
 36884-t.out
 36884-t.out
    umsa.fd    2002/01/19 v2.2g AMS font definitions
    umsb.fd    2002/01/19 v2.2g AMS font definitions
 ***********

 ) 
Here is how much of TeX's memory you used:
 5776 strings out of 94074
 77809 string characters out of 1165154
 153122 words of memory out of 1500000
 8611 multiletter control sequences out of 10000+50000
 18517 words of font info for 66 fonts, out of 1200000 for 2000
 645 hyphenation exceptions out of 8191
 34i,22n,43p,258b,497s stack positions out of 5000i,500n,6000p,200000b,5000s
 </home/widger/.texmf-var/fonts/pk/ljfour/public/cb/grmn1200.600pk></usr/shar
e/texmf-texlive/fonts/type1/bluesky/cm/cmbx12.pfb></usr/share/texmf-texlive/fon
ts/type1/bluesky/cm/cmcsc10.pfb></usr/share/texmf-texlive/fonts/type1/bluesky/c
m/cmex10.pfb></usr/share/texmf-texlive/fonts/type1/bluesky/cm/cmmi10.pfb></usr/
share/texmf-texlive/fonts/type1/bluesky/cm/cmmi12.pfb></usr/share/texmf-texlive
/fonts/type1/bluesky/cm/cmmi7.pfb></usr/share/texmf-texlive/fonts/type1/bluesky
/cm/cmmi8.pfb></usr/share/texmf-texlive/fonts/type1/bluesky/cm/cmr10.pfb></usr/
share/texmf-texlive/fonts/type1/bluesky/cm/cmr12.pfb></usr/share/texmf-texlive/
fonts/type1/bluesky/cm/cmr7.pfb></usr/share/texmf-texlive/fonts/type1/bluesky/c
m/cmr8.pfb></usr/share/texmf-texlive/fonts/type1/bluesky/cm/cmsy10.pfb></usr/sh
are/texmf-texlive/fonts/type1/bluesky/cm/cmsy7.pfb></usr/share/texmf-texlive/fo
nts/type1/bluesky/cm/cmsy8.pfb></usr/share/texmf-texlive/fonts/type1/bluesky/cm
/cmti10.pfb></usr/share/texmf-texlive/fonts/type1/bluesky/cm/cmti12.pfb></usr/s
hare/texmf-texlive/fonts/type1/bluesky/cm/cmtt10.pfb></usr/share/texmf-texlive/
fonts/type1/bluesky/ams/msam10.pfb>
Output written on 36884-t.pdf (101 pages, 429585 bytes).
PDF statistics:
 1028 PDF objects out of 1200 (max. 8388607)
 358 named destinations out of 1000 (max. 131072)
 121 words of extra memory for PDF output out of 10000 (max. 10000000)

